\documentclass[a4paper]{report}
\usepackage[utf8]{inputenc}
\usepackage[T1]{fontenc}
\usepackage[french]{babel}
\usepackage{graphicx}
\usepackage{float}
\usepackage{amsmath}
\usepackage{amsfonts}
\usepackage[overload]{empheq} 
\usepackage{lmodern}
\usepackage{pdfpages}
\usepackage{a4wide}
\usepackage[showlabels,sections,floats,textmath,displaymath]{preview}
\usepackage{hyperref}
\usepackage{algorithm2e}
\usepackage{fullwidth}
\usepackage{stmaryrd}
\usepackage{tikz}
\usepackage{enumitem}
\usepackage{bbold}
\usepackage{pgfplots}
\usepackage{ntheorem}
\usepackage{eurosym}
\usepackage{array,multirow,makecell}
\usepackage{cancel}

\usepgfplotslibrary{fillbetween}
\usetikzlibrary{patterns}

\def\singleletter#1{\rotatebox[origin=B]{90}{#1}}
\makeatletter
\def\parseletters#1{\@tfor \@tempa := #1 \do {\kern2pt\singleletter{\@tempa}}}
\makeatother
\def\verticaltext#1{\rotatebox[origin=c]{-90}{\catcode`\~=13\def~{\vbox to 0.33em{}}\parseletters{#1}}}


\newcolumntype{R}[1]{>{\raggedleft\arraybackslash }b{#1}}
\newcolumntype{L}[1]{>{\raggedright\arraybackslash }b{#1}}
\newcolumntype{C}[1]{>{\centering\arraybackslash }b{#1}}

\theoremstyle{break}
\newtheorem{mydef}{Définition}[chapter]
\newtheorem{theo}{Théorème}[chapter]
\newtheorem*{Rem}{Remarque}
\newtheorem*{Proof}{preuve:}
\newtheorem{lemme}{Lemme}[chapter]
\newtheorem{prop}{Proposition}[chapter]
\newtheorem*{propriete}{Propriétés}

\DeclareMathOperator{\divg}{div} %operateur divergence
\DeclareMathOperator{\supp}{supp}
\DeclareMathOperator{\e}{e} %exponentielle



\newcommand{\norm}[1]{\left \Vert {#1} \right \Vert}
\newcommand{\scal}[2]{\left\langle {#1} , {#2} \right\rangle}

\newcommand{\J}[1]{\mathcal{J}(#1)}
\newcommand{\R}{\mathbb{R}}
\newcommand{\p}{\mathbb{P}}

\newcommand{\vu}{\vec{u}}
\newcommand{\vv}{\vec{v}}

\newcounter{exem}
\setcounter{exem}{1}


\newcommand{\note}{$\bullet$ \textbf{Notes: }}
\newcommand{\exemple}[1]{\textbf{Exemple \theexem: #1} \addtocounter{exem}{1} }

\begin{document}

\tableofcontents
\chapter{Expérience aléatoires, probabilités et probabilités conditionnelles.}

\textbf{Mots clés: } Expérience aléatoire, probabilités,... .

\note Reformuler le cours en faisant entrer en jeu la définition d'un
espace probabilisé. 

\section{Vocabulaire, définitions et exemples:}

\begin{mydef}[Expérience aléatoire]
  On appelle \textbf{expérience aléatoire} une expérience dont on ne peut
  prévoir le résultat avec certitude. On appelle \textbf{issue} d'une expérience
  aléatoire un résultat possible de celle-ci. On appelle \textbf{univers}
  l'ensemble de toutes les issues. Un \textbf{evénement élémentaire} est une
  partie de l'univers composé d'une seule issue tandis qu'un \textbf{événement}
  est une partie de l'univers composée de plusieurs issues. 
\end{mydef}

\exemple{Lancé d'un dé à 6 faces}

Dans cette expérience on jette un dé à 6 faces numérotées de 1 à 6 et l'on
observe le numéro obtenu. Cette exprérience sera modélisée par
\textbf{l'univers} $\Omega=\{1,2,3,4,6\}$. Ainsi:
\begin{itemize}[label = $\bullet$, leftmargin=1cm]
\item ``1'', ``2'', ``3'', ``4'', ``5'' et ``6'' sont l'ensemble des issues
  possibles.
\item A : ``Obtenir un nombre pair'' est un \textbf{événement} que l'on peut
  aussi modéliser par l'ensemble $\{2,4,6\}$
\item B : ``Obtenir le nombre 5'' est un \textbf{événement élémentaire} que l'on
  peut aussi modéliser par l'ensemble $\{5\}$
\end{itemize}

\exemple{Lancé d'une pièce de monnaie}

Dans cette expérience on lance une pièce de monnaie à 2 faces ``Pile'' et
``Face'' et on note la face exposée. Ici \textbf{l'univers} $\Omega=\{P,F\}$
n'est composé que de deux issues: ``Pile'' et ``Face''.
\begin{itemize}[label = $\bullet$, leftmargin=1cm]
\item A: ``Le résultat est pile'' et B: ``Le résultat est face'' sont les deux
  \textbf{événements éléméntaires} de l'expérience. On peut aussi les noter en
  langage ensembliste de la manières suivante: A=$\{P\}$ et B=$\{F\}$.
\end{itemize}

\exemple{Tirage d'une carte dans un jeu de 52.}

Ici on tire une carte dans un jeu de 52 (sans joker). Dans ce jeu il y a deux
couleurs, rouge et noire, quatre familles: Pique, Coeur, Carreau et Trêfle et
chaque famille est composée de 13 cartes numérotées de 1 à 10 et comprenant 3
figures: Valet, Dame, Roi.
\begin{itemize}[label = $\bullet$, leftmargin=1cm]
\item Il y a 52 \textbf{issues} possibles.
\item \textbf{L'univers} $\Omega$ est composé des 52 cartes.
\item A: ``La carte tirée est un As'' est un \textbf{événement} composé de 4 \textbf{issues} (les
  4 as du jeu).
\item B: ``La carte tirée est une figure'' est un \textbf{événement} composé de
  12 \textbf{issues}
\item C: ``La carte tirée est un 3 de coeur'' est un \textbf{événement élémentaire} composé de
  une \textbf{issue}
\end{itemize}

\vspace{1\baselineskip}

\textbf{Analogie du vocabulaire probabiliste avec le langage ensembliste}


\begin{table}[h!]
  \small
\hspace{-2cm}
\renewcommand{\arraystretch}{2} %donne la distance entre les lignes%
\begin{tabular}{|L{5cm}|L{5cm}|L{1.5cm}|L{5cm}|}
  \hline
  Langage probabiliste & Langage des ensembles & Notation & Exemple: lancé d'un
                                                            dé \\
  \hline
  A et un \textbf{événement} & A est une partie de $\Omega$ & $A \subset \Omega$
                                                            & A: ``Obtenir un
                                                              nombre pair'':
                                                              $A=\{2,4,6\}$ \\
  \hline
  $ A=\{\omega\}$ est un \textbf{événement élementaire} & $\omega$ appartient à
                                                          $\Omega$
   &$\omega \in \Omega$ &  A: ``Obtenir le chiffre 6'' : A=$\{6\}$ \\
  \hline
  A est un \textbf{événement certain} & A est l'ensemble $\Omega$ & $A = \Omega$
                                                          & A: '' Obtenir un
                                                            chiffre entre 1 et
                                                            6'': A = $\Omega$\\
  \hline
  A est un \textbf{évenement impossible} & A est l'ensemble vide & $A = \emptyset$
                                                        & A: ``Obtenir le
                                                          chiffre 7'' \\
  \hline
  B est \textbf{l'événement contraire} de A & B et A sont complémentaires
                                            & $\scriptstyle{B = \overline{A} = \Omega
                                              \backslash A}$
                                            & A: ``Obtenir un nombre pair'', B: ``Obtenir
                                              un nombre impair'' \\
  \hline
  A implique B & A est inclus dans B & $A \subset B$ & A:''Obtenir le chiffre
                                                       2'', B : ``Obtenir un
                                                       chiffre pair'' \\
  \hline
  E est l'évenement A ou B & E est l'union de A et B & E=$A \cup B$ & A: ``Obtenir un nombre pair'' B: ``Obtenir 1''
                                    \color{red}{E}: ``Obtenir un nombre pair ou le
                                    chiffre 1'' \\
  \hline
  E est l'événement A et B & E est l'intersection de A et B & $E = A \cap B$
                           &  A: ``Obtenir un nombre pair'', B: ``Obtenir un multiple de 3''
                             \color{red}{E}: ``Obtenir un nombre pair et un
                             un multiple de 3'' = ``Obtenir 6'' \\
  \hline
  A et B sont incompatibles & l'intersection de A et B est vide & $A \cap B =
                                                                  \emptyset$
                            & A: ``Obtenir un chiffre pair'', B: ``Obtenir un
                              chiffre impair''\\
  \hline
\end{tabular}
\end{table}

\section{Probabilité}
\textbf{Version lycée}
\begin{mydef}
  Pour certaines expériences aléatoires on peut, sous certaines conditions,
  déterminé la ``chance'' qu'un événement A à pour se réaliser. C'est ce que l'on
  appelle la \textbf{probabilité} ou \textbf{probabilité théorique} de
  l'événement A.

  De manière plus générale, si on note $\Omega = \{e_1, \dots, e_n\}$ l'univers
  d'une expérience aléatoire alors chaque issue possible $e_i$ est affectée
  d'une \textbf{probabilité}, c'est à dire un nombre $p_i$ tel que:
  \[
    0 < p_i < 1 \quad \text{ et } \quad p_1 + p_2 + \dots + p_n = 1
  \]
\end{mydef}

\exemple{}

Reprenons l'exemple du lancé d'un dé parfaitement équilibré, c'est à dire que
chaque face à la même chance d'apparaître. On dit dans ce cas que le dé est non
pipé ou encore non truqué. L'univers de l'expérience est $\Omega: \{1, 2, 3, 4,
5, 6\}$. Chaque face à donc la même \textbf{probabilité théorique}
d'apparaître, à savoir 1 chance sur 6.

Si on note $E_1 :=$ "Obtenir la face 1", on a $E_1=\{1\}$ et si on note
$\mathbb{P}$ l'application de $\Omega$ dans $\R+$ qui associe à chaque événement
de $Omega$ sa \textbf{probabilité théorique} on a :$\mathbb{P}(E_1) =
\frac{1}{6} $

\begin{Rem}
  La probabilité d'un événement A est un nombre réel compris entre 0 et 1 que
  l'on note généralement:
  \begin{itemize}
  \item Sous la forme fractionnaire.
  \item Sous la forme d'un pourcentage.
  \item Sous la forme d'un nombre décimal (généralement arrondi)
  \end{itemize}
\end{Rem}

\begin{mydef}
  On est dans une situation \textbf{d'équiprobabilité} si toutes les issues de
  l'expérience aléatoire ont la même probabilité. 
\end{mydef}

\begin{prop}
  Dans une \textbf{situation d'équiprobabilité} avec un univers à n éléménts,
  la  \textbf{probabilité} de chaque événément élémentaire est $p_i = \frac{1}{n},~ i \in \{1,\dots,n\}$
\end{prop}

\begin{mydef}
  Soit $\Omega$ un univers. La probabilité d'un événement A inclus dans $\Omega$,
  notée $\mathbb{P}(A)$ est la somme des probabilités des événemets élémentaires
  qui le constitue. Ainsi si $\Omega = \{ e_1, \dots, e_n\}$ et $A = \{e_1, e_3,
  e_{n-1}\}$ alors
  \[
    \mathbb{P}(A) = p_1 + p_3 + p_{n-1}
  \]
  où $p_i$ est la probabilité de l'événement éléméntaire $\{e_i\}$
   
\end{mydef}

\begin{prop}
  Dans une situation \textbf{d'équiprobabilité}, la probabilité d'un
  événement A est donnée par:
  \[
    \mathbb{P}(A) = \frac{|A|}{|\Omega|}
  \]
\end{prop}

\begin{mydef}
  Définir une \textbf{loi de probabilité} sur un univers $\Omega$ c'est associer
  à chaque \textbf{événement élémentaire} $\{e_i\}$ une probabilité
  $p_i$ telle que:
  \begin{enumerate}
  \item  $0 < p_i < 1$
  \item $p_1 + p_2 + \dots + p_n = 1$
  \end{enumerate}
\end{mydef}

\exemple{Lancé d'un dé truqué}

On considère un dé truqué de telle façon que la probabilité
d'apparition de chaque face est proportionnelle à la valeur de celle
ci. Donner la loi de probabilité de cette expérience aléatoire.

L'univers $\Omega$ est composé de 6 issues possible: $\Omega = \{
1,2,3,4,5,6 \}$. Voici un tableau regroupant la probabilité de chaque
évenémént élémentaire: 

\begin{center}
  \begin{tabular}{|L{2cm}|C{1cm}|C{1cm}|C{1cm}|C{1cm}|C{1cm}|C{1cm}|}
    \hline
    Valeurs de $\omega$ & 1 & 2 & 3 & 4 & 5 & 6 \\
    \hline
    $\mathbb{P}(\{ \omega \})$ & a & 2 $\times$ a & 3 $\times$ a & 4 $\times$ a
                                        & 5 $\times$ a  & 6 $\times$ a \\
    \hline
  \end{tabular}
\end{center}

On doit avoir d'après la définition d'une loi de probabilité $\sum \limits_{i=1}^6 \mathbb{P}(\{ i \}) = 1$ c'est à
dire:
\[
  a + 2a + 3a +4a +5a +6a = 1 \Leftrightarrow 21a = 1 \Leftrightarrow
  a = \frac{1}{21}
\]

\begin{prop}
  Soit  $\Omega$ un univers et $\mathbb{P}$ l'application qui à chaque événement
  A de $\Omega$ associe sa probabilité. Alors pour tout événement A et
  B de $\Omega$
  \begin{enumerate}
  \item $\mathbb{P}(\overline{A}) = 1 - \mathbb{P}(A)$
  \item $\mathbb{P}(A \cup B) = \mathbb{P}(A) + \mathbb{P}(B) -
    \mathbb{P}(A \cap B)$
  \end{enumerate}
\end{prop}

\begin{theo}[Loi des grands nombres]
  Lorsqu'on répète une expérience aléatoire un grand nombre de fois,
  de façon indépendante, la fréquence de réalisation d'un événement
  tend vers la probabilité d'un événement
\end{theo}

\note Faire un exemple numérique avec un lancé de dès par exemple. 

\section{Probabilité conditionnelle: }
\begin{mydef}
  Soient A un événement tel que $\mathbb{P}(A) > 0$ et B un autre
  événement. On appelle \textbf{probabilité de} B \textbf{conditionnée} par A ou
  \textbf{probabilité} de B \textbf{sachant} A le réel, noté
  $\mathbb{P}_A(B)$, défini par:
  \[
    \mathbb{P}_A(B) = \frac{\mathbb{P}(A \cap B)}{\mathbb{P}(A)}
  \]
\end{mydef}

\begin{Rem}
  Il apparaît tout de suite que $\mathbb{P}_A(A) = 1$
\end{Rem}

\begin{prop}
  Soient A et B deux événements distincts. On suppose que
  $\mathbb{P}(A) > 0$. On a alors les équivalences suivantes:
  \begin{enumerate}[label = (\arabic*)]
  \item $\p_A(B) = \p(B)$
  \item $\p(A \cap B) = \p(A) \p(B)$
  \item $\p_B(A) = \p(A)$
  \end{enumerate}
\end{prop}

\begin{mydef}
  On dit que deux événements A et B sont indépendants si $\p(A \cap B)
  = \p(A)\p(B)$
\end{mydef}

\begin{Rem}
  Attention, indépendance et incompatibilités sont deux notions
  différentes. En effet, si A et B sont deux événements, de
  probabilités non nulles, incompatibles alors $\p(A \cap B) = 0 \neq
  \p(A) \p(B)$
\end{Rem}

\note Parler de l'indépendance pour plus de 2 événements ??

\begin{prop}
  Soient A et B deux événements indépendants. Alors A et $\overline{B}$
  sont indépendants.
\end{prop}

\begin{Proof}
  $\p(A \cap \overline{B}) = \p(A \backslash (A \cap B)) = \p(A) -
  \p(A \cap B) = \p(A) - \p(A)\p(B) = \p(A)(1-\p(B)) = \p(A) \p(\overline{B})$
\end{Proof}

\subsection{Résultats de décomposition}
\begin{prop}
  Soient A et B deux événements avec $\mathbb{P}(A) > 0$ alors $\p(A
  \cap B) = \p_A(B) \p(A)$
\end{prop}

\note Prochaine proposition à faire si on parle avant de Poincaré,
indépendances de n événements, ... .

\begin{prop}
  Soit $(A_i)_{1\leq i \leq n}$ une suite finie d'événements tels que
  $\p(\bigcap \limits_{\scriptscriptstyle{1 \leq i \leq n-1}} A_i) > 0$. On a :
  \[
    \p(\bigcap \limits_{1 \leq i \leq n} A_i) = \p(A_1) \p_{A_1}(A_2)
    \p_{A_1 \cap A_2}(A_3) \dots \p_{A_1 \cap \dots \cap A_{n-1}}(A_n)
  \]
\end{prop}

\note A démontrer ...

\subsection{Formule des probabilités totales}
\begin{mydef}
  Une famille finie $(A_i)_{1\leq i \leq n}$ d'événements deux à deux
  incompatibles et tels que $\bigcup \limits_{1 \leq i \leq n} A_i =
  \Omega$ est appelé \textbf{système complet d'événements}.
\end{mydef}

\begin{prop}
  Soit $(A_i)_{1\leq i \leq n}$ un \textbf{système complet
    d'événements} ayant tous une probabilité stritement
  positive. Alors :
  \[
    \p(A) = \sum \limits_{i=1}^n \p_{A_i}(A)\p(A_i)
  \]
\end{prop}

\note Pour la preuve, considérer $A \cap \Omega = \bigcup \limits_{1
  \leq i \leq n} (A \cap A_i)$

\subsection{Arbres pondérés}

\section{Exercices}

\chapter{Intégrales, Primitives}
`
\textbf{Pré-requis: } Fonctions continues sur un intervalle, dérivée, 

\section{Primitives: vocabulaire, définitions et propriétés}
\begin{mydef}
  Soit f une fonction continue sur un intervalle I. Une fonction F,
  définie sur I, est une \textbf{primitive de la fonction f} sur I si
  F est dérivable et pour tout x de l'intervalle I on a $F'(x) = f$
\end{mydef}

\exemple{}: On considère la fonction f définie sur $\R$ par $f(x) = 6$
pour tout x de $\R$. La fonction F définie sur $\R$ par $F(x) = 6x$
est une primitive de f. En effet, F est dérivable sur $\R$ et $F'(x) =
6 = f(x)$ pour tout x de $\R$. Si on définit maintenant la fonction G
définie sur $\R$ par $G(x) = 6x + 4$, alors G est aussi une primitive
de f sur $\R$. En effet, G est dérivable sur $\R$ et on a $G'(x) = 6 =
f(x)$.

Cet exemple amène donc à la proposition suivante:

\begin{prop}
  Si une fonction admet une primitive sur un
  intervalle I alors elle en admet une infinité.
\end{prop}

\begin{Proof}
  On note F la primitive de f sur I qui existe par hypothèse. On pose
  G la fonction définie sur I par $G(x) = F(x) + k $ où k est une
  constante de $\R$. Alors G est aussi une primitive de f sur I. En
  effet, G est dérivable, car F est dérivable, et $G'(x) = F'(x) =
  f(x)$. 
\end{Proof}

\begin{Rem}
  Toute primitive F de la fonction f est définie à une constante
  près. L'ensemble des primitives G de la fonction f est définie par
  $G(x) = F(x) + k$ où k est un réel. 
\end{Rem}

\begin{Proof}
  On a déjà montré que si f admettait une primitive F sur I alors
  l'ensemble des fonctions G définies sur I par $G(x) = F(x) + k$
  définit un ensemble de primitive de f sur I.

  Il reste donc à montrer que si f admet une deuxième primitive G sur
  I alors G est forcément de la forme $F(x) + k$ où k est un réel.
  Soit G une primitive de f sur I. On note $h:x \mapsto G(x) -
  F(x)$. On a $h'(x) = G'(x) - F'(x) = f(x) - f(x) = 0$. Ainsi pour
  tout x de \textbf{l'intervalle} I $h'(x) = 0$ donc h est constante
  sur I. Ainsi, il existe un réel k, tel que $h(x) = k$ pour tout x de
  $I$. Finalement, on arrive à $G(x) - F(x) = k $ pour tout x de $I$
  et donc $G(x) = F(x) +k$ pour tout x de l'intervalle I.
\end{Proof}

\begin{theo}[Primitives et condition d'égalité en un point]
  Soit f une fonction continue sur un intervalle I. Soient $x_0 \in I$
  et $y_0 \in \R$. Il existe une \textbf{unique} primitive $F_0$ de f sur I
  qui est telle que $F_0(x_0) = y_0$. 
\end{theo}

\begin{Proof}
  \textbf{Existence}:

  On a vu que toutes les primitives de la fonction f sur l'intervalle
  I sont de la forme $x \mapsto F(x) + k$ où F est une primitive de f
  sur I et k un réel. Pour que $F_0$ vérifie la condition imposée on
  doit avoir: $F_0(x_0) = y_0$ ou encore $F_0(x_0) = F(x_0) + k =
  y_0$. En posant $k = y_0 - F(x_0)$ on a bien:
  \[
    F_0(x_0) = F(x_0) + y_0 - F(x_0) = y_0
  \]
  \newline
  \textbf{Unicité: }

  Supposons qu'il existe $F_1$ et $F_2$ définies sur I tels que:
  \begin{enumerate}[label = (\arabic*),leftmargin=2cm]
  \item $F_1$ et $F_2$ sont des primitives de f sur I
  \item $F_1(x_0) = F_2(x_0) = y_0$
  \end{enumerate}
  et montrons que $F_1 = F_2$ sur l'intervalle I tout entier.

  On note h la fonction définie sur I par $h: x \mapsto F_1(x) -
  F_2(x)$. On a $h'(x) = F'_1(x) - F'_2(x) = f(x) - f(x) = 0$ pour
  tout x de l'intervalle I. Ainsi $h(x) = k$ sur I avec k un réel. De
  plus $k = h_(x_0) = F_1(x_0) - F_2(x_0) = y_0 - y_0 =
  0$. Finalement, on obtient pour tout x de l'intervalle I:
  \[
    0 = h(x) = F_1(x) - F_2(x) \Leftrightarrow  F_1(x) = F_2(x)
  \]

  $F_1$ et $F_2$ sont donc égales sur l'intervalle I. Il n'existe
  alors qu'une seule fonction $F_0$ définie sur I vérifiant $F'_0 = f$ et $F_0(x_0) = y_0$
\end{Proof}

\begin{prop}[Conditions d'existence d'une primitive]
  Soit f une fonction définie sur un intervalle I.
  Si f est continue sur I alors f admet une primitive sur I.
\end{prop}

\begin{Proof}
  La preuve sera vue dans la section sur le calcul intégral.
\end{Proof}

\textbf{Primitives et opérations algébriques: }

Dans cette partie on considère deux fonctions f et g définies sur un
intervalle I et admettant chacune une primitive sur I notée F et G
respectivement. On a alors les règles suivantes:
\begin{enumerate}[label = (\arabic*), leftmargin = 2cm]
\item La fonction (f+g) admet pour primitive sur I la fonction F+G. La
  primitive d'une somme est donc la somme des primitives.
\item Pour tout réel k, la primitive de la fonction $k \times f$ sur I
  est la fonction $k \times F$
\end{enumerate}

\begin{Rem}
  Attention, la primitive d'un produit n'est pas le produit des
  primitives. En effet, $(F \times G)' \neq F' G'$
\end{Rem}

\note Les 2 règles ci-dessus ainsi que la maitrîse des formules de
dérivations suffisent en général pour calculer la primitive d'une
fonction. 

\section{Intégrale d'une fonction continue}
\subsection{Notions d'unité d'aires}

\begin{mydef}
  Soit le repère orthogonal $\Big( O; \vec{i}; \vec{j} \Big) $ On note
  I le point de coordonnées (1,0), J le point de coordonnées (0,1), et
  K le point de coordonnées (1,1) dans ce repère.
  
  On appelle \textbf{unité d'aire} l'aire du rectangle OIKJ et on le
  note \textbf{u.a.}
\end{mydef}

\exemple{}

\begin{minipage}{0.45\linewidth}
  \centering
  \begin{tikzpicture}
    \begin{axis}[
      width = 5cm,
      height = 3cm,
      xmax = 5,
      xmin = -4,
      ymin = -2,
      ymax = 2.4,
      yscale = 2,
      xscale = 1.,
      xtick distance = 1,
      ytick distance = 1,
      xticklabels = {},
      yticklabels = {},
      axis x line= center,
      axis y line= center        
      ]
      \addplot [domain=0:1, samples=20, fill=red!50!white] {1}
      \closedcycle;
      \addplot [->,very thick] coordinates {( 0,0)(1,0)};
      \addplot [->,very thick] coordinates {( 0,0)(0,1)};
      \node at (axis cs: 0,0) [below left] {\tiny O};
      \node at (axis cs: 1,0) [below] {\tiny I};
      \node at (axis cs: 0.8,1) [right] {\tiny K};
      \node at (axis cs: 0,1) [left] {\tiny J};
      \node at (axis cs: 0.5,0) [below] {$\scriptscriptstyle{\vec{i}}$};
      \node at (axis cs: 0,0.5) [left] {$\scriptscriptstyle{\vec{j}}$};
      \addplot [<->,thick] coordinates {(
        -2,-0.1)(-1,-0.1)};
      \addplot [<->,thick] coordinates {(
        -0.1,-2)(-0.1,-1)};
      \node at (axis cs: -1.5,-0.1) [below] {1cm};
      \node at (axis cs: -0.1,-1.5) [left] {2cm};

      \addplot [domain=2:4, samples=20, fill=blue!50!white] {2}
      \closedcycle;
      \node at (axis cs: 2,0) [below] {\tiny A};
      \node at (axis cs: 4,0) [below] {\tiny B};
      \node at (axis cs: 4,2) [above] {\tiny C};
      \node at (axis cs: 2,2) [above] {\tiny D};
    \end{axis}
  \end{tikzpicture}
\end{minipage}
\hfill
\begin{minipage}{0.45\linewidth}
  Dans cet exemple on a 1 \textbf{u.a} $= 1 \times 2 = 2cm^2$
  Ainsi on à l'aire du rectangle ABCD qui est:
  \[
    \mathcal{A}(ABCD) = 2u.x*2u.y = 4 \text{ \textbf{ua} } = 8 cm^2
  \]
  où u.x et u.y représente l'unité des abscisses et u.y l'unité des
  ordonnées. 
\end{minipage}

\subsubsection{Approche de la notion d'intégrale}

\begin{minipage}{0.45\linewidth}
\begin{center}
  \begin{tikzpicture}
    \begin{axis}[
      width = 8cm,
      height = 6cm,
      xmax = 5,
      xmin = -1,
      ymin = -1,
      ymax = 3,
      xtick distance = 1,
      ytick distance = 1,
      xticklabels = {},
      yticklabels = {},
      axis x line= center,
      axis y line= center        
      ]

      %\addplot[domain=1:1.9, samples=20, fill=orange!30!white] {-1/x + 2};

      \addplot[name path = g,domain=1:4, samples=20] {-1/x + 2};

      \addplot[domain=0:1.9] {-1/1.9+2};
      
      \addplot[domain=1.9:2.1,
      samples=50,fill=green!50!white, postaction={pattern=north east lines}]
      {-1/1.9 + 2}\closedcycle;

      \node at (axis cs: 0,0) [below left] {O};
      \addplot [->,very thick] coordinates {( 0,0)(1,0)};
      \addplot [->,very thick] coordinates {( 0,0)(0,1)};
      
            
      \node at (axis cs: 0.5,0) [below] {$\scriptscriptstyle{\vec{i}}$};
      \node at (axis cs: 0,0.5) [left]
      {$\scriptscriptstyle{\vec{j}}$};

      \path[name path=axis] (axis cs:1,0) -- (axis cs:4,0);
      
      \addplot [thick,fill=orange!30!white]
      fill between [ of=g and axis];

      \node at (axis cs: 0,{-1/1.9+2}) {$\bullet$};
      \node at (axis cs: 0,{-1/1.9+2}) [left] {\small f(x)};

      \addplot [<->] coordinates {(1.9,-0.1)(2.1,-0.1)};
      \node at (axis cs: 1.95,-0.1) [below] {\small h};

      \node at (axis cs: 4,0) [below] {\small b};
      \node at (axis cs: 1,0) [below] {\small a};
    \end{axis}
  \end{tikzpicture}
\end{center}
\end{minipage}
\begin{minipage}{0.45\linewidth}
  On appelle $\mathcal{A}$ l'aire de la surface orange situé sous la
  courbe et mesurée en unité d'aire.

  \vspace{1\baselineskip}
  
  L'aire du rectangle vert sur la figure est égale à $h \times
  f(x)$. On pourrait approcher l'aire de la surface orange en
  recouvrant celle-ci par des rectangles de ce type pour x variant de
  a à b et en sommant leurs aires. Plus h deviendra \og petit\fg{}
  plus la somme des aires des rectangles sera \og proche\fg{} de $\mathcal{A}$.
\end{minipage}

Si l'on exprime ça de manière \og plus mathématiques\fg{} on dira que
la somme des h*f(x) tend vers $\mathcal{A}$ lorsque h tend vers 0,
pour x allant de a à b.

Cette limite de somme est notée avec le symbole $\int$ qui se lit
intégrale. Les bornes de l'intervalle sont appelés bornes de
l'intégrale et notées :$\int_a^b$.

\subsection{Intégrale d'une fonction continue positive: }
\begin{mydef}
  On appelle \textbf{hypographe} d'une fonction f défini sur un
  intervalle I l'ensemble noté $hyp f$:
  \[
    hyp f = \{ (x,\alpha) \in I \times \R | ~f(x) \geq \alpha \}
  \]
\end{mydef}

Ainsi, pour une fonction f continue et positive sur un intervalle I=(a,b),
l'hypographe est la partie du plan délimitée par:
\begin{itemize}
\item la courbe représentative de f
\item l'axe des abscisses
\item la droite d'équation x = a
\item la droite d'équation x = b
\end{itemize}

\begin{mydef}
  Soient $a,b \in \R$ avec a<b. Soit $f:[a,b] \to \R$  une fonction
  positive et continue. On appelle intégrale de a à b et on note
  $\int_a^b f(t) dt$ la mesure de l'aire de l'hypographe de f défini
  ci-dessus. 
\end{mydef} 

\begin{Rem}
  \begin{enumerate}[label = (\alph*), leftmargin=2cm]
  \item a et b sont appelés bornes de l'intégrale.
  \item t est appelé la variable d'intégration qui est une variable
    muette. Elle peut donc être remplacé par n'importe quelle lettre
    et ne sert qu'à réaliser les calculs.
  \end{enumerate}
\end{Rem}

\exemple{}

\subsection{Intégrale d'une fonction continue négative: }
\begin{mydef}
  On appelle \textbf{épigraphe} d'une fonction f défini sur un
  intervalle I l'ensemble noté $epi f$:
  \[
    epi f = \{ (x,\alpha) \in I \times \R | ~f(x) \leq \alpha \}
  \]
\end{mydef}

Ainsi, pour une fonction f continue et négative sur un intervalle I=(a,b),
l'épigraphe est la partie du plan délimitée par:
\begin{itemize}
\item la courbe représentative de f
\item l'axe des abscisses
\item la droite d'équation x = a
\item la droite d'équation x = b
\end{itemize}

\begin{mydef}
  Soient $a,b \in \R$ avec a<b. Soit $f:[a,b] \to \R$  une fonction
  négative et continue. On appelle intégrale de a à b et on note
  $\int_a^b f(t) dt$ l'opposé de la mesure de l'aire de l'épigraphe de f défini
  ci-dessus. 
\end{mydef} 

\subsection{Intégrale d'une fonction continue}
\begin{mydef}
  Soient $a,b \in \R$ avec a<b. Soit $f:[a,b] \to \R$  une fonction
  continue. On appelle intégrale de a à b et on note
  $\int_a^b f(t) dt$ la différence entre:
  \begin{itemize}[label=$\bullet$, leftmargin=2cm]
  \item La mesure de l'aire de $f^+=max(0,f)$ et
  \item la mesure de l'aire de $f^-=max(-f,0)$
  \end{itemize}
\end{mydef}

\note Graphique
\section{Lien entre intégrale et primitives}
\begin{theo}
  Soit f une fonction continue sur un intervalle I. La fonction F
  définie sur I par $\int_a^x f(t) dt$ est l'unique primitive de la fonction
  f s'annulant en a. Si G est une primitive de f quelconque on a :
  \[
    \int_a^b f(t) dt = G(b) - G(a)
  \]
\end{theo}

\begin{Proof}
  \textbf{Cas monotone: f croissante}

  Calculons la limite du taux d'accroissement de F. On a :
  \[
    T_h(F) = \frac{F(x+h)-F(x)}{h} = \frac{\int_x^{x+h} f(t) dt}{h}
  \]

  Supposons (sans perte de généralité) que h>0. Ainsi on a $\cancel{h}
  \frac{f(x)}{\cancel{h}} \leq T_h(F) \leq \cancel h \frac{f(x+h)}{h}$
  car f est croissante. Puisque f est continue on a $\lim
  \limits_{h\to 0} f(x+h) = f(x)$ et donc d'après le théorème des
  gendarmes on a:
  \[
    \lim \limits_{h \to 0} T_h(F) = f(x)
  \]
  Et donc par définition du nombre dérivée, $F'(x) = f(x)$ (F(a) = 0 OK)

  De ce fait on a, $F(b) = \int_a^b f(t) dt$. Puisque $F(a) = 0$ on a
  aussi $F(b)-F(a) = \int_a^b f(t)dt$. Si G est une autre primitive de
  f, la formule est également vraie car $G(x) = F(x)+k$ pour tout x de
  I avec k constante réel.

  \textbf{Cas général}
  
  Utilisons la continuité de f. On se fixe $x \in I$. Puisque f est continue on a:
  \[
    \forall \varepsilon>0, ~ \exists \eta>0 ~\big( \forall t \in I,~ |x-t|<\eta
    \Rightarrow |f(x)-f(t)|<\varepsilon \big)
  \]

  Soit $\varepsilon >0$. Prenons h t.q. $|h|<=\frac{\eta}{2}$. On a donc
  \[
    f(x) - \varepsilon < f(t) < f(x) + \varepsilon ~ \forall t \in [x,x+h]
  \]

  Ainsi $f(x)-\epsilon < T_h(F) < f(x)+\varepsilon \Leftrightarrow
  |T_h(F) - f(x)| < \varepsilon $. Finalement on a montré:
  \[
    \forall \varepsilon > 0, ~ \exists \eta>0,~\big( \forall h ~ |h| < \eta
    \Rightarrow |T_h(F) - f(x)|<\varepsilon \big) 
  \]
  
\end{Proof}

%%% Local Variables:
%%% mode: latex
%%% TeX-master: "Experience_aleatoire"
%%% End:

\chapter{Produit scalaire}

\section{Produit scalaire de deux vecteurs: vocabulaire et définition}
\begin{mydef}
  Soient $\vec{u}$ et $\vec{v}$ deux vecteurs. On appelle produit scalaire de
  $\vec{u}$ et $\vec{v}$, et on note $\vu \cdot \vv$, le nombre:
  \[
    \vu \cdot \vv = \norm{\vu} \norm{\vv} \cos(\vu,\vv)
  \]
\end{mydef}

\note insertion dessin

\begin{Rem}
  \begin{itemize}[label=$\bullet$, leftmargin=2cm]
  \item \textbf{Attention: } Le produit scalaire de deux vecteurs est un nombre
    réel et non un vecteur
  \item Le produit scalaire de deux vecteurs est proportionnel à la norme de ces
    deux vecteurs ainsi qu'à l'angle formé par les deux.
  \end{itemize}
\end{Rem}

\begin{propriete}[Expression analytique du produit scalaire]
  Soient $\vec{u}$ et $\vec{v}$ deux vecteurs ayant pour coordonnées $(x,y)$ et
  $(x',y')$  respectivement dans un repère orthonormé. On a:
  \[
    \vec{u} \vec{v} = xx' + yy'
  \]
\end{propriete}

\begin{mydef}[Carré scalaire]
  Soit $\vu$ un vecteur du plan. On appelle \textbf{carré scalaire} de $\vu$ le
  nombre réel noté $\vu^2$ et défini par:
  \[
    \vu^2 = \vu \cdot \vu = \norm{u}^2
  \]
\end{mydef}

\section{Produit scalaire et orthogonalité}
\begin{mydef}[Orthogonalité de deux vecteurs]
  Soient $\vu$ et $\vv$ deux vecteurs non nuls du plan. Soient A, B et C trois
  points tels que $\vec{AB} = \vec{u}$ et $\vec{AC} = \vv$. Les vecteurs $\vu$
  et $\vv$ sont dit \textbf{orthogonaux} si les droites $(AB)$ et $(BC)$ sont
  perpendiculaires. On note alors: $\vu \perp \vv$
\end{mydef}
\section{Applications}


%%% Local Variables:
%%% mode: latex
%%% TeX-master: "Experience_aleatoire"
%%% End:


\end{document}
%%% Local Variables:
%%% mode: latex
%%% TeX-master: t
%%% End:
