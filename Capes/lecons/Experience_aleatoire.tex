\documentclass[a4paper]{report}
\usepackage[utf8]{inputenc}
\usepackage[T1]{fontenc}
\usepackage[french]{babel}
\usepackage{graphicx}
\usepackage{float}
\usepackage{amsmath}
\usepackage{amsfonts}
\usepackage[overload]{empheq} 
\usepackage{lmodern}
\usepackage{pdfpages}
\usepackage{a4wide}
\usepackage[showlabels,sections,floats,textmath,displaymath]{preview}
\usepackage{hyperref}
\usepackage{algorithm2e}
\usepackage{fullwidth}
\usepackage{stmaryrd}
\usepackage{tikz}
\usepackage{enumitem}
\usepackage{bbold}
\usepackage{pgfplots}
\usepackage{ntheorem}
\usepackage{eurosym}
\usepackage{array,multirow,makecell}
\def\singleletter#1{\rotatebox[origin=B]{90}{#1}}
\makeatletter
\def\parseletters#1{\@tfor \@tempa := #1 \do {\kern2pt\singleletter{\@tempa}}}
\makeatother
\def\verticaltext#1{\rotatebox[origin=c]{-90}{\catcode`\~=13\def~{\vbox to 0.33em{}}\parseletters{#1}}}


\newcolumntype{R}[1]{>{\raggedleft\arraybackslash }b{#1}}
\newcolumntype{L}[1]{>{\raggedright\arraybackslash }b{#1}}
\newcolumntype{C}[1]{>{\centering\arraybackslash }b{#1}}

\theoremstyle{break}
\newtheorem{mydef}{Définition}[chapter]
\newtheorem{theo}{Théorème}[chapter]
\newtheorem*{Rem}{Remarque}
\newtheorem*{Proof}{preuve:}
\newtheorem{lemme}{Lemme}[chapter]
\newtheorem{prop}{Proposition}[chapter]
\newtheorem*{propriete}{Propriétés}

\DeclareMathOperator{\divg}{div} %operateur divergence
\DeclareMathOperator{\supp}{supp}
\DeclareMathOperator{\e}{e} %exponentielle



\newcommand{\norm}[1]{\left \Vert {#1} \right \Vert}
\newcommand{\scal}[2]{\left\langle {#1} , {#2} \right\rangle}

\newcommand{\J}[1]{\mathcal{J}(#1)}
\newcommand{\R}{\mathbb{R}}
\newcommand{\p}{\mathbb{P}}

\newcounter{exem}
\setcounter{exem}{1}


\newcommand{\note}{$\bullet$ \textbf{Notes: }}
\newcommand{\exemple}[1]{\textbf{Exemple \theexem: #1} \addtocounter{exem}{1} }

\begin{document}

\tableofcontents
\chapter{Expérience aléatoires, probabilités et probabilités conditionnelles.}

\textbf{Mots clés: } Expérience aléatoire, probabilités,... .

\note Reformuler le cours en faisant entrer en jeu la définition d'un
espace probabilisé. 

\section{Vocabulaire, définitions et exemples:}

\begin{mydef}[Expérience aléatoire]
  On appelle \textbf{expérience aléatoire} une expérience dont on ne peut
  prévoir le résultat avec certitude. On appelle \textbf{issue} d'une expérience
  aléatoire un résultat possible de celle-ci. On appelle \textbf{univers}
  l'ensemble de toutes les issues. Un \textbf{evénement élémentaire} est une
  partie de l'univers composé d'une seule issue tandis qu'un \textbf{événement}
  est une partie de l'univers composée de plusieurs issues. 
\end{mydef}

\exemple{Lancé d'un dé à 6 faces}

Dans cette expérience on jette un dé à 6 faces numérotées de 1 à 6 et l'on
observe le numéro obtenu. Cette exprérience sera modélisée par
\textbf{l'univers} $\Omega=\{1,2,3,4,6\}$. Ainsi:
\begin{itemize}[label = $\bullet$, leftmargin=1cm]
\item ``1'', ``2'', ``3'', ``4'', ``5'' et ``6'' sont l'ensemble des issues
  possibles.
\item A : ``Obtenir un nombre pair'' est un \textbf{événement} que l'on peut
  aussi modéliser par l'ensemble $\{2,4,6\}$
\item B : ``Obtenir le nombre 5'' est un \textbf{événement élémentaire} que l'on
  peut aussi modéliser par l'ensemble $\{5\}$
\end{itemize}

\exemple{Lancé d'une pièce de monnaie}

Dans cette expérience on lance une pièce de monnaie à 2 faces ``Pile'' et
``Face'' et on note la face exposée. Ici \textbf{l'univers} $\Omega=\{P,F\}$
n'est composé que de deux issues: ``Pile'' et ``Face''.
\begin{itemize}[label = $\bullet$, leftmargin=1cm]
\item A: ``Le résultat est pile'' et B: ``Le résultat est face'' sont les deux
  \textbf{événements éléméntaires} de l'expérience. On peut aussi les noter en
  langage ensembliste de la manières suivante: A=$\{P\}$ et B=$\{F\}$.
\end{itemize}

\exemple{Tirage d'une carte dans un jeu de 52.}

Ici on tire une carte dans un jeu de 52 (sans joker). Dans ce jeu il y a deux
couleurs, rouge et noire, quatre familles: Pique, Coeur, Carreau et Trêfle et
chaque famille est composée de 13 cartes numérotées de 1 à 10 et comprenant 3
figures: Valet, Dame, Roi.
\begin{itemize}[label = $\bullet$, leftmargin=1cm]
\item Il y a 52 \textbf{issues} possibles.
\item \textbf{L'univers} $\Omega$ est composé des 52 cartes.
\item A: ``La carte tirée est un As'' est un \textbf{événement} composé de 4 \textbf{issues} (les
  4 as du jeu).
\item B: ``La carte tirée est une figure'' est un \textbf{événement} composé de
  12 \textbf{issues}
\item C: ``La carte tirée est un 3 de coeur'' est un \textbf{événement élémentaire} composé de
  une \textbf{issue}
\end{itemize}

\vspace{1\baselineskip}

\textbf{Analogie du vocabulaire probabiliste avec le langage ensembliste}


\begin{table}[h!]
  \small
\hspace{-2cm}
\renewcommand{\arraystretch}{2} %donne la distance entre les lignes%
\begin{tabular}{|L{5cm}|L{5cm}|L{1.5cm}|L{5cm}|}
  \hline
  Langage probabiliste & Langage des ensembles & Notation & Exemple: lancé d'un
                                                            dé \\
  \hline
  A et un \textbf{événement} & A est une partie de $\Omega$ & $A \subset \Omega$
                                                            & A: ``Obtenir un
                                                              nombre pair'':
                                                              $A=\{2,4,6\}$ \\
  \hline
  $ A=\{\omega\}$ est un \textbf{événement élementaire} & $\omega$ appartient à
                                                          $\Omega$
   &$\omega \in \Omega$ &  A: ``Obtenir le chiffre 6'' : A=$\{6\}$ \\
  \hline
  A est un \textbf{événement certain} & A est l'ensemble $\Omega$ & $A = \Omega$
                                                          & A: '' Obtenir un
                                                            chiffre entre 1 et
                                                            6'': A = $\Omega$\\
  \hline
  A est un \textbf{évenement impossible} & A est l'ensemble vide & $A = \emptyset$
                                                        & A: ``Obtenir le
                                                          chiffre 7'' \\
  \hline
  B est \textbf{l'événement contraire} de A & B et A sont complémentaires
                                            & $\scriptstyle{B = \overline{A} = \Omega
                                              \backslash A}$
                                            & A: ``Obtenir un nombre pair'', B: ``Obtenir
                                              un nombre impair'' \\
  \hline
  A implique B & A est inclus dans B & $A \subset B$ & A:''Obtenir le chiffre
                                                       2'', B : ``Obtenir un
                                                       chiffre pair'' \\
  \hline
  E est l'évenement A ou B & E est l'union de A et B & E=$A \cup B$ & A: ``Obtenir un nombre pair'' B: ``Obtenir 1''
                                    \color{red}{E}: ``Obtenir un nombre pair ou le
                                    chiffre 1'' \\
  \hline
  E est l'événement A et B & E est l'intersection de A et B & $E = A \cap B$
                           &  A: ``Obtenir un nombre pair'', B: ``Obtenir un multiple de 3''
                             \color{red}{E}: ``Obtenir un nombre pair et un
                             un multiple de 3'' = ``Obtenir 6'' \\
  \hline
  A et B sont incompatibles & l'intersection de A et B est vide & $A \cap B =
                                                                  \emptyset$
                            & A: ``Obtenir un chiffre pair'', B: ``Obtenir un
                              chiffre impair''\\
  \hline
\end{tabular}
\end{table}

\section{Probabilité}
\textbf{Version lycée}
\begin{mydef}
  Pour certaines expériences aléatoires on peut, sous certaines conditions,
  déterminé la ``chance'' qu'un événement A à pour se réaliser. C'est ce que l'on
  appelle la \textbf{probabilité} ou \textbf{probabilité théorique} de
  l'événement A.

  De manière plus générale, si on note $\Omega = \{e_1, \dots, e_n\}$ l'univers
  d'une expérience aléatoire alors chaque issue possible $e_i$ est affectée
  d'une \textbf{probabilité}, c'est à dire un nombre $p_i$ tel que:
  \[
    0 < p_i < 1 \quad \text{ et } \quad p_1 + p_2 + \dots + p_n = 1
  \]
\end{mydef}

\exemple{}

Reprenons l'exemple du lancé d'un dé parfaitement équilibré, c'est à dire que
chaque face à la même chance d'apparaître. On dit dans ce cas que le dé est non
pipé ou encore non truqué. L'univers de l'expérience est $\Omega: \{1, 2, 3, 4,
5, 6\}$. Chaque face à donc la même \textbf{probabilité théorique}
d'apparaître, à savoir 1 chance sur 6.

Si on note $E_1 :=$ "Obtenir la face 1", on a $E_1=\{1\}$ et si on note
$\mathbb{P}$ l'application de $\Omega$ dans $\R+$ qui associe à chaque événement
de $Omega$ sa \textbf{probabilité théorique} on a :$\mathbb{P}(E_1) =
\frac{1}{6} $

\begin{Rem}
  La probabilité d'un événement A est un nombre réel compris entre 0 et 1 que
  l'on note généralement:
  \begin{itemize}
  \item Sous la forme fractionnaire.
  \item Sous la forme d'un pourcentage.
  \item Sous la forme d'un nombre décimal (généralement arrondi)
  \end{itemize}
\end{Rem}

\begin{mydef}
  On est dans une situation \textbf{d'équiprobabilité} si toutes les issues de
  l'expérience aléatoire ont la même probabilité. 
\end{mydef}

\begin{prop}
  Dans une \textbf{situation d'équiprobabilité} avec un univers à n éléménts,
  la  \textbf{probabilité} de chaque événément élémentaire est $p_i = \frac{1}{n},~ i \in \{1,\dots,n\}$
\end{prop}

\begin{mydef}
  Soit $\Omega$ un univers. La probabilité d'un événement A inclus dans $\Omega$,
  notée $\mathbb{P}(A)$ est la somme des probabilités des événemets élémentaires
  qui le constitue. Ainsi si $\Omega = \{ e_1, \dots, e_n\}$ et $A = \{e_1, e_3,
  e_{n-1}\}$ alors
  \[
    \mathbb{P}(A) = p_1 + p_3 + p_{n-1}
  \]
  où $p_i$ est la probabilité de l'événement éléméntaire $\{e_i\}$
   
\end{mydef}

\begin{prop}
  Dans une situation \textbf{d'équiprobabilité}, la probabilité d'un
  événement A est donnée par:
  \[
    \mathbb{P}(A) = \frac{|A|}{|\Omega|}
  \]
\end{prop}

\begin{mydef}
  Définir une \textbf{loi de probabilité} sur un univers $\Omega$ c'est associer
  à chaque \textbf{événement élémentaire} $\{e_i\}$ une probabilité
  $p_i$ telle que:
  \begin{enumerate}
  \item  $0 < p_i < 1$
  \item $p_1 + p_2 + \dots + p_n = 1$
  \end{enumerate}
\end{mydef}

\exemple{Lancé d'un dé truqué}

On considère un dé truqué de telle façon que la probabilité
d'apparition de chaque face est proportionnelle à la valeur de celle
ci. Donner la loi de probabilité de cette expérience aléatoire.

L'univers $\Omega$ est composé de 6 issues possible: $\Omega = \{
1,2,3,4,5,6 \}$. Voici un tableau regroupant la probabilité de chaque
évenémént élémentaire: 

\begin{center}
  \begin{tabular}{|L{2cm}|C{1cm}|C{1cm}|C{1cm}|C{1cm}|C{1cm}|C{1cm}|}
    \hline
    Valeurs de $\omega$ & 1 & 2 & 3 & 4 & 5 & 6 \\
    \hline
    $\mathbb{P}(\{ \omega \})$ & a & 2 $\times$ a & 3 $\times$ a & 4 $\times$ a
                                        & 5 $\times$ a  & 6 $\times$ a \\
    \hline
  \end{tabular}
\end{center}

On doit avoir d'après la définition d'une loi de probabilité $\sum \limits_{i=1}^6 \mathbb{P}(\{ i \}) = 1$ c'est à
dire:
\[
  a + 2a + 3a +4a +5a +6a = 1 \Leftrightarrow 21a = 1 \Leftrightarrow
  a = \frac{1}{21}
\]

\begin{prop}
  Soit  $\Omega$ un univers et $\mathbb{P}$ l'application qui à chaque événement
  A de $\Omega$ associe sa probabilité. Alors pour tout événement A et
  B de $\Omega$
  \begin{enumerate}
  \item $\mathbb{P}(\overline{A}) = 1 - \mathbb{P}(A)$
  \item $\mathbb{P}(A \cup B) = \mathbb{P}(A) + \mathbb{P}(B) -
    \mathbb{P}(A \cap B)$
  \end{enumerate}
\end{prop}

\begin{theo}[Loi des grands nombres]
  Lorsqu'on répète une expérience aléatoire un grand nombre de fois,
  de façon indépendante, la fréquence de réalisation d'un événement
  tend vers la probabilité d'un événement
\end{theo}

\note Faire un exemple numérique avec un lancé de dès par exemple. 

\section{Probabilité conditionnelle: }
\begin{mydef}
  Soient A un événement tel que $\mathbb{P}(A) > 0$ et B un autre
  événement. On appelle \textbf{probabilité de} B \textbf{conditionnée} par A ou
  \textbf{probabilité} de B \textbf{sachant} A le réel, noté
  $\mathbb{P}_A(B)$, défini par:
  \[
    \mathbb{P}_A(B) = \frac{\mathbb{P}(A \cap B)}{\mathbb{P}(A)}
  \]
\end{mydef}

\begin{Rem}
  Il apparaît tout de suite que $\mathbb{P}_A(A) = 1$
\end{Rem}

\begin{prop}
  Soient A et B deux événements distincts. On suppose que
  $\mathbb{P}(A) > 0$. On a alors les équivalences suivantes:
  \begin{enumerate}[label = (\arabic*)]
  \item $\p_A(B) = \p(B)$
  \item $\p(A \cap B) = \p(A) \p(B)$
  \item $\p_B(A) = \p(A)$
  \end{enumerate}
\end{prop}

\begin{mydef}
  On dit que deux événements A et B sont indépendants si $\p(A \cap B)
  = \p(A)\p(B)$
\end{mydef}

\begin{Rem}
  Attention, indépendance et incompatibilités sont deux notions
  différentes. En effet, si A et B sont deux événements, de
  probabilités non nulles, incompatibles alors $\p(A \cap B) = 0 \neq
  \p(A) \p(B)$
\end{Rem}

\note Parler de l'indépendance pour plus de 2 événements ??

\begin{prop}
  Soient A et B deux événements indépendants. Alors A et $\overline{B}$
  sont indépendants.
\end{prop}

\begin{Proof}
  $\p(A \cap \overline{B}) = \p(A \backslash (A \cap B)) = \p(A) -
  \p(A \cap B) = \p(A) - \p(A)\p(B) = \p(A)(1-\p(B)) = \p(A) \p(\overline{B})$
\end{Proof}

\subsection{Résultats de décomposition}
\begin{prop}
  Soient A et B deux événements avec $\mathbb{P}(A) > 0$ alors $\p(A
  \cap B) = \p_A(B) \p(A)$
\end{prop}

\note Prochaine proposition à faire si on parle avant de Poincaré,
indépendances de n événements, ... .

\begin{prop}
  Soit $(A_i)_{1\leq i \leq n}$ une suite finie d'événements tels que
  $\p(\bigcap \limits_{\scriptscriptstyle{1 \leq i \leq n-1}} A_i) > 0$. On a :
  \[
    \p(\bigcap \limits_{1 \leq i \leq n} A_i) = \p(A_1) \p_{A_1}(A_2)
    \p_{A_1 \cap A_2}(A_3) \dots \p_{A_1 \cap \dots \cap A_{n-1}}(A_n)
  \]
\end{prop}

\note A démontrer ...

\subsection{Formule des probabilités totales}
\begin{mydef}
  Une famille finie $(A_i)_{1\leq i \leq n}$ d'événements deux à deux
  incompatibles et tels que $\bigcup \limits_{1 \leq i \leq n} A_i =
  \Omega$ est appelé \textbf{système complet d'événements}.
\end{mydef}

\begin{prop}
  Soit $(A_i)_{1\leq i \leq n}$ un \textbf{système complet
    d'événements} ayant tous une probabilité stritement
  positive. Alors :
  \[
    \p(A) = \sum \limits_{i=1}^n \p_{A_i}(A)\p(A_i)
  \]
\end{prop}

\note Pour la preuve, considérer $A \cap \Omega = \bigcup \limits_{1
  \leq i \leq n} (A \cap A_i)$

\subsection{Arbres pondérés}

\section{Exercices}

\chapter{Intégrales, Primitives}

%%% Local Variables:
%%% mode: latex
%%% TeX-master: "Experience_aleatoire"
%%% End:


\end{document}
%%% Local Variables:
%%% mode: latex
%%% TeX-master: t
%%% End:
