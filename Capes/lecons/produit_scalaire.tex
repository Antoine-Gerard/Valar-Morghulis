\chapter{Produit scalaire}

\section{Produit scalaire de deux vecteurs: vocabulaire et définition}
\begin{mydef}
  Soient $\vec{u}$ et $\vec{v}$ deux vecteurs. On appelle produit scalaire de
  $\vec{u}$ et $\vec{v}$, et on note $\vu \cdot \vv$, le nombre:
  \[
    \vu \cdot \vv = \norm{\vu} \norm{\vv} \cos(\vu,\vv)
  \]
\end{mydef}

\note insertion dessin

\begin{Rem}
  \begin{itemize}[label=$\bullet$, leftmargin=2cm]
  \item \textbf{Attention: } Le produit scalaire de deux vecteurs est un nombre
    réel et non un vecteur
  \item Le produit scalaire de deux vecteurs est proportionnel à la norme de ces
    deux vecteurs ainsi qu'à l'angle formé par les deux.
  \end{itemize}
\end{Rem}

\begin{propriete}[Expression analytique du produit scalaire]
  Soient $\vec{u}$ et $\vec{v}$ deux vecteurs ayant pour coordonnées $(x,y)$ et
  $(x',y')$  respectivement dans un repère orthonormé. On a:
  \[
    \vec{u} \vec{v} = xx' + yy'
  \]
\end{propriete}

\begin{mydef}[Carré scalaire]
  Soit $\vu$ un vecteur du plan. On appelle \textbf{carré scalaire} de $\vu$ le
  nombre réel noté $\vu^2$ et défini par:
  \[
    \vu^2 = \vu \cdot \vu = \norm{u}^2
  \]
\end{mydef}

\section{Produit scalaire et orthogonalité}
\begin{mydef}[Orthogonalité de deux vecteurs]
  Soient $\vu$ et $\vv$ deux vecteurs non nuls du plan. Soient A, B et C trois
  points tels que $\vec{AB} = \vec{u}$ et $\vec{AC} = \vv$. Les vecteurs $\vu$
  et $\vv$ sont dit \textbf{orthogonaux} si les droites $(AB)$ et $(BC)$ sont
  perpendiculaires. On note alors: $\vu \perp \vv$
\end{mydef}
\section{Applications}


%%% Local Variables:
%%% mode: latex
%%% TeX-master: "Experience_aleatoire"
%%% End:
