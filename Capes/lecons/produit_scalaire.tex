\chapter{Produit scalaire}

\section{Produit scalaire de deux vecteurs: vocabulaire et définition}
\begin{mydef}
  Soient $\vec{u}$ et $\vec{v}$ deux vecteurs. On appelle produit scalaire de
  $\vec{u}$ et $\vec{v}$, et on note $\vu \cdot \vv$, le nombre:
  \[
    \vu \cdot \vv = \norm{\vu} \norm{\vv} \cos(\vu,\vv)
  \]
\end{mydef}

\begin{Rem}
  \begin{itemize}[label=$\bullet$, leftmargin=2cm]
  \item \textbf{Attention: } Le produit scalaire de deux vecteurs est un nombre
    réel et non un vecteur
  \item Le produit scalaire de deux vecteurs est proportionnel à la norme de ces
    deux vecteurs ainsi qu'à l'angle formé par les deux.
  \end{itemize}
\end{Rem}

\section{Norme et orthogonalité}

\section{Applications}


%%% Local Variables:
%%% mode: latex
%%% TeX-master: "Experience_aleatoire"
%%% End:
