\documentclass[a4paper]{article}
\usepackage[utf8]{inputenc}
\usepackage[T1]{fontenc}
\usepackage[french]{babel}
\usepackage{graphicx}
\usepackage{float}
\usepackage{amsmath}
\usepackage{amsfonts}
\usepackage[overload]{empheq} 
\usepackage{lmodern}
\usepackage{pdfpages}
\usepackage{a4wide}
\usepackage[showlabels,sections,floats,textmath,displaymath]{preview}
\usepackage{hyperref}
\usepackage{algorithm2e}
\usepackage{fullwidth}
\usepackage{stmaryrd}
\usepackage{tikz}
\usepackage{enumitem}
\usepackage{bbold}
\usepackage{pgfplots}
\usetikzlibrary{patterns}
\usepgfplotslibrary{fillbetween}
\usepackage{ntheorem}
\usepackage{eurosym}
\usepackage{array,multirow,makecell}
\def\singleletter#1{\rotatebox[origin=B]{90}{#1}}
\makeatletter
\def\parseletters#1{\@tfor \@tempa := #1 \do {\kern2pt\singleletter{\@tempa}}}
\makeatother
\def\verticaltext#1{\rotatebox[origin=c]{-90}{\catcode`\~=13\def~{\vbox to 0.33em{}}\parseletters{#1}}}


\newcolumntype{R}[1]{>{\raggedleft\arraybackslash }b{#1}}
\newcolumntype{L}[1]{>{\raggedright\arraybackslash }b{#1}}
\newcolumntype{C}[1]{>{\centering\arraybackslash }b{#1}}

\theoremstyle{break}
\newtheorem{mydef}{Définition}[section]
\newtheorem{theo}{Théorème}[section]
\newtheorem*{Rem}{Remarque}
\newtheorem*{Proof}{preuve:}
\newtheorem{lemme}{Lemme}[section]
\newtheorem{prop}{Proposition}[section]

\DeclareMathOperator{\divg}{div} %operateur divergence
\DeclareMathOperator{\supp}{supp}
\DeclareMathOperator{\e}{e} %exponentielle



\newcommand{\norm}[1]{\left \Vert {#1} \right \Vert}
\newcommand{\scal}[2]{\left\langle {#1} , {#2} \right\rangle}

\newcommand{\J}[1]{\mathcal{J}(#1)}
\newcommand{\R}{\mathbb{R}}

\newcommand{\note}{$\bullet$ \textbf{Notes: }}

\begin{document}
\section{Dérivées successives et Développement limités}

\subsection{Dérivées successives et polynôme de Taylor}
\begin{mydef}[Dérivées successives]
  Soit $f: I \mapsto \R$ une fonction. Pour tout $n \in \mathbb{N}*$ on définit
  les \textbf{dérivées successives} de f de proche en proche (par récurrence)
  par: pour $x_0 \in I$, la dérivée n-ième de $f$, notée $f^{(n)}(x_0)$, est
  ,si elle existe, la dérivée de $f^{(n-1)}(x_0)$. On dit alors que f est n fois
  dérivable sur I si $f^{(n)}$ existe sur I. 
\end{mydef}

\begin{mydef}[Formules de Taylor du 1er et deuxième ordre]
  Soit f une fonction dérivable tel que f' soit continue. On a alors la formule
  de Taylor du 1er ordre avec reste intégrale suivante:
  \[
    f(x_0+h) = f(x_0) + \int_{x_0}^{x_0+h} f'(t) dt = \color{red}{f(x_0) + hf'(x_0) +
    h\varepsilon(h) }
  \]

  Si f est deux fois dérivable et que $f''$ est continue on a la formule de
  Taylor d'ordre 2 suivante:
  \[
    f(x_0+h) = f(x_0) + hf'(x_0) + \int_{x_0}^{x_0+h} (x_0+h-t) f''(t) dt = \color{red}{f(x_0)
    + hf'(x_0) + \frac{h^2}{2} f''(x_0)+ h^2 \varepsilon(h) }
  \]
  où $\varepsilon(h)$ tend vers 0 lorsque h tend vers 0
\end{mydef}

\note On voit ici que si f est deux fois dérivable alors on peut ``localement
approcher'' f par un polynôme du second degré avec une erreur qui sera
négligeable devant $h^2$. Par exemple, si $h=\frac{1}{1000}$ la formule de
Taylor d'ordre 1 nous donne une erreur qui sera négligeable devant $h
=\frac{1}{1000}$ tandis que la formule d'ordre 2 nous donne une erreur
négligeable devant $h^2 = \frac{1}{1.000.000}$
\end{document}