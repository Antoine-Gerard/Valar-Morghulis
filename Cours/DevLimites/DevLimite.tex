\documentclass[a4paper]{article}
\usepackage[utf8]{inputenc}
\usepackage[T1]{fontenc}
\usepackage[french]{babel}
\usepackage{graphicx}
\usepackage{float}
\usepackage{amsmath}
\usepackage{amsfonts}
\usepackage[overload]{empheq} 
\usepackage{lmodern}
\usepackage{pdfpages}
\usepackage{a4wide}
\usepackage[showlabels,sections,floats,textmath,displaymath]{preview}
\usepackage{hyperref}
\usepackage{algorithm2e}
\usepackage{fullwidth}
\usepackage{stmaryrd}
\usepackage{tikz}
\usepackage{enumitem}
\usepackage{bbold}
\usepackage{pgfplots}
\usetikzlibrary{patterns}
\usepgfplotslibrary{fillbetween}
\usepackage{ntheorem}
\usepackage{eurosym}
\usepackage{array,multirow,makecell}
\usepackage{cancel}

\def\singleletter#1{\rotatebox[origin=B]{90}{#1}}
\makeatletter
\def\parseletters#1{\@tfor \@tempa := #1 \do {\kern2pt\singleletter{\@tempa}}}
\makeatother
\def\verticaltext#1{\rotatebox[origin=c]{-90}{\catcode`\~=13\def~{\vbox to 0.33em{}}\parseletters{#1}}}


\newcolumntype{R}[1]{>{\raggedleft\arraybackslash }b{#1}}
\newcolumntype{L}[1]{>{\raggedright\arraybackslash }b{#1}}
\newcolumntype{C}[1]{>{\centering\arraybackslash }b{#1}}

\theoremstyle{break}
\newtheorem{mydef}{Définition}[section]
\newtheorem{theo}{Théorème}[section]
\newtheorem*{Rem}{Remarque}
\newtheorem*{Proof}{preuve:}
\newtheorem{lemme}{Lemme}[section]
\newtheorem{prop}{Proposition}[section]

\DeclareMathOperator{\divg}{div} %operateur divergence
\DeclareMathOperator{\supp}{supp}
\DeclareMathOperator{\e}{e} %exponentielle



\newcommand{\norm}[1]{\left \Vert {#1} \right \Vert}
\newcommand{\scal}[2]{\left\langle {#1} , {#2} \right\rangle}

\newcommand{\J}[1]{\mathcal{J}(#1)}
\newcommand{\R}{\mathbb{R}}

\newcommand{\note}{$\bullet$ \textbf{Notes: }}

\begin{document}
\section{Dérivées successives et Développement limités}

\subsection{Dérivées successives et polynôme de Taylor}
\begin{mydef}[Dérivées successives]
  Soit $f: I \mapsto \R$ une fonction. Pour tout $n \in \mathbb{N}*$ on définit
  les \textbf{dérivées successives} de f de proche en proche (par récurrence)
  par: pour $x_0 \in I$, la dérivée n-ième de $f$, notée $f^{(n)}(x_0)$, est
  ,si elle existe, la dérivée de $f^{(n-1)}(x_0)$. On dit alors que f est n fois
  dérivable sur I si $f^{(n)}$ existe sur I. 
\end{mydef}

\begin{mydef}[Formules de Taylor du 1er et deuxième ordre]
  Soit f une fonction dérivable tel que f' soit continue. On a alors la formule
  de Taylor du 1er ordre avec reste intégrale suivante:
  \[
    f(x_0+h) = f(x_0) + \int_{x_0}^{x_0+h} f'(t) dt = \color{red}{f(x_0) + hf'(x_0) +
    h\varepsilon(h) }
  \]

  Si f est deux fois dérivable et que $f''$ est continue on a la formule de
  Taylor d'ordre 2 suivante:
  \[
    f(x_0+h) = f(x_0) + hf'(x_0) + \int_{x_0}^{x_0+h} (x_0+h-t) f''(t) dt = \color{red}{f(x_0)
    + hf'(x_0) + \frac{h^2}{2} f''(x_0)+ h^2 \varepsilon(h) }
  \]
  où $\varepsilon(h)$ tend vers 0 lorsque h tend vers 0
\end{mydef}

\note On voit ici que si f est deux fois dérivable alors on peut ``localement
approcher'' f par un polynôme du second degré avec une erreur qui sera
négligeable devant $h^2$. Par exemple, si $h=\frac{1}{1000}$ la formule de
Taylor d'ordre 1 nous donne une erreur qui sera négligeable devant $h
=\frac{1}{1000}$ tandis que la formule d'ordre 2 nous donne une erreur
négligeable devant $h^2 = \frac{1}{1.000.000}$

\subsection{Lien entre polynôme de Taylor d'ordre 2 et extrema locaux}
\begin{prop}
  Soit $f$ une fonction continue sur un intervalle I. Soit $x_0$ un
  extremum local de f sur I, on suppose que f est dérivable en $x_0$,
  alors on a nécessairement $f'(x_0) = 0$
\end{prop}

\begin{Rem}
Attention, l'annulation de la dérivée est une condition nécessaire
mais elle n'est pas suffisante:
\[
  x_0 \text{ est un \textbf{extremum local}} \Rightarrow f'(x_0) = 0
  \quad \text{ mais } \quad
  f'(x_0) = 0 \not \Rightarrow x_0 \text{ est un \textbf{extremum local}}
\]
\end{Rem}

\textbf{Exemple: } La fonction $f: x \mapsto x^3$ en $x=0$ à une dérivée
nulle mais $x=0$ n'est pas un extremum local de la fonction.  

\begin{prop}
  Soit $f$ une fonction continue sur un intervalle I. Soit $x_0$ un
  extremum local de la fonction f sur I. On suppose que f est deux
  fois dérivable en $x_0$. Alors on a:
  \begin{enumerate}[label=(\alph*), leftmargin=2cm]
  \item Si $f''(x_0) > 0$ alors $x_0$ est un minimum local.
  \item Si $f''(x_0) < 0$ alors $x_0$ est un maximum local.
  \item Si $f''(x_0) = 0$ on ne peut rien conclure sur la nature de
    $x_0$.
  \end{enumerate}
\end{prop}

\textbf{Idée de la preuve: } D'après la proposition précédente on a
$f'(x_0) = 0$ ($x_0$ est un extrémum local). Ainsi, si l'on écrit le
polynôme de Taylor à l'ordre 2 de $f$ au voisinage de $x_0$ on a:
\[
  f(x_0+h) = f(x_0) + \underbrace{h * \cancel{f'(x_0)}}_{=0} +
  \frac{h^2}{2} f''(x_0) + h^2 \epsilon (h)
\]

Ainsi pour \og h assez petit\fg{} on aura:
\begin{itemize}[label=$\bullet$, leftmargin=2cm]
\item $f(x_0+h) > f(x_0)$ si $f''(x_0) >0$ et donc $x_0$ est un minimum
  local.
\item $f(x_0+h) < f(x_0)$ si $f''(x_0) <0$ et donc $x_0$ est un maximum
  local.
\end{itemize}

\subsection{Développement limité des fonctions usuelles}
\begin{mydef}
  On dit qu'une fonction f admet un développement limité d'ordre n au
  voisinage de $x_0$ si f peut s'écrire de la forme suivante:
  \[
    f(x) = a_0 + (x-x_0) a_1 + (x-x_0)^2  a_2 + \dots +
    (x-x_0)^n a_n + \epsilon(x)(x-x_0)^n 
  \]
  avec $\lim \limits_{x \to x_0} \epsilon(x) = 0$. Un développement
  limité d'une fonction en un point est une approximation polynômiale
  de cette fontion au voisinage de ce point. 
\end{mydef}

\begin{prop}
  Une fonction dérivable n fois au point $x_0$ (avec $n \geq 1$) admet
  un développement limité d'ordre n au voisinage $x_0$ qui n'est rien
  d'autre que le polynôme de Taylor d'ordre n:
  \[
    f(x) = f(x_0) + (x-x_0)f'(x_0) + \frac{(x-x_0)^2}{2} f''(x_0) +
    \dots + \frac{(x-x_0)^n}{n!} f^{(n)}(x) + \epsilon(x)(x-x_0)^n
  \]
\end{prop}

\note Dans ce cours nous ne dépasseront jamais les développements
limités d'ordre 2 c'est à dire:
\begin{itemize}[label=$\bullet$, leftmargin=2cm]
\item ordre 1 au voisinage de $x_0$:
  \[
    f(x) = f(x_0) + (x-x_0)f'(x_0) + (x-x_0) \varepsilon(x)
  \]
\item ordre 2 au voisinage de $x_0$:
  \[
    f(x) = f(x_0) + (x-x_0)f'(x_0) + \frac{(x-x_0)^2}{2} f''(x_0)
    + (x-x_0)^2 \varepsilon(x)
  \]
\end{itemize}

\textbf{Développement limités des fonctions usuelles au voisinage de
  0}
\begin{center}
  \renewcommand{\arraystretch}{1.5} %donne la distance entre les lignes%
  \begin{tabular}{|l|C{5cm}|C{5cm}|}
    \hline
    fonction & DL$_1(0)$ & DL$_2(0)$\\
    \hline
    $e^x$ & $1 + x + x \varepsilon(x)$
          & $1 + x + \frac{x^2}{2} + x^2 \varepsilon(x)$ \\
    \hline
    $\sin(x)$ & $x + x \varepsilon(x)$
              & $x + x^2 \varepsilon(x)$ \\
    \hline
    $\cos(x)$ & $1 + x\varepsilon(x)$
              & $1 - \frac{x^2}{2} + x^2 \varepsilon(x)$ \\
    \hline
    $\tan(x)$ & $x + x \varepsilon(x)$
              & $x + x^2 \varepsilon(x)$ \\
    \hline
    $\ln(1+x)$ & $x + x \varepsilon(x)$
               & $x - \frac{x^2}{2} + x^2 \varepsilon(x)$ \\
    \hline
    $\sqrt{1+x}$ & $1 + \frac{x}{2} + x \varepsilon(x)$
                 & $1 + \frac{x}{2} - \frac{x^2}{8}+ x^2
                   \varepsilon(x)$ \\
    \hline
    $\frac{1}{1-x}$ & $1 + x + x \varepsilon(x)$
                    & $1 + x + x^2 + x^2 \varepsilon(x)$  \\
    \hline
    $\frac{1}{1+x}$ & $1 - x + x \varepsilon(x)$
                    & $1 - x + x^2 + x^2 \varepsilon(x)$\\
    \hline
  \end{tabular}
\end{center}

\note Il n'est pas obligé de retenir ces développements limités par
coeur mais il faut savoir les retrouver.

\subsection{Développement limité d'un produit}
\begin{prop}
  Soient $f$ et $g$ deux fonctions définies sur I admettant chacunes un développement
  limité d'ordre n au voisinage de $x_0$. Pour effectuer le développement limité
  d'ordre n de  $f \times g$ au voisinage de $x_0$, il suffit de faire le produit
  du développement limité de $f$ par le développement limité de $g$ et de délaisser
  tous les termes de degré strictement supérieur à n.
\end{prop}

\textbf{Exemple: } Donner le développement limité d'ordre 2 de la fonction
$f: x \mapsto \e^x \ln(1+x)$ au voisinage de 0.

On a :
\[
  \e^x = 1 + x + \frac{x^2}{2} + x^2\varepsilon(x) \qquad { et } \qquad
  \ln(1+x) = x - \frac{x^2}{2} +  x^2\varepsilon(x)
\]

Effectuons le produit des deux D.L. :
\begin{eqnarray*}
  (1 + x + \frac{x^2}{2} + x^2\varepsilon(x))(x - \frac{x^2}{2} +  x^2\varepsilon(x))
  &=& x - \frac{x^2}{2} + x^2 + x^2\varepsilon(x) (\color{red}{\text{ termes de deg $>2$ dans $x^2\varepsilon(x)$})} \\
  &=& x + \frac{x^2}{2} + x^2\varepsilon(x)
\end{eqnarray*}

\subsection{Théorème des accroissements finis:}
\begin{theo}
  Soit f une fonction continue et dérivable sur un intervalle I. Alors
  il existe un réel c dans I tel que:
  \[
    f'(c)(b-a) = f(b) - f(a)
  \]
\end{theo}

\textbf{Idée de preuve: }
On considère la fonction $g: x \mapsto \int_a^b f'(x) \mathbf{dt} =
f'(x) (b-a)$. Cette fonction est continue et on a :
\[
  \min \limits_{x \in [a,b]} g(x) = \int_a^b \min \limits_{x \in
    [a,b]} f'(x) dt
  \leq  \int_a^b f'(t) dt
  \leq \int_a^b \max \limits_{x \in [a,b]} f'(x) = \max \limits_{x \in [a,b]} g(x)
\]

\begin{center}
  \begin{tikzpicture}
    \begin{axis}[
      width = 10cm,     
      xmin = -8.5,   
      xmax = 8.5,   
      ymin = -0.8,      
      ymax = 15,       
      axis x line= center,
      axis y line= none,
      xticklabels={}
      ]
      \addplot[domain=-6:5.5,samples=100,fill=blue!50!white]
      {0.0035*(-4.9+5.9)*(-4.9+3)*(-4.9)*(-4.9-1)*(-4.9-5.9) +
        8}\closedcycle;
      \addplot[domain=-6:5.5,samples=100,fill=green!50!white]
      {0.0035*(4.5+5.9)*(4.5+3)*(4.5)*(4.5-1)*(4.5-5.9) + 8}\closedcycle;
      \addplot[mark=none, domain = -6:5.5, red, smooth, thick]
      {0.0035*(x+5.9)*(x+3)*(x)*(x-1)*(x-5.9) + 8};
      \node at (axis cs: -6,10.5) [anchor=north east]
      {$\scriptscriptstyle{f'(c_2)}$};
      \node at (axis cs: -6,2.5) [anchor=north east]
      {$\scriptscriptstyle{f'(c_1)}$};
      \node at (axis cs: -6,0) [below] {\small a};
      \node at (axis cs:  5.5,0) [below] {\small b};
      \node at (axis cs: -0.25,1.5) [below] {$\mathcal{A}_1 = f'(c_1) * (b-a)$};
      \node at (axis cs:  -0.25,5) [below] {$\mathcal{A}_2 = f'(c_2) *
        (b-a)$};
      \node at (axis cs:  5.5,5) [right] {$\color{red}{f'}$};
    \end{axis}
  \end{tikzpicture}
\end{center}

Graphiquement, si on note $c_1$ le minimum de $f'$ et $c_2$ son
maximum alors $\mathcal{A}_1 = (b-a) f'(c_1)$ correspond à l'aire du
rectangle vert et $\mathcal{A}_2 = (b-a) f'(c_2)$ correspond à l'aire
cumulée des deux rectangles. Ainsi si on note $\mathcal{A}$ l'aire
sous la courbe de $f'$, c'est à dire $\mathcal{A} = \int_a^b f'(t)dt$,
alors on a
\[
  g(c_1) = \mathcal{A}_1 \leq \mathcal{A} \leq \mathcal{A}_2 = g(c_2)
\]
Par le théorème des valeurs intermédiaires, il existe au moins un $c
\in [c_1,c_2]$ tel que $\mathcal{A} = g(c)$ c'est à dire:
\[
  \int_a^b f'(t) dt = f(b) - f(a) = (b-a)f'(c)
\]

Si on remplace le reste intégrale par $\int_{x_0}^{x_0+h} f'(c) dt$
($c \in [x_0, x_0+h]$) dans la formule de Taylor d'ordre 1
on obtient:
\[
  f(x_0 +h) = f(x_0) + hf'(c)
\]

Si on fait de même dans la formule de Taylor d'ordre 2 on obtient:
\[
  f(x_0 +h) = f(x_0) + hf'(x_0) + \frac{h^2}{2}f''(c)
\]

\note Ceci n'est qu'une introduction du reste de Lagrange dans les
développements limités. Il faudra plutôt retenir la forme

Pour l'ordre 1:
\[
    \color{red}{
      f(x) = f(x_0) + (x-x_0)f'(x_0) + (x-x_0)^2 \varepsilon(x) }
\]

Pour l'ordre 2:
\[
    \color{red}{
    f(x) = f(x_0) + (x-x_0)f'(x_0) + \frac{(x-x_0)^2}{2} f''(x_0)
    + (x-x_0)^2 \varepsilon(x) }
\]


\end{document}
%%% Local Variables:
%%% mode: latex
%%% TeX-master: t
%%% End:
