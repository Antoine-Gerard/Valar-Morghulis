\documentclass[a4paper]{article}
\usepackage[utf8]{inputenc}
\usepackage[T1]{fontenc}
\usepackage[french]{babel}
\usepackage{graphicx}
\usepackage{float}
\usepackage{amsmath}
\usepackage{amsfonts}
\usepackage[overload]{empheq} 
\usepackage{lmodern}
\usepackage{pdfpages}
\usepackage{a4wide}
\usepackage[showlabels,sections,floats,textmath,displaymath]{preview}
\usepackage{hyperref}
\usepackage{algorithm2e}
\usepackage{fullwidth}
\usepackage{stmaryrd}
\usepackage{tikz}
\usepackage{enumitem}
\usepackage{bbold}
\usepackage{pgfplots}
\usetikzlibrary{patterns}
\usepgfplotslibrary{fillbetween}
\usepackage{ntheorem}
\usepackage{eurosym}
\usepackage{array,multirow,makecell}
\usepackage{cancel}

\def\singleletter#1{\rotatebox[origin=B]{90}{#1}}
\makeatletter
\def\parseletters#1{\@tfor \@tempa := #1 \do {\kern2pt\singleletter{\@tempa}}}
\makeatother
\def\verticaltext#1{\rotatebox[origin=c]{-90}{\catcode`\~=13\def~{\vbox to 0.33em{}}\parseletters{#1}}}


\newcolumntype{R}[1]{>{\raggedleft\arraybackslash }b{#1}}
\newcolumntype{L}[1]{>{\raggedright\arraybackslash }b{#1}}
\newcolumntype{C}[1]{>{\centering\arraybackslash }b{#1}}

\theoremstyle{break}
\newtheorem{mydef}{Définition}[section]
\newtheorem{theo}{Théorème}[section]
\newtheorem*{Rem}{Remarque}
\newtheorem*{Proof}{preuve:}
\newtheorem{lemme}{Lemme}[section]
\newtheorem{prop}{Proposition}[section]

\DeclareMathOperator{\divg}{div} %operateur divergence
\DeclareMathOperator{\supp}{supp}
\DeclareMathOperator{\e}{e} %exponentielle



\newcommand{\norm}[1]{\left \Vert {#1} \right \Vert}
\newcommand{\scal}[2]{\left\langle {#1} , {#2} \right\rangle}

\newcommand{\J}[1]{\mathcal{J}(#1)}
\newcommand{\R}{\mathbb{R}}

\newcommand{\note}{$\bullet$ \textbf{Notes: }}

\begin{document}
\section{Équations différentielles linéaires du 1er ordre}

\subsection{Définitions}
\begin{enumerate}[label=$\bullet$, leftmargin=2cm]
\item On appelle \emph{équation différentielle linéaire d'ordre n},
  une relation liant une fonction dérivable à ses n dérivées d'ordre
  inférieurs ou égale à n, de la forme:
  \[
    y^{(n)}(t) + a_{n-1}(t) y^{(n-1)}(t) + \dots + a_0(t) y(t) = b(t)
    \qquad (E)
  \]
\item On appelle \emph{solution} de (E) sur I une fonction y dérivable qui
  vérifie (E) sur I.
\item On appelle \emph{solution de l'équation homogène (ou encore
    solution de l'équation sans second membre) (E$_0$)} une solution de
  l'équation différentielle:
  \[
    y^{(n)}(t) + a_{n-1}(t) y^{(n-1)}(t) + \dots + a_0(t) y(t) = 0
    \qquad (E_0)
  \]
\item Résoudre sur I l'équation (E) revient à chercher l'ensemble des
  solutions de (E)
  sur I.
\end{enumerate}

\begin{prop}
  Si $y_1$ et $y_2$ sont deux solutions de (E$_0$) alors, pour tout
  $\alpha_1 \in \R$, pour tout $\alpha_2 \in \R$, $\alpha_1 y_1 +
  \alpha_2 y_2$ est solution de $(E_0)$.
\end{prop}

\begin{prop}
  Toute solution de (E) est la somme d'une solution de l'équation
  homogène (E$_0$) et d'une solution particulière de (E) 
\end{prop}

\subsection{Équations différentielles linéaires du 1er ordre}
Dans cette partie on s'intéresse à l'ensemble des solutions d'une
équation différentielle linéaire du 1er ordre, c'est à dire qui
s'écrit sous la forme:
\[
  y'(t) + a(t) y(t) = b(t) (E)
\]
\begin{theo}
  L'ensemble des solutions de l'équation homogène (E$_0$) associée à (E)
  ($y'(t) + a(t) y(t) = 0 ~ (E_0)$) est l'ensemble des fonctions y de
  la forme:
  \[
    y(t) = C \e^{A(t)}
  \]
  où C est une constante et A une primitive de la fonction a
\end{theo}

On a vu que toute solution de (E) se composaient de la somme d'une
solution particulière et d'une solution à l'équation homogène
$(E_0)$. Maintenant que l'on connait les solutions de l'équation
homogène (Théorème précédent) il nous faut calculer une solution
particulière de (E). Une solution particulière est soit évidente, soit
déterminée par la méthode dite \emph{de variation de la constante}.

\textbf{Variation de la constante: }

On suppose qu'il existe une solution particulière $y_p(t)$ de (E) de
la forme $c(t)y_h(t)$ où $y_h$ est une solution de l'équation homogène
(non nulle). Alors, en dérivant $c(t)y_h(t)$, et en se rappelant que
$y_h'(t) + a(t) y_h(t) = 0$, on obtient:
\[
  y_p'(t) = c'(t) y_h(t) + c(t) y'_h(t) = c'(t) y_h(t) - c(t) a(t) y_h(t) 
\]
Puisque $y_p$ est une solution particulière de (E) on a:
\[
  y_p'(t) + a(t) \underbrace{y_p(t)}_{:= c(t) y_h(t)} = b(t) 
\]

Finalement on obtient:
\[
  \underbrace{c'(t) y_h(t) - \cancel{c(t) a(t) y_h(t)}}_{y_p'(t)}
  + \cancel{a(t) c(t) y_h(t)} = b(t) 
\]

Ce qui nous donne donc une équation homogène pour la fonction inconnue
c(t):
\[
  c'(t) y_h(t) = b(t)
\]
Si $y_h$ est non-nulle sur un intervalle, on cherche une primitive
c(t) de $c'(t) = \frac{b(t)}{y_h(t)}$. Cette primitive est ensuite
insérée dans la formule $y_p(t) = c(t) y_h(t)$ et on a trouvé une
solution particulière de (E).

\textbf{Exemple: trouver les solutions de $y'(t) = ty(t) + t$}

\textbf{Recherche de la solution à l'équation homogène}

L'équation homogène associée à (E) ici est:
\[
  y'(t) = ty(t)
\]

L'ensemble des solutions de l'équation homogènes est alors l'ensemble
des fonctions
\[
  y_h(t) = C \e^{\frac{t^2}{2}}
\]

\vspace{1\baselineskip}

\textbf{Variation de la constante:}

On cherche une solution particulière de la forme $y_p(t) = C(t) y_h(t)
= C(t) \e^{\frac{t}{2}^2}$. Pour trouver C(t) on doit trouver la
solution de:
\[
  C'(t) \e^{\frac{t^2}{2}} = t \Rightarrow C'(t) = t\e^{-\frac{t^2}{2}}
\]

Une primitive de $t\e^{\frac{-t^2}{2}}$ étant $-\e^{\frac{-t^2}{2}}$
on trouve une solution particulière qui est:
\[
  y_p(t) = C(t) \e^{\frac{t^2}{2}} = \e^{\frac{-t^2}{2}} \times
  (-\e^{\frac{t^2}{2}}) = -1
\]
On aurait pu trouver cette solution particulière directement.

L'ensemble des solutions de (E) est donc l'ensemble des fonctions:
\[
  y(t) = C\e^{\frac{t^2}{2}} - 1
\]

\subsection{Problème de Cauchy:}
\begin{theo}{Le problème de Cauchy}
  \[
    \left\{
      \begin{array}{ll}
        y'(x)  &= a(x) y(x) \\
        y(x_0) &= y_0
      \end{array}
    \right.
  \]
  admet une solution unique qui est $y(x) = y_0 \e^{A(x)}$ où A est
  la primitive de a qui s'annule en $x_0$ c'est à dire:
  \[
    y(x) = y_0 \e^{\int_{x_0}^{x} a(t) dt}
  \]
\end{theo}

\textbf{Exemple:}
\[
    \left\{
      \begin{array}{ll}
        y'(x)  &= xy(x) + x \\
        y(0) &= 3
      \end{array}
    \right.
  \]
  On a déja vu que $y(x) = C\e^{\frac{x^2}{2}} - 1$. On veut donc
  $y(0) = 3$ c'est à dire $C - 1 = 3 \Rightarrow C = 4$. Ainsi, $y(x)
  = 4 \e^{\frac{x^2}{2}} - 1$
\end{document}

%%% Local Variables:
%%% mode: latex
%%% TeX-master: t
%%% End:
