\documentclass[a4paper]{article}
\usepackage[utf8]{inputenc}
\usepackage[T1]{fontenc}
\usepackage[french]{babel}
\usepackage{graphicx}
\usepackage{float}
\usepackage{amsmath}
\usepackage{amsfonts}
\usepackage[overload]{empheq} 
\usepackage{lmodern}
\usepackage{pdfpages}
\usepackage{a4wide}
\usepackage[showlabels,sections,floats,textmath,displaymath]{preview}
\usepackage{hyperref}
\usepackage{algorithm2e}
\usepackage{fullwidth}
\usepackage{stmaryrd}
\usepackage{tikz}
\usepackage{enumitem}
\usepackage{bbold}
\usepackage{pgfplots}
\usepackage{ntheorem}
\usepackage{eurosym}
\usepackage{array,multirow,makecell}
\def\singleletter#1{\rotatebox[origin=B]{90}{#1}}
\makeatletter
\def\parseletters#1{\@tfor \@tempa := #1 \do {\kern2pt\singleletter{\@tempa}}}
\makeatother
\def\verticaltext#1{\rotatebox[origin=c]{-90}{\catcode`\~=13\def~{\vbox to 0.33em{}}\parseletters{#1}}}


\newcolumntype{R}[1]{>{\raggedleft\arraybackslash }b{#1}}
\newcolumntype{L}[1]{>{\raggedright\arraybackslash }b{#1}}
\newcolumntype{C}[1]{>{\centering\arraybackslash }b{#1}}

\theoremstyle{break}
\newtheorem{mydef}{Définition}[section]
\newtheorem{theo}{Théorème}[section]
\newtheorem*{Rem}{Remarque}
\newtheorem*{Proof}{preuve:}
\newtheorem{lemme}{Lemme}[section]
\newtheorem{prop}{Proposition}[section]

\DeclareMathOperator{\divg}{div} %operateur divergence
\DeclareMathOperator{\supp}{supp}
\DeclareMathOperator{\e}{e} %exponentielle



\newcommand{\norm}[1]{\left \Vert {#1} \right \Vert}
\newcommand{\scal}[2]{\left\langle {#1} , {#2} \right\rangle}

\newcommand{\J}[1]{\mathcal{J}(#1)}
\newcommand{\R}{\mathbb{R}}

\newcommand{\note}{$\bullet$ \textbf{Notes: }}

\begin{document}
\section{Continuité, Dérivation et fonctions usuelles}

\subsection{Continuité}
Soit $f:\mathbb{R} \rightarrow \mathbb{R}$. On dit que f est continue
en un point $a \in \mathbb{R}$ ssi:
\[
  \forall \varepsilon > 0, ~ \exists \eta > 0, ~ (| x - a| < \eta
  \Rightarrow |f(x) - f(a)| < \epsilon )
\]

Ceci est équivalent à dire que:
\[
  \lim \limits_{x \to a} f(x) = f(a)
\]
\note Faire graphique pour insister sur le fait que f est continue si
quand ``x se rapproche de a'' alors ``f(x) se rapproche de
f(a)''. Rajouter un exemple de fonction discontinue (fonction en
escalier par exemple)

\vspace{1\baselineskip}

\begin{center}
  \begin{minipage}{0.45\linewidth}
    \begin{tikzpicture}
      \draw [ultra thin, gray] (-2,-2) grid (3,3);
      \draw [->, thick] (-2,0) -- (3,0);
      \draw [->, thick] (0,-2) -- (0,3);
      \draw (3,0) node[below] {x};
      \draw (0,3) node[below right] {y};
      \draw (2,0) node[below] {$x_0$};
      \draw [dashed, black, ultra thick] (2,0) -- (2,2);
      \draw (1.5,-1) node[below] {$f(x) = E(x)$};
      \foreach \k in {-1,...,2}
      {
        \draw [domain=\k:\k+1, red, very thick] plot(\x, {floor(\x)});
        \draw [red] (\k,\k) node {$\bullet$};
        \draw [red] (\k+1.1,\k) node {$\subset$};
      }
    \end{tikzpicture}
  \end{minipage} 
  \begin{minipage}{0.45\linewidth}
    \begin{tikzpicture}
      \draw [ultra thin, gray] (-2,-2) grid (3,3);
      \draw [->, thick] (-2,0) -- (3,0);
      \draw [->, thick] (0,-2) -- (0,3);
      \draw (3,0) node[below] {x};
      \draw (0,3) node[below right] {y};
      \draw (1.41,0) node[below] {$x_0$};
      \draw [dashed, black, ultra thick] (1.41,0) -- (1.41,2);
      \draw (1.5,-1) node[below] {$f(x) = x^2$};
      \draw [domain=-1.75:1.75, red, very thick] plot(\x,{\x*\x});
    \end{tikzpicture}
  \end{minipage}
\end{center}

\vspace{1\baselineskip}

\textbf{Opérations algébriques et continuité:}

\begin{itemize}[label =$\bullet$, leftmargin =2cm]
\item Si $f,g$ continue en $x_0 \in \mathbb{R}$ alors $f+g$ et $f
  \cdot g$ continue en $x_0$ 
\item Si $g(x_0) \neq 0$ et $f$ continue en $x_0$ alors $\frac{f}{g}$
  est continue en $x_0$
\item Si $f$ est continue en $x_0$ et g est continue en $y_0 = f(x_0)$
  alors $g \circ f$ est continue en $x_0$
\end{itemize}

\begin{theo}[\textbf{Théorème des valeurs intermédiaires}]
  
  Soit f une fonction \textbf{continue} définie sur un intervalle $I$. Soient $a$
  et $b$ deux réels de $I$ tels que:
  \[
    a < b \text{ et } f(a)<f(b) \quad (resp. f(b) > f(a))
  \]

  Alors pour tout réels k compris entre $f(a)$ et $f(b)$ (resp. entre
  $f(b)$ et $f(a)$) il existe au moins un réel $x_0$ compris entre $a$ et $b$ tel que:
  \[
    f(x_0) = k
  \]
 
\end{theo}

\begin{center}  
  \begin{tikzpicture}
    \draw [ultra thin, gray, scale=0.5] (-6,-6) grid (6,6);
    \draw [->, thick, scale=0.5] (-6,0) -- (6,0);
    \draw [->, thick,scale=0.5] (0,-6) -- (0,6);
    \draw (3,0) node[below] {x};
    \draw (0,3) node[below right] {y};
    \draw [domain=-6:6, red, very thick, scale=0.5,smooth]
    plot(\x,{0.0035*(\x+5.9)*(\x+3)*(\x)*(\x-1)*(\x-5.9)});
    \foreach \y in {-5,4}
    {
      \draw[scale=0.5]
      (\y,{0.0035*(\y+5.9)*(\y+3)*(\y)*(\y-1)*(\y-5.9)}) node
      {$\bullet$} ;
      \draw[scale=0.5, dashed, thick]
      (\y,{0.0035*(\y+5.9)*(\y+3)*(\y)*(\y-1)*(\y-5.9)}) --
      (0,{0.0035*(\y+5.9)*(\y+3)*(\y)*(\y-1)*(\y-5.9)});
      \draw[scale=0.5, dashed,thick]
      (\y,{0.0035*(\y+5.9)*(\y+3)*(\y)*(\y-1)*(\y-5.9)}) --
      (\y,0);
    }
    \draw[scale=0.5] (0,2.01) node[right] {f(a)};
    \draw[scale=0.5] (0,2.02) node {$\bullet$};
    \draw[scale=0.5] (0,-5.4) node[left] {f(b)};
    \draw[scale=0.5] (0,-5.5) node {$\bullet$};
    \draw[scale=0.5] (-5,0) node[below] {a};
    \draw[scale=0.5] (-5,0) node {$\bullet$};
    \draw[scale=0.5] (4,0) node[above] {b};
    \draw[scale=0.5] (4,0) node {$\bullet$};
    \draw[scale=0.5,thick,blue] (-6,-3) -- (6,-3);
    \draw[scale=0.5,blue] (0,-3) node[above left] {k};
    \draw[scale=0.5,blue] (0,-3) node {$\bullet$};
    \draw[scale=0.5,blue] (2.85,-3) node {$\bullet$};
    \draw[scale=0.5,blue] (5.55,-3) node {$\bullet$};
    \draw[scale=0.5,blue] (2.85,0) node {$\bullet$};
    \draw[scale=0.5,blue] (5.55,0) node {$\bullet$};
    \draw[scale=0.5,blue] (2.85,0) node[above] {$x_0$};
    \draw[scale=0.5,blue] (5.55,0) node[above] {$x_1$};
    \draw[scale=0.5,blue,dashed, thick] (2.85,0) -- (2.85,-3);
    \draw[scale=0.5,blue,dashed, thick] (5.55,0) -- (5.55,-3) ;
  \end{tikzpicture}
\end{center}

\begin{prop}
  Soit I un intervalle de $\mathbb{R}$ et $f:\mathbb{R} \to
  \mathbb{R}$ une fonction continue. Alors l'image de I par f est un
  intervalle de $\mathbb{R}$
\end{prop}
\begin{Proof}
  On assume ici que I est un intervalle de $\mathbb{R}$ ssi pour tout
  x,y dans I tels que $x \leq y$ alors $[x,y] \subset I$
  Si $f(I)$ est réduit à un point le résultat est évident.
  \newline
  Sinon, soient $y_1$ et $y_2$ dans $f(I)$ tels que $y_1 < y_2$. Il
  existe $x_1$ et $x_2$ dans I tels que $f(x_1) = y1$ et $f(x_2) =
  y2$. D'après le théorème des valeurs intermédiaires (f continue sur
  I) , pour tout $y$ tels que $y_1 = f(x_1) \leq y \leq f(x_2) = y_2$ il existe $x$ entre
  $x_1$ et $x_2$ (ou entre $x_2$ et $x_1$) tel que $f(x) = y$. Ainsi, $[y_1,y_2]
  = [f(x_1), f(x_2)] \subset f(I)$.
  \newline
  Finalement, on vient de montrer que pour tout $y_1,~y_2$ dans $f(I)$ tels que $y_1<y_2$ on a $[y_1,
  y_2] \subset f(I)$. Donc $f(I)$ est un intervalle de $\mathbb{R}$
  $\square$
\end{Proof}

\subsection{Dérivation}

\textbf{Rappel:} L'équation d'une droite dans le plan muni d'un repère
$(O; \vec{i}; \vec{j})$ est donnée par:
\[
  y = ax +b 
\]

\textbf{Exercices:} Donner l'équation de la droite passant par les
points (1,1) et (3,2).

Solution:

\vspace{1\baselineskip}

\begin{minipage}{0.3\linewidth}
  \begin{align*}
    &1 = a * 1 + b\\
    &2 = a * 3 + b 
  \end{align*}
\end{minipage}
$\Leftrightarrow$
\begin{minipage}{0.3\linewidth}
  \begin{align*}
    &2*a = 1\\
    &2 = a * 3 + b 
  \end{align*}
\end{minipage}
$\Leftrightarrow$
\begin{minipage}{0.3\linewidth}
  \begin{align*}
    &a = \frac{1}{2}\\
    &b = 2 - 3*a = 2 - \frac{3}{2}=\frac{1}{2} 
  \end{align*}
\end{minipage}

Finalement:
\[
  y = \frac{1}{2} * (x+1)
\]

Donner l'équation de la droite passant par les
points $(x_0,f(x_0))$ et $(x_1,f(x_1))$.

Solution:

\vspace{1\baselineskip}

\begin{tabular}{ccccc}
  $\begin{aligned}
    f(x_0) = a * x_0 + b\\
    f(x_1) = a * x_1 + b 
  \end{aligned}$
& $\Leftrightarrow$
& $\begin{aligned}
  &a *(x_0-x_1) = f(x_0)-f(x_1)\\
  &f(x_1) = a * x_1 + b 
\end{aligned}$
&$\Leftrightarrow$
& $\begin{aligned}
  &a = \frac{f(x_0)-f(x_1)}{x_0-x_1}\\
  &b = f(x_1) - x_1*a = \frac{f(x_1)x_0 - f(x_0)x_1}{x_0-x_1} 
\end{aligned}$
\end{tabular}

\vspace{1\baselineskip}

Finalement:
\[
  y = \frac{f(x_0)-f(x_1)}{x_0-x_1} * x + \frac{f(x_1)x_0 - f(x_0)x_1}{x_0-x_1} 
\]

\begin{mydef}
  On dit qu'un fonction $f:\mathbb{R} \to \mathbb{R}$ est dérivable en
  $x_0$ ssi
  \[
    \lim \limits_{x \to x_0} \frac{f(x)-f(x_0)}{x-x_0} = \lim \limits_{h \to 0} \frac{f(x_0+h)-f(x_0)}{h}
  \]
  existe. Cette limite est alors appelée \textbf{dérivée de f en $x_0$}
  et est notée: $\mathbf{f'(x_0)}$. On a donc:
  \[
    f'(x_0)=\lim \limits_{h \to 0} \frac{f(x_0+h)-f(x_0)}{h}
  \]
  Cette définition est équivalent eà l'éxistence d'une fonction
  $\varepsilon(h)$ avec $\lim \limits_{h \to 0}
  \frac{\varepsilon(h)}{h} = 0$ et une constante A telles que:
  \[
    f(x_0 + h) = f(x_0) + A.h + \varepsilon(h) 
  \]
  $(f(x_0+h) = f(x_0) + hf'(x_0) + f(x_0+h) - f(x_0) - h*f'(x_0))$
\end{mydef}

\note Insister sur la linéarisation de f autour de $x_0$. Faire exercice 1 1-4 

\textbf{Opérations algébriques sur les dérivées:}

\begin{itemize}[label =$\bullet$, leftmargin =2cm]
\item Si $f,g$ dérivable en $x_0 \in \mathbb{R}$ alors $f+g$ continue
  en $x_0$ et $(f+g)'(x_0) = f'(x_0)+g'(x_0)$
\item Si $f,g$ dérivable en $x_0 \in \mathbb{R}$ alors $f \cdot g$ dérivable
  en $x_0$ et $(f \cdot g)'(x_0) = f'(x_0)g(x_0) + f(x_0)g'(x_0)$
\item Si f est dérivable en $x_0$ et si $f'(x_0) \neq 0$ alors
  $\frac{1}{f}$ est dérivable en $x_0$ et:
  \[
    \big( \frac{1}{f} \big)'(x_0) = -\frac{f'(x_0)}{f^2(x_0)}
  \]
\item si f est dérivable en $x_0$, alors pour tout réel $\lambda$,
  $\lambda f$ est dérivable en $x_0$. On a alors $(\lambda f)'(x_0) =
  \lambda f'(x_0)$
\item si f est dérivable en $x_0$ et g est dérivable en $f(x_0)$ alors
  $g \circ f$ est dérivable en $x_0$ et:
  \[
    (g \circ f)'(x_0) = g'(f(x_0)) f'(x_0)
  \]
\end{itemize}

\subsection{Sens de variation d'une fonction}
Soit f une fonction continue sur un intervalle $[a,b]$ et dérivable
sur $]a,b[$

\begin{enumerate}[label=(\alph*), leftmargin =2cm]
\item f est croissante (resp. décroissante) sur $[a,b]$ ssi pour tout
  x de l'intervalle $]a,b[$ $f'(x) \geq 0$ (resp. $f'(x) \leq 0$)
\item Si $f'(x) > 0$ (resp. $f'(x)$ < 0) pour tout x de l'intervalle
  $[a,b]$  alors f est strictement croissante (resp. décroissante)
\item f est constante sur $[a,b]$ ssi pour tout x de l'intervalle
  $]a,b[$ $f'(x) = 0$
\end{enumerate}

\begin{Proof}
  \textbf{(a)} $\Rightarrow$. Supposons que f est croissante sur
  $[a,b]$ (resp. décroissante). Alors on a :
  \[
    \forall x,y \in [a,b], \quad (x<y \Rightarrow f(x) \leq f(y) )
  \]
  Montrons que $f'(x)>=0$ pour tout x dans l'intervalle $]a,b[$. Soit
  $x \in ]a,b[$. Par hypothèse f est dérivable sur $]a,b[$ donc:
  \[
    f'(x) = \lim \limits_{h \to 0} \frac{f(x+h) - f(x)}{h}
  \]
  Supposons sans perte de généralité que $h>=0$. On a donc $x+h >=
  x$. Par hypothèse $f(x+h) \geq f(x)$ (resp. $f(x+h) \leq f(x)$) et
  donc $\frac{f(x+h) -f(x)}{h} \geq 0$ (resp. $\frac{f(x+h) -f(x)}{h}
  \leq 0$ ). Par passage à la limite quand $h\to 0$ on arrive à $f'(x)
  \geq 0$ (resp. $f'(x) \leq 0$ )

  \vspace{1\baselineskip}
  
  $\Leftarrow$. Ici on suppose que $f'(x) \geq 0$ (resp. $f'(x) \leq
  0$) et on veut montrer que f est croissante (resp. décroissante)
  (Conséquence du Théoreme des accroissements finis qui sera vu plus
  tard) $\square$
\end{Proof}

\begin{theo}
  Soit $f: I \to \R$ une fonction continue et strictement croissante
  (ou strictement décroissante).
  \newline
  Il existe une fonction, dite fonction réciproque et noté
  $f^{-1}: f(I) \to I$, qui est telle que $f^{-1}(f(x)) = x $ pour
  tout x dans $I$ et $f(f^{-1}(y)) = y$ pour tout  $y \in f(I)$
\end{theo}

\textbf{Dérivée de la fonction réciproque: }

Soit $f: I \to \R$ une fonction continue, dérivable et telle que
$f'(x) \neq 0$ sur I, alors $f^{-1}(y)$ est dérivable sur $f(I)$ et:
\[
  (f^{-1})'(y) = \frac{1}{f'(f^{-1}(y))}
\]

\begin{Proof}
  Soit $y_1, ~y_2 \in f(I)$. Alors il existe $x_1, ~x_2 \in I$ tels
  que $f(x_1) = y_1$ et $f(x_2) = y_2$. Ainsi:
  \[
    \frac{f^{-1}(y_1) - f^{-1}(y_2)}{y_1-y_2} =
    \frac{f^{-1)}(f(x_1)) - f^{-1}(f(x_2))}{f(x_1)-f(x_2)}
    = \frac{x_1 - x_2}{f(x_1)-f(x_2)}
    = \frac{1}{\frac{f(x_1)-f(x_2)}{x_1 - x_2}}
  \]
  \newline
  Ainsi en passant à la limite $x_1 \to x_2$ on a:
  \[
    \lim \limits_{x_1 \to x_2}\frac{1}{\frac{f(x_1)-f(x_2)}{x_1 -
        x_2}} = \frac{1}{f'(x_2)}
    \quad \text{ ou encore }
    \lim \limits_{y_1 \to y_2} \frac{f^{-1}(y_1)-f^{-1}(y_2)}{y_1 -
        y_2} = \frac{1}{f'(f^{-1}(y_2))}
  \]
  (Développer au tableau)
\end{Proof}

\subsection{Fonctions usuelles: }
\subsubsection{$\exp(x)$ (ou $\e^{x}$)}
\begin{mydef}
  On appelle demi-vie le temps mis par une substance (molécule,
  médicament, ...) pout perdre la moitié de son activité
  pharmacologique ou physiologique. Par exemple, dans le domaine de la
  radioactivité la demi-vie est le temps au bout duquel la moitié des
  noyaux radioactifs d'une source se sont désintégré. Au bout de deux
  demi-vies de temps $\frac{1}{2} + \frac{1}{2} \times \frac{1}{2}$ de
  noyaux se sont désintégrés. 
\end{mydef}

\textbf{Exemple: le taux d'intérêt}
Imaginons que l'on place à la banque une somme s qui nous rapportera
$x = \frac{p}{100}$ au bout de N années. On est tenté de supposer que
chaque année nous rapport $s * \frac{x}{N}$. Imaginons donc maintenant
que l'on place la même somme s de départ mais que cette fois la banque nous
rembourse $\frac{x}{N}$ fois la somme capitalisée par an. Dans ce cas,
si on note $s_i$ la somme capitalisée à la i-ième on a:
\begin{align*}
  s_1 &= s * (1 + \frac{x}{N})\\
  s_2 &= s_1 * (1 + \frac{x}{N}) = s * (1 + \frac{x}{N}) (1 +
        \frac{x}{N})\\
      & \vdots\\
  s_N &= s_{N-1} * (1 + \frac{x}{N}) =  s * (1 + \frac{x}{N})^N
\end{align*}

Par exemple s=100000\euro, N=10 et p = 5\%.  1\textsuperscript{ère}
proposition: $s_{10} =  100000 * (1+\frac{5}{100}) = 105000$. 2\textsuperscript{ème}
proposition: $s_{10} =  100000 * (1+\frac{5}{10*100})^{10} \approx
105114.01$

Cette suite possède une limite, et on appelle
\[
  \e^{x} = \lim \limits_{N \to \infty} (1 + \frac{x}{N})^N
\]

\note $\e = \lim \limits_{N \to \infty} (1 + \frac{1}{N})^N \approx
2.718281828459$ (Se servir de cette égalité pour établir $\e^{x}$ en
posant $N = k*x$)

Pour des décroissances (comme la demi-vie, etc.) un \og x
négatif \fg décrit de la même façon que les taux d'intérêts positifs
les taux d'intérêts négatifs. C'est pour cette raison -- variation
d'un taux constant d'une quantité au cours du temps -- que la fonction
exponentielle apparaît dans beaucoup de formules scientifiques !!

On peut montrer que la fonction exponentielle est continue, même
(infiniment) dérivable. De fait, limite et dérivée interchangent dans
ce cas. Par conséquent:
\[
  \exp(x)'
  = \lim \limits_{n \to \infty} n(1+\frac{x}{n})^{n-1} \times
  \frac{1}{n}
  = \lim \limits_{n \to \infty} (1+\frac{x}{n})^{n} 
  (1+\frac{x}{n})^{-1}
  = \e^{x} \lim \limits_{n \to \infty} (1+\frac{x}{n})^{-1}
  = \e^{x}
\]
Un taux 0 correspond à aucune croissance ou décroissance et donc
$e^0=1$
\newpage
\textbf{Propriétés: }
\begin{enumerate}[label=(\alph*), leftmargin =2cm]
\item exp est strictement croissante sur $\R$
\item $\forall (x,y) \in \R \times \R$ on a $\exp(x+y) = \exp(x) \exp(y)$
\item $\lim \limits_{x \to -\infty} \e^{x} = 0$ et $\lim \limits_{x \to +\infty} \e^{x} = +\infty$
\item exp est dérivable sur $\R$ et $(\exp)'(x) = \exp(x)$
\end{enumerate}

\subsubsection{logarithme néperien $\ln(x)$}
La fonction réciproque de l'exponentielle est notée ln et est appelé
logarithme (néperien). Elle possède les propriétés suivantes:
\begin{enumerate}[label=(\alph*), leftmargin =2cm]
\item ln est strictement croissante sur $\R^+_*$
\item $\forall (x,y) \in \R^+_* \times \R^+_*$ on a $\ln(xy) = \ln(x) +
  \ln(y)$
\item $\lim \limits_{x \to 0^+} \ln(x) = -\infty$ et $\lim \limits_{x
    \to +\infty} \ln(x) = +\infty$
\item ln est dérivable sur $\R^+_*$ et $(\ln)'(x) = \frac{1}{x}$
\end{enumerate}

\note En biologie, chimie, ... , on utilise parfois le $\log_{10}(x)$
pour des raisons historiques. Le logarithme décimal est défini comme
la réciproque de la fonction $f(x)=10^x$.

\begin{lemme}
  Soient $x \in \R$ et $y \in ]0 ; +\infty[$.
  On a:
  \[
    y = 10^x \Leftrightarrow \log_{10}(y) = x \Leftrightarrow x\ln(10)
    = \ln(y) \Leftrightarrow x = \frac{\ln(y)}{\ln(10)}
  \]
\end{lemme}

\begin{Proof}
  Soient $x \in \R$ et $y \in ]0 ; +\infty[$.
  \newline
  $\bullet  (y = 10^x \Leftrightarrow \log_{10}(y) = x$): définition d'une
  fonction réciproque
  \newline
  $\bullet  (y = 10^x \Rightarrow x\ln(10) = \ln(y)$) :
  \[
    y = 10^x \Rightarrow \ln(y) = \ln(10^x) \Rightarrow \ln(y) = x \ln(10)   
  \]
  \newline
  $\bullet  (y = 10^x \Rightarrow x = \frac{\ln(y)}{\ln(10)}$) : OK
  \newline
  $\bullet  (x = \frac{\ln(y)}{\ln(10)} \Rightarrow y = 10^x$):
  \[
    x = \frac{\ln(y)}{\ln(10)} \Rightarrow \ln(y) = x \ln(10)
    \Rightarrow \exp(\ln(y)) =  \exp(x \ln(10) )
    \Rightarrow y = 10^x
    \quad \square
  \]
\end{Proof}

\subsubsection{la fonction $x^{\alpha}$}
\begin{mydef}
Soit $\alpha \in \R$. On définit la fonction puissance $f_{\alpha}(x)$
par:
\[
  \forall x \in \R_*^+  ~ f_{\alpha}(x) = \e^{\alpha  \ln(x)}
\]
On note: $f_{\alpha}(x)=x^{\alpha}$
\end{mydef}

On peut remarquer que $\e^{2 \ln(x)} = \e^{\ln(x)+\ln(x)} =
\e^{\ln(x)}\e^{\ln(x)} = x \cdot x = x^2$

\subsubsection{Les fonctions trigonométriques: $\sin(x), ~ \cos(x),~
  \tan(x)$}
En mathématiques, le cercle trigonométrique est un cercle qui permet
de définir des notions comme celle d'angles, de radian, et les
fonctions trigonométriques: cosinus, sinus et tangente. Il s'agit du
cercle dont le rayon est égale à 1 centré sur l'origine du repère,
dans le plan usuel muni d'un repère orthonormé.

\begin{center}  
  \begin{tikzpicture}
    \draw [ultra thin, gray, dashed] (-6,-6) grid (6,6);
    \draw [thick] (-6,0) -- (6,0);
    \draw [thick] (0,-6) -- (0,6);
    \draw [->, thick] (0,0) -- (4,0);
    \draw [->, thick] (0,0) -- (0,4);
    \draw [scale = 4, thick] (0,0) circle (1);
    \draw [scale = 4, thick] (0,0)--(1.1,1.1);
    \draw (1,0) arc (0:45:1);
    \draw (22:1.3) node {$\alpha$};
    \draw [scale=4, thick, red, dashed] (45:1) -- (0:0.7071);
    \draw [scale=4, thick, red, dashed] (90:0.7071) -- (45:1);
    \draw [scale=4, thick, red, dashed] (1,0) -- (1,1);
    \draw [scale=4, red, thick] (0.3, 0.77) node {$\cos(\alpha)$} ;
    \draw [scale=4, red, thick] (0.77, 0.33) node {\rotatebox{90}{$\sin(\alpha)$}} ;
    \draw [scale=4, red, thick] (1.1, 0.5) node {\rotatebox{90}{$\tan(\alpha)$}} ;
    \draw [scale=4] (1,1) node {{$\bullet$}} ;
    \draw [scale=4] (45:1) node {{$\bullet$}} ;
    \draw [scale=4] (1,0) node[below right] {(1,0)} ;
    \draw [scale=4] (0,1) node[above left] {(0,1)} ;
  \end{tikzpicture}
\end{center}

$\bullet ~ \cos(x)$:

Dans un triangle rectangle, le cosinus est défini comme le rapport de la
longueur de son côté adjacent sur la longueur de son hypoténuse. 

Si on jette un oeil au cercle trigonométrique, on peut définir le cosinus d'un
angle $\alpha$ de la sorte: on considère A le point d'intersection de la droite
passant par O et d'angle $\alpha$. Le cosinus de l'angle $\alpha$ est alors
la projection de cette droite sur l'axe horizontale.

Voici les principales propriétés de la fonction cosinus:
\begin{enumerate}[label=(\alph*), leftmargin =2cm]
\item $-1 <= \cos(x) <= 1 ~ \forall x \in \R$
\item $\cos(-x) = cos(x) ~ \forall x \in \R$
\item $\cos(x+2k\pi) = cos(x) ~ \forall x \in \R ~ et ~k \in \mathbb{N}$
  (2-$\pi$ périodique)
\item La fonction cosinus est dérivable sur $\R$ et $\cos'(x) = -\sin(x)$
\end{enumerate}

\vspace{1\baselineskip}

$\bullet ~ \sin(x)$:

Dans un triangle rectangle, le sinus est défini comme le rapport de la
longueur de son côté opposé sur la longueur de son hypoténuse. 

Si on jette un oeil au cercle trigonométrique, on peut définir le sinus d'un
angle $\alpha$ de la sorte: on considère A le point d'intersection de la droite
passant par O et d'angle $\alpha$. Le sinus de l'angle $\alpha$ est alors
la projection de cette droite sur l'axe verticale.

Voici les principales propriétés de la fonction sinus:
\begin{enumerate}[label=(\alph*), leftmargin =2cm]
\item $-1 <= \sin(x) <= 1 ~ \forall x \in \R$
\item $\sin(-x) = -\sin(x) ~ \forall x \in \R$
\item $\sin(x+2k\pi) = \sin(x) ~ \forall x \in \R ~ et ~k \in \mathbb{N}$
  (2-$\pi$ périodique)
\item La fonction cosinus est dérivable sur $\R$ et $\sin'(x) = \cos(x)$
\end{enumerate}

\note: Montrer que pour tout $x$ dans $\R$ on a la relation suivante:
\[
  \cos^2(x) + \sin^2(x) = 1 \text{ (Indication: Utiliser le théorème de Pythagore)}
\]

\subsubsection{La fonction tangente: tan(x)}
La tangente d'un angle $\alpha$ dans un triangle rectangle est définie comme
le rapport entre la longueur du côté opposé à cet angle et la
longueur du côté adjacent. Si on note $h$ la longueur de l'hypoténuse,
$a$ la longueur du côté adjacent à $\alpha$ et $o$ la longueur du côté
opposé on a :
\[
  \tan(\alpha) = \frac{o}{a} = \frac{\frac{o}{h}}{\frac{a}{h}} =
  \frac{\sin(\alpha)}{\cos(\alpha)}
\]

\textbf{Problème: } Que se passe t'il quand $\alpha = \frac{\pi}{2} +
k\pi$ pour $k \in \mathbb{Z}$. La tangente n'est pas défini en ces
points.

Ainsi on défini la fonction trigonométrique tangente de la manière
suivante:

\begin{center}
  $\begin{array}{cccccc}
     f & : & \R \backslash \{ \frac{pi}{2} + k \pi \} & \to & \R, &
       k \in \mathbb{Z}\\
       & & x & \mapsto & \tan(x)=\frac{\sin(x)}{\cos(x)} \\
   \end{array}$
 \end{center}

\textbf{Propriétés de la fonction tangente: }
\begin{itemize}[label=$\bullet$, leftmargin=2cm]
\item $\tan$ est continue sur chaque intervalle de la forme
  $[(2k+1)\frac{\pi}{2}, (2k+1)\frac{\pi}{2} + \pi],~ k \in \mathbb{Z}$
\item $\tan$ est dérivable sur chaque intervalle de la forme
  $[(2k+1)\frac{\pi}{2}, (2k+1)\frac{\pi}{2} + \pi]$ et $\tan'(x)
  =\frac{1}{\cos^2(x)} = 1+\tan^2(x)$ (A montrer)
\item $\tan$ est croissante sur chaque intervalle de la forme
  $[(2k+1)\frac{\pi}{2}, (2k+1)\frac{\pi}{2} + \pi]$ 
\item $\tan(x+2k\pi) = \tan(x),~ k \in \mathbb{Z}$ (2-$\pi$
  périodique)
\item $\lim \limits_{x \to -\frac{\pi}{2}^+} tan(x) = -\infty$ et $\lim
  \limits_{x \to \frac{\pi}{2}^-} tan(x) = \infty$ 
\end{itemize}

\vspace{1\baselineskip}

\hspace{-2.5cm}
\begin{minipage}{0.4\linewidth}
  \begin{tikzpicture}
    \begin{axis}[
      width = 10cm,
      xmin = -2*pi-1,
      xmax = 2*pi+1 ,
      ymin = -1.5,
      ymax = 1.5,
      axis x line= center,
      axis y line= center,        
      xtick=
      {-6.28318,-3.14159, 0, 3.14159, 6.28318},
      ticklabel style={font = \tiny, below},
      xticklabels =
      {$-2\pi$,-$\pi$,$0$,$\pi$,$2\pi$},
      extra x ticks = {-4.7123889, -1.5708, 1.5708, 4.7123889},
      extra x tick labels =
      {$-\frac{3\pi}{2}$,$-\frac{\pi}{2}$,$\frac{\pi}{2}$,$\frac{3\pi}{2}$},
      extra x tick style={tick label style={above, xshift=0.2cm}}
      ]
      \addplot[mark=none, domain = -2*pi:2*pi, red, smooth, thick]
      {cos(deg(x))};
      \legend{$\cos(x)$}
    \end{axis}
  \end{tikzpicture}
\end{minipage}
\hspace{4cm}
\begin{minipage}{0.4\linewidth}
  \begin{tikzpicture}
    \begin{axis}[
      width = 10cm,
      xmin = -2*pi-1,
      xmax = 2*pi+1 ,
      ymin = -1.5,
      ymax = 1.5,
      axis x line= center,
      axis y line= center,        
      xtick=
      {-6.28318,-3.14159, 0, 3.14159, 6.28318},
      ticklabel style={font = \tiny, below},
      xticklabels =
      {$-2\pi$,-$\pi$,$0$,$\pi$,$2\pi$},
      extra x ticks = {-4.7123889, -1.5708, 1.5708, 4.7123889},
      extra x tick labels =
      {$-\frac{3\pi}{2}$,$-\frac{\pi}{2}$,$\frac{\pi}{2}$,$\frac{3\pi}{2}$},
      extra x tick style={tick label style={above, xshift=0.2cm}}
      ]
      \addplot[mark=none, domain = -2*pi:2*pi, red, smooth, thick]
      {sin(deg(x))};
      \legend{$\sin(x)$}
    \end{axis}
  \end{tikzpicture}
\end{minipage} 

\begin{center}
  \begin{tikzpicture}
    \begin{axis}[
      xmin = -2*pi-1,
      xmax = 2*pi+1 ,
      ymin = -5,
      ymax = 5,
      axis x line= center,
      axis y line= center,        
      xtick=
      {-6.28318,-3.14159, 0, 3.14159, 6.28318},
      ticklabel style={font = \tiny, below},
      xticklabels =
      {$-2\pi$,-$\pi$,$0$,$\pi$,$2\pi$},
      extra x ticks = {-4.7123889, -1.5708, 1.5708, 4.7123889},
      extra x tick labels =
      {$-\frac{3\pi}{2}$,$-\frac{\pi}{2}$,$\frac{\pi}{2}$,$\frac{3\pi}{2}$},
      extra x tick style={tick label style={above, xshift=-0.21cm}}
      ]
      \addplot [mark=none, domain = -pi/2+0.1:pi/2-0.1, red, smooth, thick]
      {tan(deg(x))};
      \addplot [mark=none, domain = -3*pi/2+0.1:-pi/2-0.1, red, smooth, thick]
      {tan(deg(x))};
      \addplot [mark=none, domain = pi/2:3*pi/2-0.1, red, smooth, thick]
      {tan(deg(x))};
      \addplot [mark=none, domain = pi/2:3*pi/2-0.1, red, smooth, thick]
      {tan(deg(x))};
      \foreach \k in {-3,-1,...,3}
      {
        \addplot[dashed, thick, smooth, black] coordinates
        {(\k*pi/2,-5)(\k*pi/2,5)};
      }
      \legend{$\tan(x)$}
    \end{axis}
  \end{tikzpicture}
\end{center}

\begin{center}
  \renewcommand{\arraystretch}{3} %donne la distance entre les lignes%
  \begin{tabular}{|l|C{1cm}|C{1cm}|C{1cm}|C{1cm}|C{1cm}|}
    \hline
    x & 0 & $\frac{\pi}{6}$ & $\frac{\pi}{4}$ & $\frac{\pi}{3}$ & $\frac{\pi}{2}$\\
    \hline
    $\cos(x)$ & 1 & $\frac{\sqrt{3}}{2}$ & $\frac{\sqrt{2}}{2}$ & $\frac{1}{2}$ & 0\\
    \hline
    $\sin(x)$ & 0 & $\frac{1}{2}$ & $\frac{\sqrt{2}}{2}$ & $\frac{\sqrt{3}}{2}$ & 1\\
    \hline
    $\tan(x)$ & 0 & $\frac{\sqrt{3}}{3}$ & 1 & $\sqrt{3} $& Non définie\\
    \hline
  \end{tabular}
\end{center}

\note: Formules trigonométriques:

\vspace{1\baselineskip}

\hspace{-1cm}
\begin{minipage}{0.45\linewidth}
  \begin{itemize}[label=$\bullet$]
  \item $\cos(a+b) = \cos(a) \cos(b) - \sin(a)\sin(b)$
  \item $\cos(a-b) = \cos(a) \cos(b) + \sin(a)\sin(b)$
  \item $\sin(a+b) = \sin(a) \cos(b) + \sin(b)\cos(a)$
  \item $\sin(a-b) = \sin(a) \cos(b) - \sin(b)\cos(a)$  
  \end{itemize}
\end{minipage}
\begin{minipage}{0.7\linewidth}
  \begin{itemize}[label=$\bullet$]
  \item $\cos(2a) = \cos^2(a) - \sin^2(a) = 1 - 2\sin^2(a) = 2\cos^2(a)-1$
  \item $\sin(2a) = 2\sin(a) \cos(a)$
  \end{itemize}
\end{minipage}

\subsection{Les fonctions trigonométriques réciproques}
\subsubsection{L'arccosinus: $\arccos(x)$}

L'arc cosinus d'un nombre réel compris entre -1 et 1 est la ``mesure principale
de l'angle'' dont le cosinus vaut ce nombre, ``compris entre  $0$ et
$\pi$. (Montrer qu'on a bijection sur cet intervalle ? Parler de $cos^{-1}$ sur
les calculatrices)

Cette fonction est alors la fonction réciproque de la fonction cosinus définie
sur l'intervalle $[0,\pi]$, donc dans le plan muni du repère orthonormé usuel
$(O,\vec{i}, \vec
{j})$ la courbe représentative de $x \mapsto \arccos(x)$ est le
symétrique de la courbe représentative de $x \mapsto \cos(x)$ par rapport à
la droite d'équation $y=x$

\begin{center}
  \begin{tikzpicture}
    \begin{axis}[
      width = 12cm,
      xmin = -1,
      xmax = pi,
      ymin = -1,
      ymax = pi,
      axis x line= center,
      axis y line= center,
      legend style = {at={(axis cs:2,1)}, anchor=north west}
      ]
      \addplot[mark=none, domain = 0:pi, red, smooth, thick]
      {cos(deg(x))};
      \addplot[mark=none, domain=-1:1, blue, smooth, thick]
      {acos(x)/180*pi};
      \addplot[mark=none, domain=0:3, black, smooth, thick]
      {x};
      \legend{$\cos(x)$, $\arccos(x)$, $y=x$}
    \end{axis}
  \end{tikzpicture}
\end{center}

\textbf{Propriétés de la fonction arccosinus: }
\begin{itemize}[label=$\bullet$, leftmargin=2cm]
\item $\arccos$ est continue sur  $[-1,1]$
\item $\arccos$ est dérivable sur $]-1,1[$ et
  \[
    \arccos'(x)=\frac{-1}{\sqrt{1-x^2}} \text{ (À montrer avec dérivée
      d'une fonction réciproque)}
  \]
\item $\arccos$ est décroissante sur $[-1,1]$
\end{itemize}

\subsubsection{L'arcsinus: $\arcsin(x)$}

L'arc sinus d'un nombre réel compris entre -1 et 1 est la ``mesure principale
de l'angle'' dont le sinus vaut ce nombre, ``compris entre  $-\frac{\pi}{2}$ et
$\frac{\pi}{2}$. (Montrer qu'on a bijection sur cet intervalle ? Parler de $sin^{-1}$ sur
les calculatrices)

Cette fonction est alors la fonction réciproque de la fonction sinus définie
sur l'intervalle $[-\frac{\pi}{2},\frac{\pi}{2}]$, donc dans le plan muni du repère orthonormé usuel
$(O,\vec{i}, \vec{j})$ la courbe représentative de $x \mapsto \arcsin(x)$ est le
symétrique de la courbe représentative de $x \mapsto \sin(x)$ par rapport à
la droite d'équation $y=x$

\begin{center}
  \begin{tikzpicture}
    \begin{axis}[
      width = 12cm,
      xmin = -pi/2 - 0.2,
      xmax =  pi/2 + 0.2 ,
      ymin = -pi/2 - 0.2,
      ymax = pi/2  + 0.2 ,
      axis x line= center,
      axis y line= center,
      legend pos = south east
      ]
      \addplot[mark=none, domain = -pi/2:pi/2, red, smooth, thick]
      {sin(deg(x))};
      \addplot[mark=none, domain=-1:1, blue, smooth, thick]
      {asin(x)/180*pi};
      \addplot[mark=none, domain=0:3, black, smooth, thick]
      {x};
      \legend{$\sin(x)$, $\arcsin(x)$, $y=x$}
    \end{axis}
  \end{tikzpicture}
\end{center}

\textbf{Propriétés de la fonction arcsinus: }
\begin{itemize}[label=$\bullet$, leftmargin=2cm]
\item $\arcsin$ est continue sur  $[-1,1]$
\item $\arcsin$ est dérivable sur $]-1,1[$ et
  \[
    \arcsin'(x)=\frac{1}{\sqrt{1-x^2}} \text{ (À montrer avec dérivée
      d'une fonction réciproque)}
  \]
\item $\arcsin$ est croissante sur $[-1,1]$
\end{itemize}

\subsubsection{L'arctangente: $\arctan(x)$}

L'arc tangente d'un nombre réel compris entre $-\infty$ et $\infty$ est la ``mesure principale
de l'angle'' dont la tangente vaut ce nombre, ``compris entre  $-\frac{\pi}{2}$ et
$\frac{\pi}{2}$. (Montrer qu'on a bijection sur cet intervalle ? Parler de $tan^{-1}$ sur
les calculatrices)

Cette fonction est alors la fonction réciproque de la fonction tangente définie
sur l'intervalle $[-\frac{\pi}{2},\frac{\pi}{2}]$, donc dans le plan muni du repère orthonormé usuel
$(O,\vec{i}, \vec{j})$ la courbe représentative de $x \mapsto \arctan(x)$ est le
symétrique de la courbe représentative de $x \mapsto \tan(x)$ par rapport à
la droite d'équation $y=x$

\begin{center}
  \begin{tikzpicture}
    \begin{axis}[
      width = 12cm,
      xmin = -10,
      xmax =  10,
      ymin = -10,
      ymax =  10,
      axis x line= center,
      axis y line= center,
      legend pos = south east
      ]
      \addplot[mark=none, domain = -pi/2+0.1:pi/2-0.1, red, smooth, thick]
      {tan(deg(x))};
      \addplot[mark=none, domain=-10:10, blue, smooth, thick]
      {atan(x)/180*pi};
      \addplot[mark=none, domain=-10:10, black, smooth, thick]
      {x};
      \legend{$\tan(x)$, $\arctan(x)$, $y=x$}
      \foreach \k in {-1,1}
      {
        \addplot[dashed, thick, smooth, black] coordinates
        {(\k*pi/2,-10)(\k*pi/2,10)};
        \addplot[dashed, thick, smooth, black] coordinates
        {(-10,\k*pi/2)(10,\k*pi/2)};
      }
      \node at (axis cs:pi/2,8) [anchor=south west]
      {$x=\frac{\pi}{2}$};
      \node at (axis cs:-pi/2,8) [anchor=south east]
      {$x=-\frac{\pi}{2}$};
      \node at (axis cs:-pi/2,8) [anchor=south east]
      {$x=-\frac{\pi}{2}$};
      \node at (axis cs:8, pi/2)  [above]
      {$y=\frac{\pi}{2}$};
      \node at (axis cs:8, -pi/2) [below]
      {$y=-\frac{\pi}{2}$};
    \end{axis}
  \end{tikzpicture}
\end{center}

\textbf{Propriétés de la fonction arctan: }
\begin{itemize}[label=$\bullet$, leftmargin=2cm]
\item $\arctan$ est continue sur  $\R$
\item $\arctan$ est dérivable sur $\R$ et
  \[
    \arctan'(x)=\frac{1}{1+x^2} \text{ (À montrer avec dérivée
      d'une fonction réciproque)}
  \]
\item $\arctan$ est croissante sur $\R$
\end{itemize}

\subsection{Croissances comparées}
On rappelle les formules de croissance comparées sans preuve:

\begin{enumerate}[label=(\alph*)]
\item $\forall \alpha>0, \forall \beta>0 \quad \lim \limits_{x \to +\infty}
  \frac{(\ln(x))^{\alpha}}{x^{\beta}} = +\infty$ \qquad et \qquad$\lim \limits_{x \to 0}
  x^{\beta} |\ln(x)|^{\alpha} = 0$
\item $\forall \alpha>0, \forall \beta>0 \quad \lim \limits_{x \to +\infty}
  \frac{(\e^{\alpha x)}}{x^{\beta}} = +\infty$ \qquad et \qquad $\lim \limits_{x \to -\infty}
   |x|^{\beta} \e^{\alpha x} = 0$  
\end{enumerate}

\newpage
\appendix
\section{Formulaire dérivées}
\begin{center}
  \renewcommand{\arraystretch}{3} %donne la distance entre les lignes%
  \begin{tabular}{|l|C{1cm}|C{1cm}|C{1cm}|C{1cm}|C{1cm}|}
    \hline
    $f(x)$ & $x^{\alpha}$ & $\frac{1}{x}$ & $\sqrt{x}$ & $\ln(x)$ & $\e^x$\\
    \hline
    $f'(x)$ & $\alpha x^{\alpha-1}$ & $-\frac{1}{x^2}$ & $\frac{1}{2\sqrt{x}}$ 
            & $\frac{1}{x}$ & $\e^x$\\
    \hline
  \end{tabular}
\end{center}

\tiny
\begin{center}
  \renewcommand{\arraystretch}{2} %donne la distance entre les lignes%
  \begin{tabular}{|l|C{1cm}|C{1cm}|C{2.5cm}|C{1cm}|C{1cm}|C{1cm}|}
    \hline
    $f(x)$ & $\cos(x)$ & $\sin(x)$ & $\tan(x)$ & $\arcsin(x)$ & $\arccos(x)$ & $\arctan(x)$\\
    \hline
    $f'(x)$ & $-\sin(x)$ & $\cos(x)$ & $\frac{1}{\cos^2(x)}=1+\tan^2(x)$ 
            & $\frac{1}{\sqrt{1-x^2}}$ & $\frac{-1}{\sqrt{1-x^2}}$ & $\frac{1}{1+x^2}$\\
    \hline
  \end{tabular}
\end{center}


%%% Local Variables:
%%% mode: latex
%%% TeX-master: "Seance1_3"
%%% End:

\end{document}
%%% Local Variables:
%%% mode: latex
%%% TeX-master: t
%%% End:
