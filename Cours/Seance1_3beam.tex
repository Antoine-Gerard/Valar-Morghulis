\documentclass[serif,mathserif, 10pt]{beamer}

\hypersetup{pdfnewwindow}

\addtobeamertemplate{footline}{\hfill\insertframenumber/\inserttotalframenumber}

\AtBeginSection[]
{
  \begin{frame}[plain,noframenumbering]
    \begin{center}
      \tableofcontents[currentsection]
    \end{center}
  \end{frame}
}

\author{}
\title[Continuit\'e, d\'erivation et fonctions
usuelles\hspace{2em}\insertframenumber/\inserttotalframenumber]{Continuit\'e,d\'erivation et fonctions usuelles}
% \institute{UPJV \\ Amiens}
\date{\today}

\usepackage[utf8]{inputenc}
\usepackage[T1]{fontenc}
\usepackage[french]{babel}

\usepackage{graphicx}
\usepackage{float}
\usepackage{amsmath}
\usepackage{amsfonts}
\usepackage[overload]{empheq}
\usepackage{lmodern}
\usepackage{amsthm}
\usepackage{amsfonts}
\usepackage{subfigure}
\usepackage{multimedia}
\usepackage{pdfpages}
\usepackage{bbm}
\usepackage[font = small, labelfont=bf, justification = centering]{caption}
\usepackage{enumitem}
\usepackage{ulem}
\setlength{\subfigcapskip}{-.5em}
\usepackage{beamerthemesplit}
\usepackage{algorithm2e}
\usepackage{stmaryrd}
\usepackage{tikz}
\usepackage{pgfplots}
\usepackage{array,multirow,makecell}


\newcommand{\norm}[1]{\left \Vert {#1} \right \Vert}
\newcommand{\scal}[2]{\left\langle {#1} , {#2} \right\rangle}

\newcolumntype{R}[1]{>{\raggedleft\arraybackslash }b{#1}}
\newcolumntype{L}[1]{>{\raggedright\arraybackslash }b{#1}}
\newcolumntype{C}[1]{>{\centering\arraybackslash }b{#1}}

\theoremstyle{break}
\newtheorem{mydef}{Définition}[section]
\newtheorem{theo}{Théorème}[section]
\newtheorem*{Rem}{Remarque}
\newtheorem{lemme}{Lemme}[section]
\newtheorem{prop}{Proposition}[section]

\DeclareMathOperator{\divg}{div} %operateur divergence
\DeclareMathOperator{\supp}{supp}
\DeclareMathOperator{\e}{e} %exponentielle


\begin{document}

\begin{frame}[plain]
  \maketitle
\end{frame}

\begin{frame}
  \begin{minipage}{0.45\linewidth}
    \begin{figure}
      \begin{tikzpicture}
        \draw [ultra thin, gray] (-2,-2) grid (3,3);
        \draw [->, thick] (-2,0) -- (3,0);
        \draw [->, thick] (0,-2) -- (0,3);
        \draw (3,0) node[below] {x};
        \draw (0,3) node[below right] {y};
        \draw (2,0) node[below] {$x_0$};
        \draw [dashed, black, ultra thick] (2,0) -- (2,2);
        \draw (1.5,-1) node[below] {$f(x) = E(x)$};
        \foreach \k in {-1,...,2}
        {
          \draw [domain=\k+0.001:\k+1, red, very thick] plot(\x, {floor(\x)});
          \draw [red] (\k,\k) node {$\bullet$};
          \draw [red] (\k+1.1,\k) node {$\subset$};
        }
      \end{tikzpicture}
      \caption{Discontinuité en $x_0$}
    \end{figure}
  \end{minipage}
  \hfill
  \begin{minipage}{0.45\linewidth}
    \begin{figure}
      \begin{tikzpicture}
        \draw [ultra thin, gray] (-2,-2) grid (3,3);
        \draw [->, thick] (-2,0) -- (3,0);
        \draw [->, thick] (0,-2) -- (0,3);
        \draw (3,0) node[below] {x};
        \draw (0,3) node[below right] {y};
        \draw (1.41,0) node[below] {$x_0$};
        \draw [dashed, black, ultra thick] (1.41,0) -- (1.41,2);
        \draw (1.5,-1) node[below] {$f(x) = x^2$};
        \draw [domain=-1.75:1.75, red, very thick] plot(\x,{\x*\x});
      \end{tikzpicture}
      \caption{Continue en $x_0$}
    \end{figure}
  \end{minipage}
\end{frame}

\begin{frame}{Théorème des valeurs intermédiaires (T.V.I)}
  \begin{center}
    \begin{figure}
      \begin{tikzpicture}
        \draw [ultra thin, gray, scale=0.5] (-6,-6) grid (6,6);
        \draw [->, thick, scale=0.5] (-6,0) -- (6,0);
        \draw [->, thick,scale=0.5] (0,-6) -- (0,6);
        \draw (3,0) node[below] {x};
        \draw (0,3) node[below right] {y};
        \draw [domain=-6:6, red, very thick, scale=0.5,smooth]
        plot(\x,{0.0035*(\x+5.9)*(\x+3)*(\x)*(\x-1)*(\x-5.9)});
        \foreach \y in {-5,4}
        {
          \draw[scale=0.5]
          (\y,{0.0035*(\y+5.9)*(\y+3)*(\y)*(\y-1)*(\y-5.9)}) node
          {$\bullet$} ;
          \draw[scale=0.5, dashed, thick]
          (\y,{0.0035*(\y+5.9)*(\y+3)*(\y)*(\y-1)*(\y-5.9)}) --
          (0,{0.0035*(\y+5.9)*(\y+3)*(\y)*(\y-1)*(\y-5.9)});
          \draw[scale=0.5, dashed,thick]
          (\y,{0.0035*(\y+5.9)*(\y+3)*(\y)*(\y-1)*(\y-5.9)}) --
          (\y,0);
        }
        \draw[scale=0.5] (0,2.01) node[right] {f(a)};
        \draw[scale=0.5] (0,2.02) node {$\bullet$};
        \draw[scale=0.5] (0,-5.4) node[left] {f(b)};
        \draw[scale=0.5] (0,-5.5) node {$\bullet$};
        \draw[scale=0.5] (-5,0) node[below] {a};
        \draw[scale=0.5] (-5,0) node {$\bullet$};
        \draw[scale=0.5] (4,0) node[above] {b};
        \draw[scale=0.5] (4,0) node {$\bullet$};
        \draw[scale=0.5,thick,blue] (-6,-3) -- (6,-3);
        \draw[scale=0.5,blue] (0,-3) node[above left] {k};
        \draw[scale=0.5,blue] (0,-3) node {$\bullet$};
        \draw[scale=0.5,blue] (2.85,-3) node {$\bullet$};
        \draw[scale=0.5,blue] (5.55,-3) node {$\bullet$};
        \draw[scale=0.5,blue] (2.85,0) node {$\bullet$};
        \draw[scale=0.5,blue] (5.55,0) node {$\bullet$};
        \draw[scale=0.5,blue] (2.85,0) node[above] {$x_0$};
        \draw[scale=0.5,blue] (5.55,0) node[above] {$x_1$};
        \draw[scale=0.5,blue,dashed, thick] (2.85,0) -- (2.85,-3);
        \draw[scale=0.5,blue,dashed, thick] (5.55,0) -- (5.55,-3) ;
      \end{tikzpicture}
    \end{figure}
  \end{center}
\end{frame}

\begin{frame}
  \centering
  \begin{tikzpicture}
    \begin{axis}[
      width = 10cm,
      xmin = -10,
      xmax = 5 ,
      ymin = -10,
      axis x line= center,
      axis y line= center,
      legend pos= south east,
      extra x ticks = {0},
      extra y ticks = {1},
      extra x tick style = {tick label style={below right}},
      extra y tick style = {tick label style={above right}}
      ]
      \addplot[mark=none, domain = -10:4, red, smooth, thick]
      {exp(x)};
      \legend{$\e^x$};
      \node at (axis cs:0,1) {$\bullet$};
    \end{axis}
  \end{tikzpicture}
\end{frame}

\begin{frame}
  \centering
  \begin{tikzpicture}
    \begin{axis}[
      width = 10cm,
      xmin = -1,
      xmax = 10 ,
      ymin = -10,
      axis x line= center,
      axis y line= center,
      legend pos= south east,
      extra x ticks = {1}
      ]
      \addplot[mark=none, domain = 0.001:1, red, smooth, thick]
      {ln(x)};
      \addplot[mark=none, domain = 1:10, red, smooth, thick]
      {ln(x)};
      \legend{$\ln x$};
      \node at (axis cs:1,0) {$\bullet$};
    \end{axis}
  \end{tikzpicture}
\end{frame}

\begin{frame}{Le cercle trigonométrique}
  \begin{center}  
  \begin{tikzpicture}[scale=0.75]
    \draw [ultra thin, gray, dashed] (-6,-6) grid (6,6);
    \draw [thick] (-6,0) -- (6,0);
    \draw [thick] (0,-6) -- (0,6);
    \draw [->, thick] (0,0) -- (4,0);
    \draw [->, thick] (0,0) -- (0,4);
    \draw [scale = 4, thick] (0,0) circle (1);
    \draw [scale = 4, thick] (0,0)--(1.1,1.1);
    \draw (1,0) arc (0:45:1);
    \draw (22:1.3) node {$\alpha$};
    \draw [scale=4, thick, red, dashed] (45:1) -- (0:0.7071);
    \draw [scale=4, thick, red, dashed] (90:0.7071) -- (45:1);
    \draw [scale=4, thick, red, dashed] (1,0) -- (1,1);
    \draw [scale=4, red, thick] (0.3, 0.77) node {$\cos(\alpha)$} ;
    \draw [scale=4, red, thick] (0.77, 0.33) node {\rotatebox{90}{$\sin(\alpha)$}} ;
    \draw [scale=4, red, thick] (1.1, 0.5) node {\rotatebox{90}{$\tan(\alpha)$}} ;
    \draw [scale=4] (1,1) node {{$\bullet$}} ;
    \draw [scale=4] (45:1) node {{$\bullet$}} ;
    \draw [scale=4] (1,0) node[below right] {(1,0)} ;
    \draw [scale=4] (0,1) node[above left] {(0,1)} ;
  \end{tikzpicture}
\end{center}
\end{frame}

\begin{frame}{$y=\cos(x)$}
  \centering
  \begin{tikzpicture}
    \begin{axis}[
      width = 10cm,
      xmin = -2*pi-1,
      xmax = 2*pi+1 ,
      ymin = -1.5,
      ymax = 1.5,
      axis x line= center,
      axis y line= center,        
      xtick=
      {-6.28318,-3.14159, 0, 3.14159, 6.28318},
      ticklabel style={font = \tiny, below},
      xticklabels =
      {$-2\pi$,-$\pi$,$0$,$\pi$,$2\pi$},
      extra x ticks = {-4.7123889, -1.5708, 1.5708, 4.7123889},
      extra x tick labels =
      {$-\frac{3\pi}{2}$,$-\frac{\pi}{2}$,$\frac{\pi}{2}$,$\frac{3\pi}{2}$},
      extra x tick style={tick label style={above, xshift=0.2cm}}
      ]
      \addplot[mark=none, domain = -2*pi:2*pi, red, smooth, thick]
      {cos(deg(x))};
      \legend{$\cos(x)$}
    \end{axis}
  \end{tikzpicture}
\end{frame}

\begin{frame}{$y=\sin(x)$}
  \centering
  \begin{tikzpicture}
    \begin{axis}[
      width = 10cm,
      xmin = -2*pi-1,
      xmax = 2*pi+1 ,
      ymin = -1.5,
      ymax = 1.5,
      axis x line= center,
      axis y line= center,        
      xtick=
      {-6.28318,-3.14159, 0, 3.14159, 6.28318},
      ticklabel style={font = \tiny, below},
      xticklabels =
      {$-2\pi$,-$\pi$,$0$,$\pi$,$2\pi$},
      extra x ticks = {-4.7123889, -1.5708, 1.5708, 4.7123889},
      extra x tick labels =
      {$-\frac{3\pi}{2}$,$-\frac{\pi}{2}$,$\frac{\pi}{2}$,$\frac{3\pi}{2}$},
      extra x tick style={tick label style={above, xshift=0.2cm}}
      ]
      \addplot[mark=none, domain = -2*pi:2*pi, red, smooth, thick]
      {sin(deg(x))};
      \legend{$\sin(x)$}
    \end{axis}
  \end{tikzpicture}
\end{frame}

\begin{frame}{$y=\tan(x)$}
  \centering
  \begin{tikzpicture}
    \begin{axis}[
      xmin = -2*pi-1,
      xmax = 2*pi+1 ,
      ymin = -5,
      ymax = 5,
      axis x line= center,
      axis y line= center,        
      xtick=
      {-6.28318,-3.14159, 0, 3.14159, 6.28318},
      ticklabel style={font = \tiny, below},
      xticklabels =
      {$-2\pi$,-$\pi$,$0$,$\pi$,$2\pi$},
      extra x ticks = {-4.7123889, -1.5708, 1.5708, 4.7123889},
      extra x tick labels =
      {$-\frac{3\pi}{2}$,$-\frac{\pi}{2}$,$\frac{\pi}{2}$,$\frac{3\pi}{2}$},
      extra x tick style={tick label style={above, xshift=-0.21cm}},
      legend pos = south east
      ]
      \addplot [mark=none, domain = -pi/2+0.1:pi/2-0.1, red, smooth, thick]
      {tan(deg(x))};
      \addplot [mark=none, domain = -3*pi/2+0.1:-pi/2-0.1, red, smooth, thick]
      {tan(deg(x))};
      \addplot [mark=none, domain = pi/2:3*pi/2-0.1, red, smooth, thick]
      {tan(deg(x))};
      \addplot [mark=none, domain = pi/2:3*pi/2-0.1, red, smooth, thick]
      {tan(deg(x))};
      \foreach \k in {-3,-1,...,3}
      {
        \addplot[dashed, thick, smooth, black] coordinates
        {(\k*pi/2,-5)(\k*pi/2,5)};
      }
      \legend{$\tan(x)$}
    \end{axis}
  \end{tikzpicture}
\end{frame}

\begin{frame}{Valeurs particulières}
  \begin{center}
    \renewcommand{\arraystretch}{3} %donne la distance entre les lignes%
    \begin{tabular}{|l|C{1cm}|C{1cm}|C{1cm}|C{1cm}|C{1cm}|}
      \hline
      x & 0 & $\frac{\pi}{6}$ & $\frac{\pi}{4}$ & $\frac{\pi}{3}$ & $\frac{\pi}{2}$\\
      \hline
      $\cos(x)$ & 1 & $\frac{\sqrt{3}}{2}$ & $\frac{\sqrt{2}}{2}$ & $\frac{1}{2}$ & 0\\
      \hline
      $\sin(x)$ & 0 & $\frac{1}{2}$ & $\frac{\sqrt{2}}{2}$ & $\frac{\sqrt{3}}{2}$ & 1\\
      \hline
      $\tan(x)$ & 0 & $\frac{\sqrt{3}}{3}$ & 1 & $\sqrt{3} $& Non définie\\
      \hline
    \end{tabular}
  \end{center}
\end{frame}

\begin{frame}{Formules trigonométriques}
  \begin{itemize}[label=$\bullet$]
  \item $\cos(a+b) = \cos(a) \cos(b) - \sin(a)\sin(b)$
  \item $\cos(a-b) = \cos(a) \cos(b) + \sin(a)\sin(b)$
  \item $\sin(a+b) = \sin(a) \cos(b) + \sin(b)\cos(a)$
  \item $\sin(a-b) = \sin(a) \cos(b) - \sin(b)\cos(a)$  
  \item $\cos(2a) = \cos^2(a) - \sin^2(a) = 1 - 2\sin^2(a) = 2\cos^2(a)-1$
  \item $\sin(2a) = 2\sin(a) \cos(a)$
  \end{itemize}
\end{frame}

\begin{frame}{$y=\arccos(x)$}
  \centering
  \begin{tikzpicture}
    \begin{axis}[
      width = 10cm,
      xmin = -1,
      xmax = pi,
      ymin = -1,
      ymax = pi,
      axis x line= center,
      axis y line= center,
      legend style = {at={(axis cs:2,1)}, anchor=north west}
      ]
      \addplot[mark=none, domain = 0:pi, red, smooth, thick]
      {cos(deg(x))};
      \addplot[mark=none, domain=-1:1, blue, smooth, thick]
      {acos(x)/180*pi};
      \addplot[mark=none, domain=0:3, black, smooth, thick]
      {x};
      \legend{$\cos(x)$, $\arccos(x)$, $y=x$}
    \end{axis}
  \end{tikzpicture}
\end{frame}

\begin{frame}{$y=\arcsin(x)$}
  \centering
  \begin{tikzpicture}
    \begin{axis}[
      width = 10cm,
      xmin = -pi/2 - 0.2,
      xmax =  pi/2 + 0.2 ,
      ymin = -pi/2 - 0.2,
      ymax = pi/2  + 0.2 ,
      axis x line= center,
      axis y line= center,
      legend pos = south east
      ]
      \addplot[mark=none, domain = -pi/2:pi/2, red, smooth, thick]
      {sin(deg(x))};
      \addplot[mark=none, domain=-1:1, blue, smooth, thick]
      {asin(x)/180*pi};
      \addplot[mark=none, domain=0:3, black, smooth, thick]
      {x};
      \legend{$\sin(x)$, $\arcsin(x)$, $y=x$}
    \end{axis}
  \end{tikzpicture}
\end{frame}

\begin{frame}{$y=\arctan(x)$}
  \centering
  \begin{tikzpicture}
    \begin{axis}[
      width = 10cm,
      xmin = -10,
      xmax =  10,
      ymin = -10,
      ymax =  10,
      axis x line= center,
      axis y line= center,
      legend pos = south east
      ]
      \addplot[mark=none, domain = -pi/2+0.1:pi/2-0.1, red, smooth, thick]
      {tan(deg(x))};
      \addplot[mark=none, domain=-10:10, blue, smooth, thick]
      {atan(x)/180*pi};
      \addplot[mark=none, domain=-10:10, black, smooth, thick]
      {x};
      \legend{$\tan(x)$, $\arctan(x)$, $y=x$}
      \foreach \k in {-1,1}
      {
        \addplot[dashed, thick, smooth, black] coordinates
        {(\k*pi/2,-10)(\k*pi/2,10)};
        \addplot[dashed, thick, smooth, black] coordinates
        {(-10,\k*pi/2)(10,\k*pi/2)};
      }
      \node at (axis cs:pi/2,8) [anchor=south west]
      {$x=\frac{\pi}{2}$};
      \node at (axis cs:-pi/2,8) [anchor=south east]
      {$x=-\frac{\pi}{2}$};
      \node at (axis cs:-pi/2,8) [anchor=south east]
      {$x=-\frac{\pi}{2}$};
      \node at (axis cs:8, pi/2)  [above]
      {$y=\frac{\pi}{2}$};
      \node at (axis cs:8, -pi/2) [below]
      {$y=-\frac{\pi}{2}$};
    \end{axis}
  \end{tikzpicture}
\end{frame}

\begin{frame}{Formulaire dérivées}
  \begin{center}
  \renewcommand{\arraystretch}{3} %donne la distance entre les lignes%
  \begin{tabular}{|l|C{1cm}|C{1cm}|C{1cm}|C{1cm}|C{1cm}|}
    \hline
    $f(x)$ & $x^{\alpha}$ & $\frac{1}{x}$ & $\sqrt{x}$ & $\ln(x)$ & $\e^x$\\
    \hline
    $f'(x)$ & $\alpha x^{\alpha-1}$ & $-\frac{1}{x^2}$ & $\frac{1}{2\sqrt{x}}$ 
            & $\frac{1}{x}$ & $\e^x$\\
    \hline
  \end{tabular}
\end{center}

\tiny
\begin{center}
  \renewcommand{\arraystretch}{2} %donne la distance entre les lignes%
  \begin{tabular}{|l|C{1cm}|C{1cm}|C{2.5cm}|C{1cm}|C{1cm}|C{1cm}|}
    \hline
    $f(x)$ & $\cos(x)$ & $\sin(x)$ & $\tan(x)$ & $\arcsin(x)$ & $\arccos(x)$ & $\arctan(x)$\\
    \hline
    $f'(x)$ & $-\sin(x)$ & $\cos(x)$ & $\frac{1}{\cos^2(x)}=1+\tan^2(x)$ 
            & $\frac{1}{\sqrt{1-x^2}}$ & $\frac{-1}{\sqrt{1-x^2}}$ & $\frac{1}{1+x^2}$\\
    \hline
  \end{tabular}
\end{center}  
\end{frame}

\end{document}
%%% Local Variables:
%%% mode: latex
%%% TeX-master: t
%%% End:
