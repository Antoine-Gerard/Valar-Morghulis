\documentclass[a4paper]{article}
\usepackage[utf8]{inputenc}
\usepackage[T1]{fontenc}
\usepackage[french]{babel}
\usepackage{graphicx}
\usepackage{float}
\usepackage{amsmath}
\usepackage{amsfonts}
\usepackage[overload]{empheq} 
\usepackage{lmodern}
\usepackage{pdfpages}
\usepackage{a4wide}
\usepackage[showlabels,sections,floats,textmath,displaymath]{preview}
\usepackage{hyperref}
\usepackage{algorithm2e}
\usepackage{fullwidth}
\usepackage{stmaryrd}
\usepackage{tikz}
\usepackage{enumitem}
\usepackage{bbold}
\usepackage{pgfplots}
\usetikzlibrary{patterns}
\usepgfplotslibrary{fillbetween}
\usepackage{ntheorem}
\usepackage{eurosym}
\usepackage{array,multirow,makecell}
\def\singleletter#1{\rotatebox[origin=B]{90}{#1}}
\makeatletter
\def\parseletters#1{\@tfor \@tempa := #1 \do {\kern2pt\singleletter{\@tempa}}}
\makeatother
\def\verticaltext#1{\rotatebox[origin=c]{-90}{\catcode`\~=13\def~{\vbox to 0.33em{}}\parseletters{#1}}}


\newcolumntype{R}[1]{>{\raggedleft\arraybackslash }b{#1}}
\newcolumntype{L}[1]{>{\raggedright\arraybackslash }b{#1}}
\newcolumntype{C}[1]{>{\centering\arraybackslash }b{#1}}

\theoremstyle{break}
\newtheorem{mydef}{Définition}[section]
\newtheorem{theo}{Théorème}[section]
\newtheorem*{Rem}{Remarque}
\newtheorem*{Proof}{preuve:}
\newtheorem{lemme}{Lemme}[section]
\newtheorem{prop}{Proposition}[section]

\DeclareMathOperator{\divg}{div} %operateur divergence
\DeclareMathOperator{\supp}{supp}
\DeclareMathOperator{\e}{e} %exponentielle



\newcommand{\norm}[1]{\left \Vert {#1} \right \Vert}
\newcommand{\scal}[2]{\left\langle {#1} , {#2} \right\rangle}

\newcommand{\J}[1]{\mathcal{J}(#1)}
\newcommand{\R}{\mathbb{R}}

\newcommand{\note}{$\bullet$ \textbf{Notes: }}

\begin{document}
\section{Intégration}
\subsection{Introduction: }
Si f est une fonction positive, continue sur un intervalle [a,b], on
peut l'approcher avec des fonctions en escalier et donner ainsi un
sens à ``l'aire sous la courbe''. On définit alors
\[
  \int \limits_{a}^{b} f(x) dx
\]
comme l'aire sous la courbe représentative de f.

\begin{center}
  \includegraphics[scale=0.4]{Integrale}
\end{center}

Toute fonction réelle continue f sur un intervalle [a,b] s'écrit comme
la différence de deux fonctions positives continues, à savoir $f^+(x)
= \max \limits_{x \in [a,b]} (f(x),0)$ et $f^-(x)
= \max \limits_{x \in [a,b]} (-f(x),0)$. Ainsi, nous pouvons définir
la notion d'intégrale pour une fonction f continue sur $[a,b]$:
\[
  \int \limits_{a}^b f(x)dx = \int \limits_a^b f^+(x) dx -
  \int \limits_a^b f^-(x) dx
\]

\note  Faire un schéma pour une fonction qui change de signe

\subsubsection{Propriétés de l'intégrale: }
\textbf{Relation de Chasles: } Pour tout c dans l'intervalle $[a,b]$,
on a :
\[
  \int \limits_a^b f(x) dx = \int \limits_a^c f(x) dx
  + \int \limits_c^b f(x) dx
\]

\textbf{Opérations algébriques: }
\begin{enumerate}[label = (\alph*)]
\item Si f et g sont continues sur [a,b] alors :
  \[
    \int \limits_a^b f(x)+g(x)dx
    = \int \limits_a^b f(x) dx + \int \limits_a^b g(x) dx
  \]
\item Si f est continue sur [a,b] et $\lambda \in \R$ alors:
  \[
    \int \limits_a^b \lambda f(x) dx
    = \lambda \int \limits_a^b f(x) dx 
  \]
\end{enumerate}

\note Ceci se prouve facilement pour les fonctions constantes. Ensuite
on étend aux fonctions en escalier et enfin le passage à la limite
prouve le cas général.

Si f est continue et $b<a$ on a:
\[
  \int \limits_a^b f(x) dx
  = - \int \limits_b^a f(x) dx 
\]

\subsubsection{Primitive d'une fonction: }
\begin{mydef}
  Soient f et F deux fonctions continues un intervalle $I=[a,b]$. On
  dit que \textbf{F est une primitive de f sur I} si F est dérivable
  sur I et F'=f 
\end{mydef}

\begin{theo}[Théorème fondamental du calcul intégral]
  Soit f une fonction continue sur un intervalle ouvert I=(a,b) de
  $\R$, à valeurs réelles, et soient $c<d$ deux points de I.
  \begin{enumerate}[label = (\alph*)]
  \item La fonction
    \[
      F: I ~ \rotatebox[origin=c]{180}{$\in$} ~ x \mapsto \int
      \limits_a^x f(t) dt
    \]
    est une primitive de f sur I: c'est la primitive de la fonction f
    s'annulant en a.
  \item Pour toute primitive G de f sur I, on a:
    \[
      \int \limits_c^d f(x)dx = G(d) - G(c)
    \]
  \end{enumerate}
\end{theo}

\begin{center}
  \begin{tikzpicture}
    \begin{axis}[enlargelimits=0.1
      xmin = -1,
      xmax =  1,
      ymin = 0,
      ymax =  2.1,
      xtick = {0.2},
      xticklabels = {$x_0$},
      ticklabel style = {font=\tiny , below},
      extra x ticks = {0.3},
      extra x tick labels = {$x_0+h$},
      extra x tick style = {tick label style={font=\tiny ,
          xshift=0.2cm, yshift = 0.04cm}},
      axis x line= center,
      axis y line= center,
      legend pos = south east
      ]
      \addplot[domain=-.15:1.05,blue, thick] {-x^2+2};

      \node at (axis cs:0.9,1) [anchor=west]
      {$\color{blue}{f}$};
      
      \addplot[name path=f,domain=0.2:0.3,blue] {-x^2+2};
      
      \path[name path=axis] (axis cs:0.2,0) -- (axis cs:0.3,0);
      
      \addplot [thick,color=blue,pattern = north east lines]
      fill between [ of=f and axis];

      \addplot [mark=none] coordinates {(0.2,0) (0.2,{-0.2^2+2})};
      \addplot [mark=none] coordinates {(0.3,0) (0.3,{-0.3^2+2})};
    \end{axis}
  \end{tikzpicture}
\end{center}

\note On pourra remarquer en regarder la figure ci-dessus que l'aire
entre $x_0$ et $x_0+h$ est environ $h*f(x_0)$ et donc plus h ``devient
petit'' plus le quotient $\frac{F(x_0+h) - F(x_0)}{h}$ est proche de
$f(x_0)$

\textbf{Conséquences}
\begin{enumerate}[label=(\alph*)]
\item Toute fonction continue admet une primitive sur un intervalle.
\item Si F est une primitive de f, alors pour tout c $\in \R,
  \tilde{F}(x) = F(x) + c$ est aussi une primitive de f. On note
  \[
    \int f(x) dx
  \]
  pour désigner une primitive de f
\item Si on connaît une primitive de f, alors l'aire $\int \limits_a^b
  f(x) dx$ se calcule comme $F(b) - F(a)$
\end{enumerate}

\textbf{Primitives à connaître: }
\begin{center}
\renewcommand{\arraystretch}{3} %donne la distance entre les lignes%
  \begin{tabular}{|l|C{2.5cm}|C{2.5cm}|}
    \hline
    $f(x)$ & $F(x)$ & $f'(x)$\\
    \hline
    0 & $c \in \R$ & 0\\
    \hline
    $x^{\alpha}, \alpha \neq -1 $ & $\frac{x^{\alpha+1}}{\alpha+1} + c $
                                  & $ (\alpha-1)x^{\alpha-1}$  \\
    \hline
    $\frac{1}{x}$ & $\ln(|x|) + c $ & $ \frac{-1}{x^2}$ \\
    \hline
    $\e^x$ & $\e^x + c$ & $e^x$ \\
    \hline
    $\cos(x)$ & $\sin(x) + c$ & $\-sin(x)$ \\
    \hline
    $\sin(x)$ & $-\cos(x) + c$ & $\cos(x)$ \\
    \hline
    $\tan(x)$ & $-\ln(\cos(x)) + c$ & $\frac{1}{\cos^2(x)}$\\
    \hline
    $\frac{1}{1+x^2}$ & $\arctan(x) + c$ &\\
    \hline
    $\frac{1}{\sqrt{1-x^2}}$ & $\arcsin(x) + c$ &\\
    \hline
    $\frac{-1}{\sqrt{1-x^2}}$ & $\arccos(x) + c$ &\\
    \hline
  \end{tabular}
\end{center}



\end{document}
%%% Local Variables:
%%% mode: latex
%%% TeX-master: t
%%% End:
