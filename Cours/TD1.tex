\documentclass{article}% autres choix : report, book
\usepackage[utf8]{inputenc}% encodage du fichier source
\usepackage[T1]{fontenc}% gestion des accents (pour les pdf)
\usepackage[francais]{babel}% rajouter éventuellement english, greek, etc.
\usepackage{textcomp}% caractères additionnels
\usepackage{amsmath,amssymb,amsthm}% pour les maths
\usepackage{pxfonts}% remplacer éventuellement par txfonts, fourier, etc.
\usepackage[a4paper]{geometry}% taille correcte du papier
\usepackage{graphicx}% pour inclure des images
\usepackage{xcolor}% pour gérer les couleurs
\usepackage{microtype}% améliorations typographiques
\usepackage{hyperref}% gestion des hyperliens
\usepackage{answers}
%\usepackage{ntheorem}

\hypersetup{pdfstartview=XYZ}% zoom par défaut

\newenvironment{questions}{\begin{enumerate}}{\end{enumerate}}
\newenvironment{sousquestions}{\begin{enumerate}}{\end{enumerate}}
\newcommand{\R}{\mathbb{R}}
\pagestyle{empty}

\newtheorem{Exc}{EXERCICE}
\Newassociation{correction}{Soln}{mycor}
\renewcommand{\Solnlabel}[1]{CORRIGE #1}
\def\exo#1{%
  \futurelet\testchar\MaybeOptArgmyexoo}
\def\MaybeOptArgmyexoo{
  \ifx[\testchar \let\next\OptArgmyexoo
  \else \let\next\NoOptArgmyexoo \fi \next}
\def\OptArgmyexoo[#1]{%
  \begin{Exc}[#1]\normalfont}
  \def\NoOptArgmyexoo{%
    \begin{Exc}\normalfont}
    \newcommand{\finexo}{\end{Exc}}
  \newcommand{\flag}[1]{}
  \newif\ifprof
  \newcommand{\entete}[1]



\DeclareMathOperator{\e}{e} %exponentielle

\begin{document}

\Opensolutionfile{mycor}[TD1_correc]

\proftrue

\begin{center}
  \textbf{Feuille de TD1: Continuité, dérivation et fonctions usuelles}
\end{center}

\exo{}[Calculer les dérivées  des fonctions suivantes]
\begin{align*}
    1) f(x) =& x & 2) f(x) =& x^2 & 3) f(x) =& x^3 & 4) f(x) =& x^4 \\
    5) f(x) =& x^{-1} & 6) f(x) =& x^{-2} & 7) f(x) =& x^{-3} & 8)
    f(x)=&x^{-4}\\
    9) f(x) =& 1 + x + x^{2016} & 10)f(x)=&\exp^{2\ln(x)} - x^2 &11) f(x) =& x^{-1} - \frac{1}{x} 
\end{align*}
\ifprof\par
\emph{Corrigé}\par
\begin{correction}[Corrigé exercice 1]
  \begin{align*}
    1) f'(x) =& 1 & 2) f'(x) =& 2x & 3) f'(x) =& 3x^2 & 4) f'(x) =& 4x^3 \\
    5) f'(x) =& -x^{-2} & 6) f'(x) =& -2x^{-3} & 7) f(x) =& -3x^{-4} & 8) f(x)=&-4x^{-5}\\
    9) f'(x) =& 1 + 2016x^{2015} & 10)f'(x)=& 0 &11)f'(x)=& 0 
  \end{align*}
  
\end{correction}
\fi  
\finexo

\medskip

\exo{}[Calculer les dérivées  des fonctions suivantes]
\begin{align*}
    1) f(x) =& (1+x)\sin(x)   & 2) f(x) =& \sin(x) \cos(x) & 3) f(x) =& sin^2(x) \\
    4) f(x) =& x \arctan(x)   & 5) f(x) =& \arctan^2(x)   & 6) f(x) =& x^2 \e^x\\
    7) f(x) =& \sin(x) \ln(x) & 8)f(x) =& \e^x \arcsin(x)  & 9) f(x) =& x \sin(x) \e^x 
\end{align*}
\ifprof\par
\emph{Corrigé}\par
\begin{correction}[Corrigé exercice 2]
  \begin{align*}
    1) f'(x) =& (1+x)\cos(x) + \sin(x)  & 2) f'(x) =& \cos^2(x)-\sin^2(x)
                                        & 3) f'(x) =& 2\cos(x)\sin(x) \\
    4) f'(x) =& \arctan(x) + \frac{1}{1+x^2} & 5) f'(x) =& 2\frac{\arctan(x)}{1+x^2}
                                             & 6) f'(x) =& \e^x(x^2 + 2x)\\
    7) f'(x) =& \cos(x) \ln(x) + \frac{sin(x)}{x} & 8)f'(x) =& \e^x(\arcsin(x) + \frac{1}{\sqrt{1-x^2}})
                                                  & 9)f'(x) =& \e^x(x\sin(x) + \sin(x) + x\cos(x)) 
\end{align*}
\end{correction}
\fi
\finexo

\medskip

\exo{}[Calculer les dérivées  des fonctions suivantes:]
\begin{align*}
  1) f(x) =& (1+x)^2  & 2) f(x) =& (5+x)^3 & 3) f(x) =& \frac{1}{1+x} & 4) f(x) =& \frac{1}{(1+x)^2}\\
  5) f(x) =& \frac{1+x}{2+x} & 6) f(x) =& \frac{x^2+2}{x^3+3} & 7) f(x) =& \sin(1+x)
                                                              & 8)f(x) =& \sin(42x) \\
  9) f(x) =& \sin(\frac{1}{x}) &10) f(x) =& \frac{1}{\sin(x)} & 11)f(x) =& \frac{\cos(x)}{\sin(x)}
                                                              & 12)f(x) =& \ln(1+x) \\
  13) f(x) =& \ln(\sin(x)) &14) f(x) =& \ln(\e(x)) & 15)f(x) =& \ln(\sqrt{1+x})
                                                   & 16)f(x) =& \e^{ax+b} \\
  17) f(x) =& \e^{\cos(x)} & 18) f(x) =& e^{\cos(2x)} & 19) f(x) =& \sqrt{x+x^2}
                                                      & 20) f(x) =& \frac{1}{\sqrt{x+x^2}}\\
  21) f(x) =& \frac{\sqrt{1+x}}{\sqrt{2+x}} \text{ (Etudier ensemble de définition) }
\end{align*}
\ifprof\par
\emph{Corrigé}\par
\begin{correction}
  \begin{align*}
    1) f'(x) =& 2(1+x)  & 2) f'(x) =& 3(5+x)^2 & 3) f'(x) =& \frac{-1}{(1+x)^2}
                                               & 4) f'(x) =& \frac{-2}{(1+x)^3}\\
    5) f'(x) =& \frac{1}{(2+x)^2} & 6) f'(x) =& \frac{-x^4-6x^2+6x}{(x^3+3)^2}
 &  7) f'(x) =& \cos(1+x)         & 8) f'(x) =& 42\cos(42x) \\
    9) f'(x) =& -\frac{1}{x^2}\cos(\frac{1}{x}) &10) f'(x) =& \frac{1}{\sin(x)}
 & 11) f'(x) =& \frac{-1}{\sin^2(x)}           &12) f'(x) =& \frac{1}{1+x} \\
   13) f'(x) =& \frac{\cos(x)}{\sin(x)} &14) f'(x) =& 1
 & 15) f'(x) =& \frac{1}{2(1+x)}        &16) f'(x) =& a\e^{ax+b} \\
   17) f'(x) =& -\sin(x)\e^(\cos(x))    &18) f'(x) =& -2\sin(x)e^{\cos(2x)}
 & 19) f'(x) =& \frac{x}{\sqrt{x+x^2}}  &20) f'(x) =& -\frac{x}{(x+x^2) \sqrt{x+x^2}}\\
   21) f'(x) =& \frac{1}{2(2+x)\sqrt{(1+x)(2+x)}}
  \end{align*}
\end{correction}
\fi
\finexo

\medskip

\exo{}[Calculer les limites en utilisant la définition de la dérivée]
\[
  \lim  \limits_{x \to 0} \frac{\sin(x)}{x},
\quad \lim  \limits_{x \to 0} \frac{\sin(x^2)}{x},
\quad \lim  \limits_{x \to 0} \frac{\sqrt{x+1} - 1}{x},
\quad \lim  \limits_{x \to 1} \frac{\sin(\pi x)}{x-1},
\quad \lim  \limits_{x \to 0} \frac{\e^x-1}{x},
\quad \lim  \limits_{n \to \infty} (1+\frac{1}{n})^n
\]
\ifprof\par
\emph{Corrigé}\par
\begin{correction}
  \begin{align*}
    &1) \lim \limits_{x \to 0} \frac{\sin(x)}{x} = \lim \limits_{x \to 0}
                                                   \frac{\sin(x)-\sin(0)}{x-0}
                                                   = (\sin(x))'_{|x=0} = 1 \\
    \\
    &2) \lim \limits_{x \to 0} \frac{\sin(x^2)}{x} = \lim \limits_{x \to 0}
                                                   \frac{\sin(x^2)-\sin(0)}{x-0}
                                                    = (\sin(x^2))'_{|x=0} = 0 \\
    \\
    &3) \lim \limits_{x \to 0} \frac{\sqrt{x+1}-1}{x} = \lim \limits_{x \to 0}
                                                        \frac{\sqrt{x+1}-\sqrt{0+1}}{x-0}
                                                     = (\sqrt{x+1})'_{|x=0} =
                                                        \frac{1}{2} \\
    \\
    &4) \lim \limits_{x \to 0} \frac{\e^{x}-1}{x} = \lim \limits_{x \to 0}
                                                        \frac{\e^{x}-\e^{0}}{x-0}
                                                  = (\e^{x})'_{|x=0} = 1 \\
  \end{align*}

  Pour le $5)$ c'est un peu plus subtil. Dans un premier temps on pose $x =
  \frac{1}{n}$. Ainsi on a:
  \[
    \lim  \limits_{n \to \infty} (1+\frac{1}{n})^n =
    \lim  \limits_{x \to 0} (1+x)^{\frac{1}{x}}
  \]

  Puis on pose $y = x-1$ et on a:
  \[
    \lim  \limits_{x \to 0} (1+x)^{\frac{1}{x}} =
    \lim  \limits_{y \to 1} y^{\frac{1}{y-1}} =
    \lim  \limits_{y \to 1} \e^{\frac{\ln(y)-ln(1)}{y-1}}=
    \e^{(ln(x))'_{|x=1}} = \e^1
  \]
\end{correction}
\fi
\finexo

\exo{}[Etude de $\arccos(x) + \arcsin(x)$]
~\\
Dans cette exercice on se propose d'étudier la fonction:
\[
  \begin{array}{cccccc}
     f & : & [-1,1] & \to & \R \\ 
       & & x & \mapsto & \arccos(x) + \arcsin(x) \\
  \end{array}
\]
\begin{enumerate}
\item Montrer que f est dérivable sur $]-1,1[$
\item Calculer $f'(x)$ pour tout x de $]-1,1[$
\item Calculer $f(0)$
\item Conclure
\end{enumerate}

\ifprof\par
\emph{Corrigé}\par
\begin{correction}
  ~ \\
  \begin{enumerate}
  \item f est la somme de deux fonctions dérivables sur $]-1,1[$ donc f est
    dérivable sur $]-1,1[$
  \item $f'(x) = \frac{-1}{\sqrt{1-x^2}} + \frac{1}{\sqrt{1-x^2}} = 0$
  \item $f(0) = \frac{\pi}{2} + 0 = \frac{\pi}{2}$
  \item $f'(0)= 0$ sur l'intervalle $]-1,1[$ donc f est constante sur $[-1,1]$
    et on a
    \[
      \forall x \in [-1,1] \qquad f(x) = \arccos(x) + \arcsin(x) = \frac{\pi}{2} = f(0)
    \]
  \end{enumerate}
\end{correction}
\fi
\finexo

\newpage
\setcounter{page}{1}
\Closesolutionfile{mycor}
\Readsolutionfile{mycor}

\end{document}
%%% Local Variables:
%%% mode: latex
%%% TeX-master: t
%%% End:
