\documentclass{article}% autres choix : report, book
\usepackage[utf8]{inputenc}% encodage du fichier source
\usepackage[T1]{fontenc}% gestion des accents (pour les pdf)
\usepackage[francais]{babel}% rajouter éventuellement english, greek, etc.
\usepackage{textcomp}% caractères additionnels
\usepackage{amsmath,amssymb,amsthm}% pour les maths
\usepackage{pxfonts}% remplacer éventuellement par txfonts, fourier, etc.
\usepackage[a4paper]{geometry}% taille correcte du papier
\usepackage{graphicx}% pour inclure des images
\usepackage{xcolor}% pour gérer les couleurs
\usepackage{microtype}% améliorations typographiques
\usepackage{hyperref}% gestion des hyperliens
\usepackage{answers}
\usepackage{tikz, tkz-tab}
\usepackage{cancel}

\hypersetup{pdfstartview=XYZ}% zoom par défaut


\newenvironment{questions}{\begin{enumerate}}{\end{enumerate}}
\newenvironment{sousquestions}{\begin{enumerate}}{\end{enumerate}}
\newcommand{\R}{\mathbb{R}}
\pagestyle{empty}

\newtheorem{Exc}{EXERCICE}
\Newassociation{correction}{Soln}{mycor}
\renewcommand{\Solnlabel}[1]{CORRIGE #1}
\def\exo#1{%
  \futurelet\testchar\MaybeOptArgmyexoo}
\def\MaybeOptArgmyexoo{
  \ifx[\testchar \let\next\OptArgmyexoo
  \else \let\next\NoOptArgmyexoo \fi \next}
\def\OptArgmyexoo[#1]{%
  \begin{Exc}[#1]\normalfont}
  \def\NoOptArgmyexoo{%
    \begin{Exc}\normalfont}
    \newcommand{\finexo}{\end{Exc}}
  \newcommand{\flag}[1]{}
  \newif\ifprof
  \newcommand{\entete}[1]


 
\newcommand{\note}{$\bullet$ \textbf{Notes: }}
\DeclareMathOperator{\e}{e} %exponentielle

\begin{document}

\Opensolutionfile{mycor}[TD_correc]

\proftrue

\tableofcontents
\clearpage

\newpage

\begin{center}
  \textbf{Feuille de TD supplémentaire}
\end{center}

\section{Continuité, Dérivation, fonctions usuelles}

\exo{}[Calculer les dérivées  des fonctions suivantes]
\begin{align*}
    1) f(x) =& (3x^2+7)\ln(x)   & 2) f(x) =& \frac{e^x}{x^2+1} & 3) f(x) =& \sqrt{x^4+8} \\
    4) f(x) =& \cos(2-x)   & 5) f(x) =& \ln(7-x^2)
                           & 6) f(x) =& (\sin(x) + 3)^4\\
    7) f(x) =& 3x^2 \ln(x) & 8)f(x) =& \e^{\sqrt{x}}  & 9) f(x) =& \arccos(\arctan(x)) 
\end{align*}
\ifprof\par
\emph{Corrigé}\par
\begin{correction}[Corrigé exercice 2]
  \begin{align*}
    1) f'(x) =& \frac{(3x^2+7)}{x} + 6x \ln(x)
    & 2) f'(x) =& \frac{e^x (x^2 + 1 - 2x)}{(x^2+1)^2} = \frac{e^x(x-1)^2}{(x^2+1)^2}
    & 3) f'(x) =& \frac{4x^3}{2\sqrt{x^4+8}} = \frac{2x^3}{\sqrt{x^4+8}}\\
    4) f'(x) =& \sin(2-x)
    & 5) f'(x) =& \frac{-2x}{7-x^2}
    & 6) f'(x) =& 4 \cos(x) (\sin(x) + 3)^3 \\
    7) f'(x) =& 6x \ln(x) + \frac{3x^2}{x} = 6x \ln(x) + 3x
    & 8)f'(x) =& \frac{ e^{ \sqrt{x} }}{ 2\sqrt{x} }
    & 9)f'(x) =& \frac{1}{1+x^2} \frac{-1}{ \sqrt{ 1 - \arctan^2(x) } }
  \end{align*}
\end{correction}
\fi
\finexo

\exo{}[Calculer les limites en utilisant la définition de la dérivée]
\[
\lim  \limits_{ x \to -\frac{\pi}{2} } \frac{ \cos(x) }{ x + \frac{\pi}{2}},
\qquad \lim  \limits_{ x \to \frac{\pi}{3} }\frac{ \sin(x)-\frac{\sqrt{3}}{2} }{x-\frac{\pi}{3}},
\]
\ifprof\par
\emph{Corrigé}\par
\begin{correction}
  \begin{align*}
    &1) \lim \limits_{x \to -\frac{\pi}{2}}
      \frac{\cos(x)}{x+\frac{\pi}{2}} = \lim \limits_{x \to -\frac{\pi}{2}}
      \frac{\cos(x)-\cos(\frac{-\pi}{2})}{x-(-\frac{\pi}{2})}
      = (\cos(x))'_{|x=-\frac{\pi}{2}} = 1 \\
    \\
    &2) \lim \limits_{x \to \frac{\pi}{3}}
      \frac{\sin(x)-\frac{\sqrt{3}}{2}}{x-\frac{\pi}{3}}
      = \lim \limits_{x \to \frac{\pi}{3}}
      \frac{\sin(x)-\sin(\frac{\pi}{3})}{x-\frac{\pi}{3}}
      = (\sin(x))'_{|x=\frac{\pi}{3}} = \frac{1}{2} \\
  \end{align*}
\end{correction}
\fi
\finexo

\exo{}[Etude de $\arccos(x) + \arcsin(x)$]
~\\
Dans cette exercice on se propose d'étudier la fonction:
\[
  \begin{array}{cccccc}
     f & : & [-1,1] & \to & \R \\ 
       & & x & \mapsto & \arccos(x) + \arcsin(x) \\
  \end{array}
\]
\begin{enumerate}
\item Montrer que f est dérivable sur $]-1,1[$
\item Calculer $f'(x)$ pour tout x de $]-1,1[$
\item Calculer $f(0)$
\item Conclure
\end{enumerate}

\ifprof\par
\emph{Corrigé}\par
\begin{correction}
  ~ \\
  \begin{enumerate}
  \item f est la somme de deux fonctions dérivables sur $]-1,1[$ donc f est
    dérivable sur $]-1,1[$
  \item $f'(x) = \frac{-1}{\sqrt{1-x^2}} + \frac{1}{\sqrt{1-x^2}} = 0$
  \item $f(0) = \frac{\pi}{2} + 0 = \frac{\pi}{2}$
  \item $f'(0)= 0$ sur l'intervalle $]-1,1[$ donc f est constante sur $[-1,1]$
    et on a
    \[
      \forall x \in [-1,1] \qquad f(x) = \arccos(x) + \arcsin(x) = \frac{\pi}{2} = f(0)
    \]
  \end{enumerate}
\end{correction}
\fi
\finexo

\exo{}
~\\
Étudier la suite $u_n = (1+\frac{1}{n})^n$ en exprimant $u_n$ à l'aide
de $\ln(.)$ et $\exp(.)$ et en posant $n=\frac{1}{x}$.

\note: On pourra découper l'exercice comme suit:
\begin{enumerate}
\item Montrer que $u_n = g(\frac{1}{n})$ avec g une fonction dont on
  précisera l'ensemble de définition. (Indications: poser $x = \frac{1}{n}$)
\item Donner l'ensemble de dérivabilité de g et donner sa dérivée sur
  cet ensemble.
\item Soit $x \in ]0,+\infty[$. Donner le tableau de variations de $x \ln(x) - x$.
\item Dresser le tableau de variation de g. 
\item Conclure quant à $u_n$ (Sens de variation, limite)
\end{enumerate}

\ifprof\par
\emph{Corrigé}\par
\begin{correction}
  ~ \\
  On pose $x=\frac{1}{n}$ et on étudie la fonction
  \[
    \begin{array}{cccccc}
      g & : & ]0,1] & \to & \R \\ 
        & & x & \mapsto & (1+x)^{\frac{1}{x}} = \e^{\frac{\ln(1+x)}{x}} \\
    \end{array}
  \]

  On a $u_n = g(\frac{1}{n})$. $g$ est dérivable sur $]0,1]$ et
  \[
    g'(x) = \e^{\frac{\ln(1+x)}{x}} \Big( -\frac{\ln(1+x)}{x^2} +
            \frac{1}{x(1+x)} \Big) 
          = \e^{\frac{\ln(1+x)}{x}} \Big( \frac{-(x+1)\ln(1+x) +
            x}{x^2(1+x)} \Big)
  \]

  Étude de $f : x \mapsto x\ln(x) - x$ sur $]0,+\infty[$
  \begin{center}
    \begin{tikzpicture}
      \tkzTab{$x$ / 1 , $f'(x)=\ln(x)$ / 1, $f(x)$ / 1.5}{$0$,
        $1$, $+\infty$}{, -, z, +, }{+C /$0$ , -/ -1, +/ $+\infty$}
    \end{tikzpicture}
  \end{center}

  Ainsi on peut conclure que $\forall x \in \R^+_*,~ x\ln(x) \geq
  x-1$. Donc si on revient à $g'(x)$ on a $(x+1) \ln(1+x) \geq x
  ~\forall x \in ]0,1]$ et donc $x - (x+1) \ln(1+x) <=0$. D'où:
  \begin{center}
    \begin{tikzpicture}
      \tkzTabInit{$x$ / 1 , $f(x)$ / 1.5}{$0$,$1$}
      \tkzTabVar{+C / $\e^1$, -/ 2}
    \end{tikzpicture}
  \end{center}

  Ainsi puisque $u_n = g(\frac{1}{n})$ on a $u_n \leq u_{n+1} \quad (\frac{1}{n}>\frac{1}{n+1})$ et
  $\lim \limits_{n \to \infty} = e^1$
\end{correction}

\fi
\finexo

\newpage
\section{Intégration}
\exo{}
~\\
\begin{questions}
\item Calculer en effectuant le changement de variable x = cos(t)
  \[
    I = \int_{-1}^{1} \sqrt{1-x^2} dx \qquad \text{\textbf{Indication:
      } $\cos^2(t) + \sin^2(t) = 1$ et $\sin^2(t) = \frac{1-\cos(2t)}{2}$}
  \]
\item Calculer en effectuant le changement de variable  $t = \sqrt(x)$
  \[
    I = \int_{1}^{4} \frac{1-\sqrt{x}}{\sqrt{x}} dx
  \]
\item Calculer en effectuant le changement de variable  $t = \ln(x)$
  \[
    I = \int_{1}^{\e^1} \frac{(\ln(x))^n}{x} dx
  \]
\item Calculer en effectuant le changement de variable  $t = \sqrt{e^x}$
  \[
    I = \int_{0}^{\ln(3)} \frac{\e^x}{(\e^x+1)\sqrt{\e^x}} dx
  \]
\item Calculer en effectuant une I.P.P.
  \[
    \int_0^{\pi} x \sin(x) dx
  \]
\item Calculer en effectuant une I.P.P.
  \[
    \int_0^{\frac{\pi}{2}} x \cos(x) dx
  \]
\item Calculer en effectuant une I.P.P. et la question 5)
  \[
    \int_0^{\pi} x^2 \cos(x) dx
  \]
\item Calculer en effectuant une I.P.P. la primitive de $\arctan(x)$
  (\textbf{Indications: } Poser $u(x) = \arctan(x)$ et $v'(x) = 1$)
\item Calculer en effectuant une I.P.P.
  \[
    \int_0^{\pi} \cos(x) \e^x \qquad (\text{\textbf{Indications: }
      effectuer l'I.P.P deux fois)}
  \]
\end{questions}
\ifprof\par
\emph{Corrigé}\par
\begin{correction}
  ~\\
  \begin{questions}
  \item $x = \cos(t) \Rightarrow dx = -\sin(t)dt$
    Ainsi:
    \[
      \int_{-1}^{1} \sqrt{1-x^2} dx
      = \int_{\pi}^{0} \sqrt{1-\underbrace{\cos^2(t)}_{x=cos(t)}}
      \times \underbrace{-\sin(t) dt}_{dx} 
    \]

    Lorsque $x=-1$ on a $t = \arccos(-1) = \pi$ et lorsque $x=1$ on a
    $t=\arccos(1) = 0$. D'où les nouvelles bornes de l'intégrale.
    Et donc:
    \begin{eqnarray*} 
      \int_{-1}^{1} \sqrt{1-x^2} dx = \int_0^{\pi} \sin^2(t) dt
      &=& \int_0^{\pi} \frac{1 - \cos(2t)}{2}dt \text{ (voir
          Indication)} \\
      &=& \frac{1}{2}\big[ t - \frac{\sin(2t)}{2} \big]_0^{\pi}\\
      &=& \frac{\pi}{2}
    \end{eqnarray*}
  \item $t=\sqrt{x} \Rightarrow dt = \frac{1}{2\sqrt{x}}dx \Rightarrow dx =
    2\sqrt{x}dt$ Ainsi:
    \[
      \int_{1}^{4} \frac{1+\sqrt{x}}{\sqrt{x}}dx =
      \int_{1}^{2} \frac{1+t}{\cancel{\sqrt{x}}} \underbrace{2
        \cancel{\sqrt{x}} dt}_{dx}
      = \int_{1}^{2} (1+t) dt
    \]
    
    Lorsque $x=1$ on a $t = \sqrt{1} = 1$ et lorsque $x=4$ on a
    $t=\sqrt{4} = 2$. D'où les nouvelles bornes de l'intégrale.
    Et donc:
    \begin{eqnarray*}
      \int_{1}^4 \frac{1+\sqrt{x}}{\sqrt{x}}dx = \int_1^2 (1+t) dt
      &=& \big[ t + \frac{t^2}{2} \big]_1^2 \\
      &=& 2 + \frac{4}{2} - (1 + \frac{1}{2}) \\
      &=& \frac{5}{2}
    \end{eqnarray*}
  \item $ t = \ln(x) \Rightarrow dt = \frac{dx}{x} \Rightarrow dx = x dt$
    Ainsi:
    \[
      \int_1^{\e^1} \frac{(\ln(x))^n}{x}dx = \int_0^1 \frac{\overbrace{t^n}^{t =
          \ln(x)}}{\cancel{x}} \underbrace{\cancel{x} dt}_{=dx}
    \]

    Lorsque $x=1$ on a $t = \ln(1) = 0$ et lorsque $x=\e^1$ on a
    $t=\ln(\e^1) = 1$. D'où les nouvelles bornes de l'intégrale.
    Et donc:
    \begin{eqnarray*}
      \int_1^{\e^1} \frac{(\ln(x))^n}{x} dx = \int_0^1 t^n dt &=& \big[
      \frac{t^{n+1}}{n+1} \big]_0^1 \\
      &=& \frac{1}{n+1}
    \end{eqnarray*}
    \item $t = \sqrt{\e^x} \Rightarrow dt = \frac{\e^x}{2\sqrt{\e^x}}
      \Rightarrow dx = \frac{2\sqrt{\e^x}}{\e^x} dt$ Ainsi,
      \[
        \int_0^{\ln(3)} \frac{ \e^x }{ (\e^x+1) \sqrt{e^x} } dx =
        \int_1^{\sqrt{3}} \frac{ \cancel{\e^x} }{ (u^2+1) \cancel{\sqrt{\e^x}} }
        \overbrace{ 2\frac{\cancel{ \sqrt{\e^x} }}{ \cancel{\e^x} } du}_{=dx}
      \]
      
      Lorsque $x=0$ on a $t = \sqrt{\e^0} = 1$ et lorsque $x=\ln(3)$ on a
      $t=\sqrt{\e^{\ln(3)}} = \sqrt{3}$. D'où les nouvelles bornes de l'intégrale.
      Et donc:
      \begin{eqnarray*}
        \int_0^{\ln(3)} \frac{ \e^x }{ (\e^x+1) \sqrt{e^x} } dx =
        2\int_1^{\sqrt{3}} \frac{du}{1+u^2}
        &=& 2\big[ \arctan(u) \big]_1^{\sqrt{3}} \\
        &=& 2(\arctan(\sqrt{3}) - \arctan(1)) \\
        &=& 2(\frac{\pi}{3} - \frac{\pi}{4}) \\
        &=& \frac{\pi}{6}
      \end{eqnarray*}
    \item On pose $u(x) = x$ et $v'(x) = \sin(x)$. On a donc $u'(x) = 1$ et
      $v(x) = -\cos(x)$. En utilisant la formule d'I.P.P. on trouve:
      \begin{eqnarray*}
        \int_0^{\pi} x \sin(x) dx = \big[ -x \cos(x)
        \big]_0^{\pi} - \int_0^{\pi} (-\cos(x)) dx
        &=& -\pi \cos(\pi)-(-0 \cos(0)) +  \underbrace{\big[ \sin(x)
            \big]_0^{\pi}}_{=0} \\
        &=& \pi
      \end{eqnarray*}
    \item On pose $u(x) = x$ et $v'(x) = \cos(x)$. On a donc $u'(x) = 1$ et
      $v(x) = \sin(x)$. En utilisant la formule d'I.P.P. on trouve:
      \begin{eqnarray*}
        \int_0^{\frac{\pi}{2}} \underbrace{x \cos(x)}_{u(x)v'(x)} dx = \underbrace{\big[ x \sin(x)
        \big]_0^{\frac{\pi}{2}}}_{[u(x)v(x)]} - \int_0^{\frac{\pi}{2}} \underbrace{(\sin(x))}_{u'(x)v(x)} dx
        &=& \frac{\pi}{2} \sin(\frac{\pi}{2})- (0 \sin(0)) -  \big[ - \cos(x)
            \big]_0^{\frac{\pi}{2}} \\
        &=& \frac{\pi}{2} -1 = \frac{\pi-2}{2}
      \end{eqnarray*}
    \item On pose $u(x) = x^2$ et $v'(x) = \cos(x)$. On a donc $u'(x) = 2x$ et
      $v(x) = \sin(x)$. En utilisant la formule d'I.P.P. on trouve:
      \begin{eqnarray*}
        \int_0^{\pi} x^2 \cos(x) dx = \underbrace{\big[ x^2 \sin(x)
        \big]_0^{\pi}}_{=0} - 2\underbrace{ \int_0^{\pi} x \sin(x) dx}_{= \pi
        \text{ voir Question 5} }
        &=& 2 \pi
      \end{eqnarray*}
    \item On pose $u(x) = e^{x}$ et $v'(x) = \cos(x)$. On a donc $u'(x) = e^x$ et
      $v(x) = \sin(x)$. En utilisant la formule d'I.P.P. on trouve:
      \begin{eqnarray*}
        \int_0^{\pi} e^x \cos(x) dx = \underbrace{ \big[ e^x \sin(x)
        \big]_0^{\pi}}_{=0} - \int_0^{\pi} e^x \sin(x) dx
      \end{eqnarray*}

      On effectue une nouvelle I.P.P. en posant $u(x) = \e^x$ et $v'(x) =
      \sin(x)$. On a donc $u'(x) = \e^x$ et $v(x) = -\cos(x)$. La formule
      d'I.P.P. nous donne:
      \begin{eqnarray*}
        \int_0^{\pi} e^x \cos(x) dx = - \int_0^{\pi} e^x \sin(x) dx
        &=& - \Big( \big[ -e^x \cos(x) \big]_0^{\pi} - \int_0^{\pi} \e^x (-\cos(x))
            dx \Big)\\
        &=&  \e^{\pi} \cos(\pi) - \e^0 \cos(0) - \int_0^{\pi} \e^x \cos(x) dx 
      \end{eqnarray*}

      Finalement, on a :
      \[
        \int_0^{\pi} e^x \cos(x) dx = -(\e^{\pi} + 1) - \int_0^{\pi} e^x \cos(x)
        dx
        \Leftrightarrow
        2 \int_0^{\pi} e^x \cos(x) dx = -(\e^{\pi} + 1)
        \Leftrightarrow
        \int_0^{\pi} e^x \cos(x) dx = -\frac{(\e^{\pi} + 1)}{2}
      \]
    \end{questions}
\end{correction}
\fi
\finexo

\exo{}
~\\
Soient $\lambda, T>0$. Calculer $I(T) = \int_0^T \lambda \e^{-\lambda
  t} dt$ et $E(T) = \int_0^T  t \lambda \e^{-\lambda t}$

Calculer les limites de $I(T)$ et $E(T)$ lorsque T tend vers $+\infty$
\ifprof\par
\emph{Corrigé}\par
\begin{correction}
  \begin{eqnarray*}
    I(T) &=& \int_0^T \lambda \e^{-\lambda t} dt \\
         &=& \big[ -e^{\lambda t} \big]_0^T \\
         &=& -e^{\lambda T} + e^0 \\
         &=& 1 - e^{-\lambda T}
  \end{eqnarray*}

  Passons à $E(T)$. Pour cette intégrale on utilise un I.P.P. en posant:
  \[
    u(t) = t \qquad \text{ et } \qquad v'(t) = \lambda \e^{-\lambda t} 
  \]

  Ainsi on trouve,
  \[
    u'(t) = 1  \qquad \text{ et } v(t) = -\e^{-\lambda t} 
  \]

  Et finalement, en utilisant la formule d'I.P.P. ($\int uv'dt = [uv]
  - \int u'vdt$)
  \begin{eqnarray*}
    E(T) &=& \int_0^T t \lambda \e^{-\lambda t} dt \\
         &=& \big[ -t e^{\lambda t} \big]_0^T - \int_0^T -\e^{-\lambda
    t}\\
         &=& T \e^{-\lambda T} - \big[ \frac{ \e^{-\lambda t} }{\lambda} \big]_0^T\\
         &=& T \e^{-\lambda T} - \frac{\e^{-\lambda T} -
             \e^0}{\lambda} \\
         &=& T \e^{-\lambda T} + \frac{1 - \e^{-\lambda T}}{\lambda}
  \end{eqnarray*}

  Maintenant passons aux limites lorsque $T \to +\infty$
  \[
    \lim \limits_{T \to +\infty} I(T) = \lim \limits_{T \to +\infty}
        (1 - e^{-\lambda T}) = 1 - 0 = 1              
  \]
  \[
    \lim \limits_{T \to +\infty} E(T) = \lim \limits_{T \to +\infty}
    \lim \limits_{T \to +\infty} T \e^{-\lambda T} + \frac{1 -
      \e^{-\lambda T}}{\lambda}
    = \underbrace{ \lim \limits_{T \to +\infty} T \e^{-\lambda
       T} }_{\text{croissance comparée } = 0}
   + \lim \limits_{T \to +\infty} \frac{1 - \e^{-\lambda T}}{\lambda}
   = 0 + \frac{1 - 0}{\lambda} = \frac{1}{\lambda}
 \]

 \textbf{Remarques: } Pour les gens ayant fait un peu de probabilités
 on pourra remarquer que la fonction $f: t \mapsto \lambda
 \e^{-\lambda t}$ n'est
 rien d'autre que la densité de probabilité de la loi
 exponentielle. $\lim \limits_{T \to \infty} E(T)$ est en fait
 l'espérance d'une variable aléatoire suivant la loi exponentielle et
 il est donc normal de trouver $\frac{1}{\lambda}$. Ainsi si vous
 essayez de calculer $\lim \limits_{T \to \infty} \int_0^T t^2 \lambda
 \e^{-\lambda t} dt - \frac{1}{\lambda}^2$ vous devriez trouver $\frac{1}{\lambda^2}$ c'est
 à dire la variance d'une V.A. suivant la loi exponentielle.  
\end{correction}
\fi
\finexo

\exo{}
~\\
Soient $n \in \mathbb{N}$ et $x>0$. On pose:
\[
  \mathcal{I}_n(x) = \int_0^x t^n \e^{-t} dt \qquad  \text{ et } \qquad \mathcal{J}_n(x) = \lim
  \limits_{x \to +\infty} \mathcal{I}_n(x)
\]
Établir une relation de récurrence (sur n) vérifiée par $\mathcal{I}_n(x)$. En
déduire une relation de récurrence vérifiée par $\mathcal{J}_n(x)$. Enfin
calculer $\mathcal{J}_n(x)$ pour tout $n>0$
\ifprof\par
\emph{Corrigé}\par
\begin{correction}
  ~\\
  Ici on cherche une relation de réccurence sur n pour
  $\mathcal{I}_n = \int_0^x t^n \e^{-t} dt$. Pour cela il faudrait
  faire apparaitre le terme $t^{n-1}\e^{-t}$ dans l'intégrale
  $\mathcal{I}_n$ c'est à dire dérivée $t^n$. Nous avons une formule
  qui nous permet d'avoir une relation entre intégrale et dérivée qui
  est l'I.P.P. Essayons donc l'IPP en posant:
  \[
    u(t) = t^n \qquad \text{ et } \qquad v'(t) = \e^{-t}
  \]

  On a donc:
  \[
    u'(t) = n t^{n-1} \qquad \text{ et } \qquad v(t) = -\e^{-t}
  \]
  
  Ainsi en utilisant la formule d'IPP on a:
  \begin{eqnarray*} 
    \mathcal{I}_n(x) &=& \int_0^x t^n \e^{-t} dt \\
                  &=& \big[ -t^n \e^{-t} \big]_0^x - \int_0^x nt^{n-1}
                      \times (-e^{-t})dt \\
                  &=& x^n \e^{-x} + n \underbrace{\int_0^x t^{n-1}
                      e^{-t}dt}_{\mathcal{I}_{n-1}(x)}    \\
                  &=& x^n e^{-x} + n\mathcal{I}_{n-1}
  \end{eqnarray*}

  Maintenant, interessons nous à $\mathcal{J}_n(x) = \lim
  \limits_{x\to +\infty} \mathcal{I}_n(x) $
  \[
    \mathcal{J}_n(x) = \lim \limits_{x\to +\infty} \mathcal{I}_n(x)
    = \lim \limits_{x\to +\infty} (x^n e^{-x} + n\mathcal{I}_{n-1}(x))
    = \underbrace{\lim \limits_{x\to +\infty} x^n
      e^{-x}}_{\text{croissance comparée} = 0}
    + \underbrace{\lim \limits_{x\to +\infty}
      n\mathcal{I}_{n-1}(x)}_{n\mathcal{J}_{n-1}(x)}
    = n\mathcal{J}_{n-1}(x)
  \]

  On peut donc en déduire:
  \[
    \mathcal{J}_n(x) =  n \mathcal{J}_{n-1}(x)
    = n(n-1)\mathcal{J}_{n-2}(x)
    = \dots
    = n(n-1) \dots 2.1 \mathcal{J}_0(x)
    = n! ~\mathcal{J}_0(x)
  \]

  Or: $\mathcal{J}_0(x) = \lim \limits_{x \to +\infty} \int_0^x t^0
  e^{-t}dt = \lim \limits_{x \to +\infty} \int_0^x \e^{-t}dt = \lim
  \limits_{x \to +\infty} \big[ -e^{-t} \big]_0^x = \lim \limits_{x \to +\infty}
  -e^{-x} - (-e^0) = 0 + 1 = 1$

  \vspace{1\baselineskip}
  
  Finalement pour tout $n \geq 1 \quad \mathcal{J}_n(x) = n!$
\end{correction}
\fi
\finexo


\exo{}
~\\
Calculer: 
\[
  \int \frac{dx}{x \ln(x) \ln(\ln(x))} 
\]
\ifprof\par
\emph{Corrigé}\par
\begin{correction}
  ~\\
  On commence par poser $u=\ln(x)$. On a donc $du=\frac{dx}{x} \Rightarrow dx =
  x du$. Ainsi, l'intégrale devient:
  \[
    \int \frac{dx}{x \ln(x) \ln(\ln(x))} =
    \int \frac{\overbrace{\cancel{x} du}^{dx = xdu}}{\cancel{x} \underbrace{u \ln(u)}_{u = \ln(x)}} 
  \]
  
  Maintenant, il y a deux façon de calculer la nouvelle intégrale que l'on vient
  de trouver. La première façon est de voir que l'intégrale est de la forme
  $\int \frac{f'(u)}{f(u)}$ avec $f(u) = \ln(u)$. En effet, si $f(u) = \ln(u)$
  alors $f'(u) = \frac{1}{u}$ et donc $\frac{f'(u)}{f(u)} =
  \frac{1}{u\ln(u)}$. Ainsi la primitive de cette fonction est $\ln(|f(u)|) =
  \ln(|\ln(u)|)$. Finalement, puisque $u=\ln(x)$, on trouve pour primitive de
  $\frac{1}{x\ln(x) \ln(\ln(x))}$ la fonction $F(x) = \ln(|\ln(\ln(x))|)$

  La deuxième façon de faire est de faire un nouveau changement de variable. On
  pose cette fois $t=\ln(u)$. On a donc $dt=\frac{du}{u} \Rightarrow du =
  u dt$ et l'intégrale
  devient:
  \[
    \int \frac{du}{u \ln(u)} =
   \int \frac{\overbrace{\cancel{u} dt}^{du = udt}}{\cancel{u} \underbrace{t}_{t
       = \ln(u)}}
   = \int \frac{dt}{t}
 \]

 On reconnait ici la dérivée de la fonction $\ln(t)$ donc
 \[
   \int \frac{dt}{ln(t)} = \ln(|t|)
 \]

 Ainsi, puisque $t = \ln(u)$ et $u = \ln(x)$ on arrive à $t = \ln(\ln(x))$ et:
 \[
    \int \frac{dx}{x \ln(x) \ln(\ln(x))} = \ln(|\ln(\ln(x))|)
 \]
\end{correction}
\finexo

\newpage
\setcounter{page}{1}
\Closesolutionfile{mycor}
\Readsolutionfile{mycor}

\end{document}
%%% Local Variables:
%%% mode: latex
%%% TeX-master: t
%%% End:
