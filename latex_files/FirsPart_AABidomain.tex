
\chapter{Analyse asymptotique bidomaine: Coeur+Torse}
\label{cha:analyse-asympt-bidom}

\section{Modèle Bidomaine}

On considère une pièce de tissu cardiaque idéalisée $\mathbb{H}$ ayant une épaisseur $h>0$. On définit $\mathbb{H}\ = \omega \times (0,h) \subset \mathbb{R}^3$, où $\omega$ est un ensemble ouvert de $\mathbb{R}^2$. De plus nous faisons l'hypothèse que $\mathbb{H} \subset \Omega$ et que $\mathbb{T} = \Omega - \mathbb{\overline{H}}$ représent le thorax. Considérons à présent les équations bidomaine: 

\begin{empheq}[left=\empheqlbrace]{align}
  \label{eq:bidomaine}
  &\divg((\sigma_i+\sigma_e) \nabla u_e = -\divg(\sigma_i \nabla v) \quad & \mathbb{H} \times (0,T)\\
  &A(C \partial_t v + f(v,w)) = \divg(\sigma_i \nabla(u_e + v)) \quad & \mathbb{H} \times (0,T) \\
  &\partial_t w = g(v,w) \quad & \mathbb{H} \times (0,T)\\
  &\divg(\sigma_T \nabla u_T) = 0 \quad & \mathbb{T} \times (0,T)
\end{empheq}

De plus nous avons les conditions limites:
\begin{empheq}[left=\empheqlbrace]{align}
  \label{eq:bidomaine}
  &(\sigma_i \nabla v) \cdot n = -(\sigma_i \nabla u_e) \cdot n \quad & \partial \mathbb{H} \\
  &(\sigma_e \nabla u_e) \cdot n = (\sigma_T \nabla u_T) \cdot n  \quad & \partial \mathbb{H} \\
  & u_e = u_T \quad & \partial \mathbb{H} \\
  & (\sigma_T \nabla u_T) \cdot n = 0 \quad & \partial \mathbb{T}
\end{empheq}


\subsection{Adimensionnement des équations}
\label{Adim}

Nous réécrivons ici les équations bidomaine dans une version adimensionnée. En fixant des dimensions caractéristiques $t_0$, $x_0$ et en considérant l'épaisseur $h>0$ de notre pièce de tissu, nous définissons d'abord les variables adimensionnée : 
\[
\overline{x} = \frac{x'}{x_0}, \quad \overline{t} = \frac{t}{t_0}, \quad \overline{z} = \frac{z}{h}
\]

Le changement d'échelle sur les variables spatiales transforme le domaine $\mathbb{H} = \omega \times (0,h)$ en $\overline{\omega} \times (0,1)$ où $\overline{\omega} = \frac{1}{x_0} \omega \subset \mathbb{R}^2$. Nous appliquons également les changements d'échelles suivant aux quantités physiques : 
\[
\overline{u_k}(\overline{t}, \overline{x}, \overline{z}) = \frac{u_k(t,x',z) - u_r}{\delta u}, ~ k = e, i, T
\]

\[
\overline{\sigma}_k = \frac{1}{\sigma_0} \sigma'_k, \quad \overline{\sigma}^{(3)}_k = \frac{\sigma^{(3)}_k}{\sigma_0} \sigma'_k
\]

\[
\overline{f}(\overline{v}, \overline{w}) = \frac{1}{f_0} f(v,w), \quad \overline{g}(\overline{v}, \overline{w}) = t_0 g(v,w)
\]

Les équations bidomaines s'écrivent alors : 
\begin{empheq}[left=\empheqlbrace]{align*}
  \label{eq:adim_bidomaine}
  &\divg_{\overline{x}}((\overline{\sigma}_i+\overline{\sigma}_e) \nabla_{\overline{x}} \overline{u}_e) 
  + \frac{x_0^2}{h^2} \overline{\sigma}_{(i+e)}^{(3)} \partial_{\overline{z} \overline{z}} \overline{u}_e
  = -\divg_{\overline{x}}(\overline{\sigma}_i \nabla \overline{v})
  + \frac{x_0^2}{h^2} \overline{\sigma}_{i}^{(3)} \partial_{\overline{z} \overline{z}} \overline{v}\\
  &\frac{AC x_0^2}{t_0 \sigma_0}( \partial_{\overline{t}} \overline{v} 
  + \frac{f_0 t_0}{C \delta u} \overline{f}(\overline{v},\overline{w})) 
  = \divg_{\overline{x}}(\overline{\sigma}_i \nabla(\overline{u}_e + \overline{v})) 
  + \frac{x_0^2}{h^2} \overline{\sigma}_{i}^{(3)} \partial_{\overline{z} \overline{z}} \overline{v}  \\
  &\partial_{\Ov{t}} \Ov{w} = g(\Ov{v},\Ov{w}) \\
  &\divg(\sigma_T \nabla u_T) = 0 
\end{empheq}

On définit à présent les nombres sans dimensions suivants: 
\[
\alpha = AC \frac{x_0^2}{t_0 \sigma_0}, \quad \beta = \frac{f_0 t_0}{C \delta u}, \quad \varepsilon = \frac{h}{x_0}
\]

Comme on considère une couche de tissu cardiaque, le ratio $\varepsilon = \frac{h}{x_0}$ est supposé petit. On arrive finalement aux équations bidomaine suivantes:
\begin{empheq}[left=\empheqlbrace]{align*}
  \label{eq:adim}
  &\divg_{\overline{x}}((\overline{\sigma}_i+\overline{\sigma}_e) \nabla_{\overline{x}} \overline{u}_e) 
  + \frac{1}{\varepsilon^2} \overline{\sigma}_{(i+e)}^{(3)} \partial_{\overline{z} \overline{z}} \overline{u}_e
  = -\divg_{\overline{x}}(\overline{\sigma}_i \nabla \overline{v})
  + \frac{1}{\varepsilon^2} \overline{\sigma}_{i}^{(3)} \partial_{\overline{z} \overline{z}} \overline{v}\\
  & \alpha (\partial_{\overline{t}} \overline{v} 
  + \beta \overline{f}(\overline{v},\overline{w})) 
  = \divg_{\overline{x}}(\overline{\sigma}_i \nabla(\overline{u}_e + \overline{v})) 
  + \frac{1}{\epsilon^2} \overline{\sigma}_{i}^{(3)} \partial_{\overline{z} \overline{z}} \overline{v}  \\
  &\partial_{\Ov{t}} \Ov{w} = g(\Ov{v},\Ov{w}) \\
  &\divg(\sigma_T \nabla u_T) = 0 
\end{empheq}


\section{Modèle asymptotique}
\label{sec:modele-asymptotique}
Nous rappelons le problème tri-dimensionnel adimensionné, en éliminant les $\Ov{~\cdot~}$ au dessus des quantités sans dimensions:

\begin{empheq}[left=\empheqlbrace]{align}
  \label{eq:adim_bid}
  &\divg_{x}((\sigma_i+\sigma_e) \nabla_{x} {u}_e) 
  + \frac{1}{\varepsilon^2} {\sigma}_{(i+e)}^{(3)} \partial_{z z} {u}_e
  = -\divg_{x}({\sigma}_i \nabla v)
  + \frac{1}{\varepsilon^2} \sigma_{i}^{(3)} \partial_{z z} v\\
  & \alpha (\partial_{t} v 
  + \beta f(v,w)) 
  = \divg_{x}(\sigma_i \nabla(u_e + v)) 
  + \frac{1}{\epsilon^2} \sigma_{i}^{(3)} \partial_{z z} v  \\
  &\partial_{t} w = g(v,w) \\
  &\divg(\sigma_T \nabla u_T) = 0 
\end{empheq}


%%% Local Variables:
%%% mode: latex
%%% TeX-master: "main_bidomain"
%%% End:
