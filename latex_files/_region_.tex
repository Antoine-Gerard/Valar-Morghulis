\message{ !name(main_bidomain.tex)}\documentclass[a4paper]{report}
\usepackage[utf8]{inputenc}
\usepackage[T1]{fontenc}
\usepackage[french]{babel}
\usepackage{graphicx}
\usepackage{float}
\usepackage{amsmath}
\usepackage{amsfonts}
\usepackage[overload]{empheq} 
\usepackage{lmodern}
\usepackage{amsthm}
\usepackage{pdfpages}
\usepackage{a4wide}
\usepackage[showlabels,sections,floats,textmath,displaymath]{preview}
\usepackage{subfigure}
\usepackage{hyperref}
\usepackage{algorithm}
\usepackage{algorithmic}
\usepackage{fullwidth}

\newtheorem{mydef}{Définition}[chapter]
\newtheorem*{Rem}{Remarque}
\newtheorem{lemme}{Lemme}[chapter]

\DeclareMathOperator{\divg}{div} %operateur divergence
%\DeclareMathOperator{\exp}{exp} %exponentielle



\newcommand{\norm}[1]{\left \Vert {#1} \right \Vert}
\begin{document}

\message{ !name(FirsPart_AABidomain.tex) !offset(-33) }
\section{Analyse asymptotique bidomaine: Coeur + Torse}

On considère une pièce de tissu cardiaque idéalisée $\Omega$ ayant une épaisseur $h>0$. On définit $\mathbb{H}\ = \omega \times (0,h) \subset \mathbb{R}^3$, où $\omega$ est un ensemble ouvert de $\mathbb{R}^2$. De plus nous faisons l'hypothèse que $\Omega \subset \mathcal{E}$ et que $\mathbb{T} = \mathcal{E} - \mathbb{\overline{H}}$  


%%% Local Variables:
%%% mode: latex
%%% TeX-master: "main_bidomain"
%%% End:

\message{ !name(main_bidomain.tex) !offset(-11) }

\end{document}

%%% Local Variables:
%%% mode: latex
%%% TeX-master: t
%%% End:
